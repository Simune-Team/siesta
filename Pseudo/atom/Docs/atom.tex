% Manual for the ATOM program
%
% To generate the printed version:
%
% latex manual
% makeindex manual    (Optional if you have a current manual.ind)
% latex manual
% [ dvips manual ]
%
%
\documentclass[11pt]{article}
%\usepackage{makeidx}
%\usepackage{graphicx}

\tolerance 10000
\textheight 24cm
\textwidth 16cm
\oddsidemargin 1mm
\topmargin -20mm

%\makeindex

\parindent=0cm
\baselineskip=12pt
\parskip 5pt

\newcommand{\gpfig}[2][1.0]{
\includegraphics[angle=-90,width=#1\textwidth] {#2}
}

\newcommand{\singlefig}[2][1.0]{
\begin{figure}
\centering
\gpfig[#1]{#2}
\end{figure}
}

\begin{document}

% TITLE PAGE
% --------------------------------------------------------------

\begin{titlepage}

\begin{center}
~
\vfill
\vspace{1cm}
{\Huge {\bf ATOM User Manual}}
\par\vspace{3cm}
\hrulefill
\par\vspace{3cm}
{\Large {\bf Version 3.3.1, 18 June 2010}}
\par\vspace{2cm}
\hrulefill

{\Large Alberto Garc\'{\i}a \\
ICMAB-CSIC, Barcelona

albertog@icmab.es}
\vfill
\end{center}

\end{titlepage}
% END TITLE PAGE
% --------------------------------------------------------------

\tableofcontents
\newpage

\section{PREFACE}

{\sc ATOM} is the name of a program originally written (circa 1982) by
Sverre Froyen at the University of California at Berkeley, modified
starting in 1990 by Norman Troullier and Jose Luis Martins at the
University of Minnesota, and currently maintained by Alberto Garcia
(wdpgaara@lg.ehu.es), who added some features and made substantial
structural changes to the April 1990 (5.0) Minnesota version while at
Berkeley and elsewhere.

Jose Luis Martins is maintaining his own version of the code:
\begin{verbatim}{\tt http://bohr.inesc-mn.pt/~jlm/pseudo.html\end{verbatim}


The program's basic capabilities are:

\begin{itemize}
\item All-electron DFT atomic calculations for arbitrary electronic
configurations.

\item Generation of ab-initio pseudopotentials (several flavors).

\item Atomic calculations in which the effect of the core is represented
by a previously generated pseudopotential. These are useful to make
sure that the pseudopotential correctly reproduces the all-electron
results for the valence complex.

\end{itemize}



\section{A PRIMER ON AB-INITIO PSEUDOPOTENTIALS}

Time constraints prevent the inclusion of this section in this first
release of the {\sc ATOM} manual. But, in this case more than ever,
there is a lot to be gained from reading the original literature...
Here are some basic references:

\begin{itemize}
\item Original idea of the ab-initio pseudopotential:

Kerker, J. Phys. C 13, L189-94 (1980)\\
Hamann, Schluter, Chiang, Phys. Rev. Lett. 43, 1494 (1979)

\item More on HSC scheme:

Bachelet, Schluter, Phys. Rev. B 25, 2103 (1982)\\
Bachelet, Hamann, Schluter, Phys. Rev. B 26, 4199 (1982)

\item Troullier-Martins elaboration of Kerker method:

Troullier, Martins, Phys. Rev. B 43, 1993 (1991)\\
Troullier, Martins, Phys. Rev. B 43, 8861 (1991)

\item Core corrections:

Louie, Froyen, Cohen, Phys. Rev. B 26, 1738 (1982)

\item The full picture of plane-wave pseudopotential ab-initio calculations:

W. E. Pickett, ``Pseudopotential Methods in Condensed Matter
Applications'', Computer Physics Reports 9, 115 (1989)

M. C. Payne, M. P. Teter, D. C. Allan, T. A. Arias and
J. D. Joannopoulos, ``Iterative minimization techniques for ab initio
total-energy calculations: molecular dynamics and conjugate
gradients'', Rev. Mod. Phys. 64, 1045, (1992)

Also, the book by Richard Martin
``Electronic Structure: Basic Theory and Practical Methods''
(Cambridge University Press) has a chapter on pseudopotentials.

\item Use in {\sc SIESTA}:

J.M. Soler, E. Artacho, J.D. Gale, A. Garcia, J. Junquera, P. Ordejon,
D. Sanchez-Portal, ``The SIESTA method for ab initio O(N)
materials simulation'', Jour. Phys.: Condens. Matter, 14, 2745-2779
(2002). 

\end{itemize}

\section{COMPILING THE PROGRAM}

(Please note that {\sc ATOM} now depends on the SiestaXC and FoX libraries
for its correct compilation. It is not currently possible to have
a ``standalone'' version independent of the rest of the Siesta package.)

The Fortran compiler and auxiliary file settings are specified in the
appropriate {\tt arch.make} file in the Siesta compilation directory.

Just type {\tt make} in the main {\sc ATOM} source directory ({\tt
Pseudo/atom}). After a short while you will have the executable (called
{\tt atm}) in that same directory. The program should work for any atom
without recompilation. 

Directory {\tt Tutorial} in the source distribution contains a set of
scripts to automate the process of running {\sc ATOM} and to analyze
the results. The file-manipulation details involved in each of the
basic functions of all-electron calculations, generation of
pseudopotentials, and testing of the pseudopotentias, are taken care
of by {\tt ae.sh}, {\tt pg.sh}, and {\tt pt.sh}, respectively, all in
the {\tt Tutorial/Utils} directory.  These
scripts need to know where the {\sc ATOM} executable {\tt atm}
is. If you have moved the {\tt Tutorial} directory around, or you do
not have the source, the default location might not be right for you.
The easiest way to fix it is to define an environmental
variable {\tt ATOM\_PROGRAM}. Assuming {\tt atm} is in {\tt
/somedir/somewhere}, you would do:
%
\begin{verbatim}
ATOM_PROGRAM=/somedir/somewhere/atm ; export ATOM_PROGRAM  # sh-derived shells
setenv ATOM_PROGRAM /somedir/somewhere/atm                 # csh-derived shells
\end{verbatim}
%

Due to the shortcommings of the basic (GNUplot) plotting package used in the
Tutorial section, it is also necessary to copy some scripts from a
central repository. Again, if the default does not work for you,
define the {\tt ATOM\_UTILS\_DIR} variable:

\begin{verbatim}
ATOM_UTILS_DIR=/somewhere ; export ATOM_UTILS_DIR  # sh-derived shells
setenv ATOM_UTILS_DIR /somewhere                   # csh-derived shells
\end{verbatim}
%

\section{USING THE ATOM PROGRAM}


\subsection{All-electron calculations}

Assume we want to find the orbital eigenvalues, total energy, and/or
charge density of Si in its ground state. (You should now go to the
{\tt Tutorial/All\_electron} directory and try the following.)  Our
input file is named {\tt si.ae.inp} and contains the lines (see 
Sect.~\ref{sec:inputfile} for more details):

\begin{verbatim}
   ae Si ground state all-electron
   Si   ca
       0.0
    3    2
    3    0      2.00      0.00
    3    1      2.00      0.00

#2345678901234567890123456789012345678901234567890      Ruler
\end{verbatim}

We can run the calculation by using the {\tt ae.sh} script. Following
the layout of the {\tt Tutorial} directory, we will assume that the
script is in the {\tt Tutorial/Utils} directory. We run the
script and go into the directory created for the calculation (named
as the input file without the extension {\tt .inp}):

\begin{verbatim}
$ sh ../Utils/ae.sh si.ae.inp
==> Output data in directory si.ae
$ cd si.ae
$ ls
AECHARGE  CHARGE  RHO       charge.gplot   vcharge.gps
AEWFNR0   INP     ae.gplot  charge.gps     vspin.gplot
AEWFNR1   OUT     ae.gps    vcharge.gplot  vspin.gps
$
\end{verbatim}

We see some data files (those in all caps) and a few GNUPLOT plotting
scripts\footnote{GNUPLOT is not a publication-quality package, and
suffers from serious shortcomings, but it is free, and installed
almost everywhere. Hence we have chosen it as the lowest-common
denominator for basic plotting} . 

The files are:

\begin{itemize}
\item {\tt INP:} A copy of the input file for the calculation.
\item {\tt OUT:} Contains detailed information about the run. 



\item  {\tt AECHARGE:} Contains in four
columns values of $r$, the ``up'' and ``down'' parts of the {\sl
total} charge density, and the total core charge density (the charges
multiplied by $4\pi r^2$). {\tt CHARGE} is exactly identical and is
generated for backwards compatibility.

\item {\tt RHO:} Like {\tt CHARGE}, but without the $4\pi r^2$ factor.
\item {\tt AEWFNR0...AEWFNR3:} All-electron valence wavefunctions as function
of radius, for $s$, $p$, $d$, and $f$ valence orbitals (0, 1, 2, 3,
respectively --- some channels might not be available). They include 
a factor of $r$, the $s$ orbitals also going to zero at the origin.

\end{itemize}

It is interesting to peruse the {\tt OUT} file.
In particular, it lists the orbital eigenvalues (in Rydbergs, as
every other energy in the program):

\begin{verbatim}
 nl    s      occ         eigenvalue    kinetic energy      pot energy

 1s   0.0    2.0000    -130.36911241     183.01377616    -378.73491463
 2s   0.0    2.0000     -10.14892694      25.89954259     -71.62102169
 2p   0.0    6.0000      -7.02876268      24.42537874     -68.74331203
 3s   0.0    2.0000      -0.79662742       3.23745215     -17.68692611
 3p   0.0    2.0000      -0.30705179       2.06135782     -13.62572515
\end{verbatim}

(For a relativistic or spin-polarized calculation, there would be
``up'' and ``down'' flags in the {\tt s} column above.)


The {\bf plotting} scripts come in two flavors: {\tt .gplot} for
terminal use (default X11, use {\tt gnuplot -persist}), and {\tt .gps}
for postscript output.  

For all-electron calculations, the relevant scripts (without {\tt
.gplot} or {\tt .gps} extensions) are:

\begin{itemize}
\item {\tt charge:} Charge density (separated core and valence
contributions, multiplied by $4\pi r^2$). 
\item {\tt vcharge:} Valence charge density (same normalization).
\item {\tt ae:} Orbital valence wavefunctions (radial part multiplied by $r$)
\end{itemize}

\subsection{Pseudopotential generation}

(You should now go to the {\tt Tutorial/Si} directory and try the
following.)  We are going to generate a pseudopotential for Si, using
the Troullier-Martins scheme. The calculation is relativistic and we
use the LDA (Ceperley-Alder flavor). The input file is named {\tt
Si.tm2.inp} and contains the lines (see Sect.~\ref{sec:inputfile} for more
details):

\begin{verbatim}
#
#  Pseudopotential generation for Silicon
#  pg: simple generation
#
   pg      Silicon
        tm2      3.0             # PS flavor, logder R
 n=Si c=car                      # Symbol, XC flavor,{ |r|s}
       0.0       0.0       0.0       0.0       0.0       0.0
    3    4                       # norbs_core, norbs_valence
    3    0      2.00      0.00   # 3s2
    3    1      2.00      0.00   # 3p2
    3    2      0.00      0.00   # 3d0
    4    3      0.00      0.00   # 4f0
      1.90      1.90      1.90      1.90      0.00      0.00
#
# Last line (above): 
#    rc(s)     rc(p)     rc(d)     rc(f)   rcore_flag  rcore
#
#23456789012345678901234567890123456789012345678901234567890
\end{verbatim}

Note the two extra lines with respect to an all-electron calculation.
The pseudopotential core radii for all channels are 1.90 bohrs. Even
though they are nominally empty in the ground state, we include the
$3d$ and $4f$ states in order to generate the corresponding
pseudopotentials. 

We can run the calculation by using the {\tt pg.sh} script. Following
the layout of the {\tt Tutorial} directory, we will assume that the
script is in the {\tt Tutorial/Utils} directory. We run the
script and go into the directory created for the calculation (named
as the input file without the extension {\tt .inp}):

\begin{verbatim}
$ sh ../../Utils/pg.sh Si.tm2.inp
==> Output data in directory Si.tm2
==> Pseudopotential in Si.tm2.vps and Si.tm2.psf
$ cd Si.tm2
$ ls [A-Z]*    # show only the data filesAE
CHARGE  AEWFNR3   PSLOGD3  PSPOTR3  PSWFNR3     
AELOGD0   CHARGE    PSPOTQ0  PSWFNQ0  RHO         
AELOGD1   INP       PSPOTQ1  PSWFNQ1  SCRPSPOTR0  
AELOGD2   OUT       PSPOTQ2  PSWFNQ2  SCRPSPOTR1  
AELOGD3   PSCHARGE  PSPOTQ3  PSWFNQ3  SCRPSPOTR2  
AEWFNR0   PSLOGD0   PSPOTR0  PSWFNR0  SCRPSPOTR3  
AEWFNR1   PSLOGD1   PSPOTR1  PSWFNR1  VPSFMT      
AEWFNR2   PSLOGD2   PSPOTR2  PSWFNR2  VPSOUT      
\end{verbatim}

There are quite a few data files now. The new ones are:
\begin{itemize}

\item  {\tt PSCHARGE:} Contains in four
columns values of $r$, the ``up'' and ``down'' parts of the {\sl
pseudo valence} charge density, and the pseudo core charge density
(see Sect.~\ref{sec:cc}) (the charges
multiplied by $4\pi r^2$).

\item {\tt PSWFNR0...PSWFNR3:} Valence pseudowavefunctions as function
of radius, for $s$, $p$, $d$, and $f$ valence orbitals (0, 1, 2, 3,
respectively --- some channels might not be available). They include 
a factor of $r$, the $s$ orbitals also going to zero at the origin.

\item {\tt PSPOTR0...PSPOTR3:} Ionic pseudopotentials (i.e. unscreened)
as a function of $r$, for $s$, $p$, $d$, and $f$ channels (0, 1, 2, 3,
respectively --- some channels might not be available). The last
column is $-2Z_{ps}/r$, that is, the Coulomb potential of the pseudo
atom. All the ionic pseudopotentials tend to this Coulomb tail for $r$
beyond the range of the core electrons.

\item {\tt PSPOTQ0...PSPOTQ3:} Fourier transform $V(q)$ (times
$q^2/Z_{ps}$) of the ionic pseudopotentials as a function of $q$ (in
bohr$^{-1}$), for $s$, $p$, $d$, and $f$ channels (0, 1, 2, 3,
respectively --- some channels might not be available).

\item {\tt PSWFNQ0...PSWFNQ3:} Fourier transform $\Psi(q)$ of the 
valence pseudowavefunctions as a function of $q$ (in
bohr$^{-1}$), for $s$, $p$, $d$, and $f$ channels (0, 1, 2, 3,
respectively --- some channels might not be available).

\item {\tt VPSOUT, VPSFMT:} Files (formatted and unformatted)
containing pseudopotential information. They are used for {\it
ab-initio} codes such as {\sc SIESTA} and {\sc PW}. Copies of these
files are deposited in the top directory after the run.

\end{itemize}

The {\tt OUT} file has two sections, one for the all-electron (AE) run, and
another for the pseudopotential (PS) generation itself. It is instructive to
compare the AE and PS eigenvalues. Simply do

\begin{verbatim}
$ grep '&v' OUT
 ATM3      12-JUL-02        Silicon
 3s   0.5    2.0000      -0.79937161       0.00000000     -17.74263363
 3p  -0.5    0.6667      -0.30807129       0.00000000     -13.66178958
 3p   0.5    1.3333      -0.30567134       0.00000000     -13.60785822
 3d  -0.5    0.0000       0.00000000       0.00000000      -0.27407047
 3d   0.5    0.0000       0.00000000       0.00000000      -0.27407047
 4f  -0.5    0.0000       0.00000000       0.00000000      -0.26482365
 4f   0.5    0.0000       0.00000000       0.00000000      -0.26482365
---------------------------- &v
 3s   0.5    2.0000      -0.79936061       0.50555315      -3.74113059
 3p  -0.5    0.6667      -0.30804995       0.77243805      -3.26356669
 3p   0.5    1.3333      -0.30565760       0.76702460      -3.25197500
 3d  -0.5    0.0000       0.00000000       0.00140505      -0.07847269
 3d   0.5    0.0000       0.00000000       0.00140505      -0.07847269
 4f  -0.5    0.0000       0.00000000       0.00243411      -0.07586534
 4f   0.5    0.0000       0.00000000       0.00243411      -0.07586534
---------------------------- &v
\end{verbatim}

(The AE and PS eigenvalues are not {\sl exactly} identical because the
pseudopotentials are changed slightly to make them approach their
limit tails faster).

The relevant plotting scripts (without {\tt .gplot} or {\tt .gps}
extensions) are:

\begin{itemize}
\item {\tt charge:} It compares the AE and PS charge densities.
\item {\tt pseudo:} A multi-page plot showing, on one page/window per
channel:
\begin{itemize}
\item The AE and PS wavefunctions
\item The AE and PS logarithmic derivatives.
\item The real-space pseudopotential
\item The Fourier-transformed pseudopotential (times $q^2/Z_{ps}$)
\end{itemize}
\item {\tt pots:} All the real-space pseudopotentials.
\end{itemize}

\subsubsection{Core Corrections}
\label{sec:cc}
The program can generate pseudopotentials with the non-linear
exchange-correlation correction proposed in S. G. Louie, S. Froyen,
and M. L. Cohen, Phys. Rev. B 26, 1738 (1982).

In the traditional approach (which is the default for LDA
calculations), the pseudocore charge density equals the charge density
outside a given radius $r_{pc}$, and has the smooth form
$$
\rho_{pc}(r) = A r   \sin(b r)
$$
inside that radius. A smooth matching is provided with suitable $A$ 
and $b$ parameters calculated by the program.

A new scheme has been implemented to fix some problems in the generation
of GGA pseudopotentials. The smooth function is now
$$
\rho_{pc}(r) =  r^2  \exp{(a + b r^2 +c r^4)}
$$
and derivatives up to the second are continuous  at $r_{pc}$.

To use core corrections in the pseudopotential generation
the jobcode in the first line should be {\tt pe} instead of {\tt pg}.

The radius $r_{pc}$ should be  given in the sixth slot in the last
input line (see above). If it is negative or zero (or blank), the
radius is then computed using the fifth number in that line ({\tt
rcore\_flag}, see the example input file above)
and the following criterion: at $r_{pc}$ the core charge density 
equals {\tt rcore\_flag}*(valence charge density).
It is {\it highly recommended} to set an explicit value for the pseudocore
radius $r_{pc}$, rather than letting the program provide a default.

If {\tt rcore\_flag} is input as negative, the full core charge is used.
If {\tt rcore\_flag} is input as zero, it is set equal to one, which will be
thus the default if {\tt pe} is given but no numbers are given for these
two variables.

The output file contains the radius used and the $A$ ($a$) and $b$ (and $c$)
parameters used for the matching. The {\tt VPSOUT} and {\tt VPSFMT}
files will  contain the pseudocore charge in addition to the pseudopotential.

It is possible to override the default (new scheme for GGA
calculations, old scheme for LDA calculations) by using the directives
\begin{verbatim}
%define NEW_CC
%define OLD_CC
\end{verbatim}
The program will issue the appropriate warnings. (See Sect.~\ref{sec:inputfile})

Relevant files:

\begin{itemize}
\item  {\tt PSCHARGE:} Contains the pseudocore charge in column four.
(multiplied by $4\pi r^2$).

\item {\tt COREQ:} Fourier transform of the pseudocore charge density
$\rho_{pc}(q)$ in units of electrons, with $q$ in  bohr$^{-1}$.
\end{itemize}

Useful plotting scripts (without {\tt .gplot} or {\tt .gps}
extensions) are:

\begin{itemize}
\item {\tt charge:} Shows also the pseudocore charge.
\item {\tt coreq:}  Shows the Fourier transform of the pseudocore charge.
\end{itemize}

\subsection{Pseudopotential test}

While it is helpful to ``have a look'' at the plots of the
pseudopotential generation to get a feeling for its quality, there is
no substitute for a proper {\bf transferability testing}. A
pseudopotential with good transferability will reproduce the
all-electron energy levels and wavefunctions in arbitrary
environments, (i.e., in the presence of charge transfer, which always
takes place when forming solids and molecules).  We know that norm
conservation guarantees a certain degree of transferability (usually
seen clearly in the plot of the logarithmic derivative), but we can
get a better assessment by performing all-electron and ``pseudo''
calculations on the same series of atomic configurations, and comparing
the eigenvalues and excitation energies.

In the same {\tt Tutorial/Si} directory we can find file {\tt
Si.test.inp}, containing the concatenation of ten jobs. The first
five are all-electron ({\tt ae}) calculations, and the last five,
pseudopotential test ({\tt pt}) runs for the same configurations:
%
\begin{verbatim}
#
# All-electron calculations for a series of Si configurations
#
   ae Si Test -- GS 3s2 3p2
   Si   ca
       0.0
    3    2
    3    0      2.00
    3    1      2.00
   ae Si Test -- 3s2 3p1 3d1
   Si   ca
       0.0
    3    3
    3    0      2.00
    3    1      1.00
    3    2      1.00
   ae Si Test -- 3s1 3p3
   Si   ca
       0.0
    3    2
    3    0      1.00
    3    1      3.00
   ae Si Test -- 3s1 3p2 3d1
   Si   ca
       0.0
    3    3
    3    0      1.00
    3    1      2.00
    3    2      1.00
   ae Si Test -- 3s0 3p3 3d1
   Si   ca
       0.0
    3    3
    3    0      0.00
    3    1      3.00
    3    2      1.00
#
# Pseudopotential test calculations
#
   pt Si Test -- GS 3s2 3p2
   Si   ca
       0.0
    3    2
    3    0      2.00
    3    1      2.00
   pt Si Test -- 3s2 3p1 3d1
   Si   ca
       0.0
    3    3
    3    0      2.00
    3    1      1.00
    3    2      1.00
   pt Si Test -- 3s1 3p3
   Si   ca
       0.0
    3    2
    3    0      1.00
    3    1      3.00
   pt Si Test -- 3s1 3p2 3d1
   Si   ca
       0.0
    3    3
    3    0      1.00
    3    1      2.00
    3    2      1.00
   pt Si Test -- 3s0 3p3 3d1
   Si   ca
       0.0
    3    3
    3    0      0.00
    3    1      3.00
    3    2      1.00
\end{verbatim}

The configurations differ in the promotion of electrons from one level
to another (it is also possible to transfer {\sl fractions} of an
electron). 

We can run the file by using the {\tt pt.sh} script. Following
the layout of the {\tt Tutorial} directory, we will assume that the
script is in the directory directly above the current one. We need to
give it {\bf two} arguments: the calculation input file, and the
file containing the pseudopotential we want to test. Let's make the
latter {\tt Si.tm2.vps}:

\begin{verbatim}
$ sh ../../Utils/pt.sh Si.test.inp Si.tm2.vps
==> Output data in directory Si.test-Si.tm2
$ cd Si.test-Si.tm2/
$ ls [A-Z]*
AECHARGE  AEWFNR1  CHARGE  OUT       PTWFNR0  PTWFNR2  VPSIN
AEWFNR0   AEWFNR2  INP     PTCHARGE  PTWFNR1  RHO
\end{verbatim}

The working directory is named after {\sl both} the test and
pseudopotential files. It contains several new files:

\begin{itemize}
\item {\tt VPSIN:} A copy of the pseudopotential file to be tested.
\item  {\tt PTCHARGE:} Contains in four
columns values of $r$, the ``up'' and ``down'' parts of the {\sl
pseudo valence} charge density, and the pseudo core charge density
(see Sect.~\ref{sec:cc}) (the charges
multiplied by $4\pi r^2$).

\item {\tt PTWFNR0...PTWFNR3:} Valence pseudowavefunctions as function
of radius, for $s$, $p$, $d$, and $f$ valence orbitals (0, 1, 2, 3,
respectively --- some channels might not be available). They include 
a factor of $r$, the $s$ orbitals also going to zero at the origin.

\end{itemize}

The {\tt OUT} file has two sections, one for the all-electron (AE) runs, and
another for the pseudopotential tests (PT). At the end of each series
of runs there is a table showing the excitation energies. A handy way
to compare the AE and PT energies is:
\begin{verbatim}
$ grep '&d' OUT
[...elided...]
 &d total energy differences in series
 &d          1        2        3        4        5
 &d  1    0.0000
 &d  2    0.4308   0.0000
 &d  3    0.4961   0.0653   0.0000
 &d  4    0.9613   0.5305   0.4652   0.0000
 &d  5    1.4997   1.0689   1.0036   0.5384   0.0000
*----- End of series ----* spdfg &d&v
 ATM3      12-JUL-02   Si Test -- GS 3s2 3p2
 ATM3      12-JUL-02   Si Test -- 3s2 3p1 3d1
 ATM3      12-JUL-02   Si Test -- 3s1 3p3
 ATM3      12-JUL-02   Si Test -- 3s1 3p2 3d1
 ATM3      12-JUL-02   Si Test -- 3s0 3p3 3d1
 &d total energy differences in series
 &d          1        2        3        4        5
 &d  1    0.0000
 &d  2    0.4299   0.0000
 &d  3    0.4993   0.0694   0.0000
 &d  4    0.9635   0.5336   0.4642   0.0000
 &d  5    1.5044   1.0745   1.0051   0.5409   0.0000
*----- End of series ----* spdfg &d&v
\end{verbatim}

The tables (top AE, bottom PT) give the cross-excitations among all
configurations. Typically, one should be all right if the AE-PT
differences are not much larger than 1~mRy.

You can also compare the AE and PT eigenvalues. Simply do

\begin{verbatim}
$ grep '&v' OUT | grep s
 ATM3      12-JUL-02   Si Test -- GS 3s2 3p2
 3s   0.0    2.0000      -0.79662742       3.23745215     -17.68692611
 ATM3      12-JUL-02   Si Test -- 3s2 3p1 3d1
 3s   0.0    2.0000      -1.08185979       3.53885995     -18.40569836
 ATM3      12-JUL-02   Si Test -- 3s1 3p3
 3s   0.0    1.0000      -0.85138783       3.35438895     -17.96219240
 ATM3      12-JUL-02   Si Test -- 3s1 3p2 3d1
 3s   0.0    1.0000      -1.11431855       3.62997498     -18.60814708
 ATM3      12-JUL-02   Si Test -- 3s0 3p3 3d1
 3s   0.0    0.0000      -1.14358268       3.71462770     -18.79448684
*----- End of series ----* spdfg &d&v
 ATM3      12-JUL-02   Si Test -- GS 3s2 3p2
 1s   0.0    2.0000      -0.79938037       0.50556261      -3.74114712
 ATM3      12-JUL-02   Si Test -- 3s2 3p1 3d1
 1s   0.0    2.0000      -1.08384468       0.55070398      -3.81988817
 ATM3      12-JUL-02   Si Test -- 3s1 3p3
 1s   0.0    1.0000      -0.85392666       0.52020429      -3.76852577
 ATM3      12-JUL-02   Si Test -- 3s1 3p2 3d1
 1s   0.0    1.0000      -1.11546463       0.56048425      -3.83646615
 ATM3      12-JUL-02   Si Test -- 3s0 3p3 3d1
 1s   0.0    0.0000      -1.14353959       0.56945741      -3.85106049
*----- End of series ----* spdfg &d&v
\end{verbatim}

(and similarly for $p$, $d$, and $f$, if desired). Again, the typical
difference should be of around 1~mRyd for a ``good'' pseudopotential.
(The {\sl real\/} proof of good transferability, remember, can only come
from a molecular or solid-state calculation). Note that the PT levels
are labeled starting from principal quantum number 1. 

The relevant plotting scripts (without {\tt .gplot} or {\tt .gps}
extensions) are:

\begin{itemize}
\item {\tt charge:} It compares the AE and PT charge densities.
\item {\tt pt:} Compares the valence all-electron and pseudo-wavefunctions.
\end{itemize}



\section{APPENDIX: THE INPUT FILE}
\label{sec:inputfile}

For historical reasons, the input file is in a rigid column
format. Fortunately, most of the column fields line up, so the
possibility of errors is reduced.  We will begin by describing in
detail a very simple input file for an {\bf all-electron calculation} for
the ground state of Si. More examples can be found in the {\tt
Tutorial} directory.

The file itself is:
\begin{verbatim}
#
#  Comments allowed here
#
   ae Si ground state all-electron
   Si   car  
       0.0
    3    2
    3    0      2.00      0.00
    3    1      2.00      0.00
#
# Comments allowed here
#
#2345678901234567890123456789012345678901234567890      Ruler
\end{verbatim}

\begin{itemize}
\item The first line specifies:
	\begin{itemize}
	\item The calculation code ({\tt ae} here stands for ``all-electron'').
	\item A title for the job (here {\tt Si ground state all-electron}).
	\end{itemize} 
	(format 3x,a2,a50)

\item Second line:
	\begin{itemize}

	\item Chemical symbol of the nucleus (here {\tt Si}, obviously)
	\item Exchange-correlation type. Here, {\tt ca} stands for
          Ceperley-Alder. Other options are {\tt wi}
          (Wigner), {\tt hl} (Hedin-Lundqvist), {\tt gl}
          (Gunnarson-Lundqvist), and {\tt bh} (von Barth-Hedin). The
          ``best'' LDA choice should be {\tt ca}.  It is also possible
          to use a gradient-corrected functional: {\tt pb} indicates use
          of the PBE scheme by Perdew, Burke, and Ernzerhof (PRL 77,
          3865 (1996)),{\tt rp} indicates the RPBE
          scheme by Hammer, Hansen, and Norskov, (PRB 59, 7413 (1999)),
          {\tt rv} indicates the revPBE
          scheme by Zhang and Yang, (PRL 80,890(1998)), 
          {\tt wc} indicates the WC
          scheme by Z. Wu and R.E. Cohen (PRB 73, 235116 (2006)),
          {\tt bl} selects the BLYP (Becke-Lee-Yang-Parr) scheme,
          {\tt am} indicates the AM05 scheme by R. Armiento and A.E. Mattsson 
          (PRB 72, 085108 (2005), PRB 79, 155101 (2009)),
          and {\tt ps} selects the PBEsol scheme by Perdew et al (PRL 100, 136406 (2008))
          (see source file {\tt xc.f} for more details).
          To use the van der Waals functional of M. Dion et al, PRL 92, 246401
          (2004) (as implemented by G. Rom\'an-P\'erez and J. M. Soler,
          PRL 103, 096102 (2009)), set the string to {\tt vw} or {\tt
            vf}.
          Another version of the vdW functional, that of Lee et
          al. (arXiv:1003.5255v1 (2010)) (LMKLL) can be enabled by using
          {\tt vl},
          and the Klimes et al. parametrization (JPCM 22, 022201
          (2009)) (KBM)
          by using {\tt vk}.
	  By default, the exchange-correlation energy and potential
	  are computed using the Balb\'as-Martins-Soler package 
	  (PRB 64, 165110 (2001)), except for
	  the {\tt wi}, {\tt hl}, {\tt gl}, and {\tt bh} flavors. 
   
	\item The character {\tt r} next to {\tt ca} is a flag to perform the
          calculation relativistically, that is, solving the Dirac equation
          instead of the Schrodinger equation. 
	  The full range of options is:
	    \begin{itemize}
		\item {\tt s} : Spin-polarized calculation, non-relativistic.
		\item {\tt r}: Relativistic calculation, obviously polarized.
		\item (blank) : Non-polarized (spin ignored), non-relativistic
           		calculation.
	    \end{itemize}
	\end{itemize}
	
	(format 3x,a2,3x,a2,a1,2x)

\item Third line. Its use is somewhat esoteric and for most
  calculations it should contain just a 0.0 in the position shown, but
  that first field might be useful to generate pseudopotentials for
  ``atoms'' with a fractional atomic number (see the example for ON in
  the {\tt Tutorial/PS\_Generation} directory).
\end{itemize}

The rest of the file is devoted to the specification of the electronic
configuration:

\begin{itemize} 

\item Fourth line:\\
	 Number of core and valence orbitals. For example, for Si, we
	have $1s$, $2s$, and $2p$ in the core (a total of 3 orbitals), and
	$3s$ and $3p$ in the valence complex (2 orbitals).

	(format 2i5)

\item Fifth, sixth... lines: (there is one line for each valence
orbital)
	\begin{itemize}
	\item {\tt n} (principal quantum number)
	\item {\tt l} (angular momentum quantum number)
	\item Occupation of the orbital in electrons. 
	\end{itemize}

	(format 2i5,2f10.3)

	(There are two f input descriptors to allow the input of ``up''
	and ``down'' occupations in spin-polarized calculations (see
	example below))

\end{itemize}

Comments or blank lines may appear in the file at the beginning and at the end.
It is possible to perform two or more calculations in
succession by simply concatenating blocks as the one described above.
For example, the following file is used to study the ground state of N
and an excited state with one electron promoted from the $2s$ to the $2p$
orbital taking into account the spin polarization:

\begin{verbatim}
#
   ae N ground state all-electron
   N    cas
       0.0
    1    2
    2    0      2.00      0.00
    2    1      3.00      0.00
#
#  Second calculation starts here
#
   ae N 1s2 2s1 2p4  all-electron
   N    cas
       0.0
    1    2
    2    0      1.00      0.00
    2    1      3.00      1.00

#2345678901234567890123456789012345678901234567890      Ruler
\end{verbatim}

	
The different treatment of core and valence orbitals in the input for an
all-electron calculation is purely cosmetic. The program ``knows'' how
to fill the internal orbitals in the right order, so it is only
necessary to give their number. That is handy for heavy atoms...
Overzealous users might want to check the output to make sure that the
core orbitals are indeed correctly treated.



For a {\bf pseudopotential test calculation}, the format is exactly
the same, except that the job code is {\tt pt} instead of {\tt ae}. 


For a {\bf pseudopotential generation run}, in addition to the
electronic configuration chosen for the generation of the
pseudopotentials (which is input in the same manner as above), one has
to specify the ``flavor'' (generation scheme) and the set of core
radii $r_c$ for the construction of the pseudowavefunction. Here is an
example for Si using the Hamann-Schluter-Chiang scheme:

\begin{verbatim}
# 
   pg Si Pseudopotencial
        hsc     2.00
   Si   ca
         0
    3    3
    3    0      2.00
    3    1      0.50
    3    2      0.50
      1.12      1.35      1.17       0.0       0.0       0.0
#
#23456789012345678901234567890123456789012345678901234567890   Ruler
---------------------------------------
\end{verbatim}

Apart from the {\tt pg} (pseudopotential generation) job code in the
first line, there are two extra lines:

\begin{itemize}
\item Second line: 

Flavor and radius at which to compute logarithmic
derivatives for test purposes. 

The flavor can be one of :
\begin{tabular}{ll}
	hsc	&Hamann-Schluter-Chiang\\
	ker	&Kerker\\
	tm2	&Improved Troullier-Martins\\
\end{tabular}

The {\tt ker} and {\tt tm2} schemes can get away with larger $r_c$,
due to their wavefunction matching conditions.

(format 8x, a3, f9.3)

\item The last line (before the blank line) specifies:

\begin{itemize}
\item The values of the $r_c$ in atomic units (bohrs) for the $s$,
$p$, $d$, and $f$ orbitals (it is a good practice to input the valence
orbitals in the order of increasing angular momentum, so that there is
no possible confusion).

(format 4f10.5)  

\item Two extra fields (2f10.5) which are relevant only if non-local
core corrections are used (see Sect~\ref{sec:cc}).
\end{itemize}
\end{itemize}

In the {\tt hsc} example above, only $s$ ,$p$, and $d$ $r_c$'s are
given. Here is an example for Silicon in which we are only interested
in the $s$ and $p$ channels for our pseudopotential, and use the Kerker
scheme:

\begin{verbatim}
# 
   pg Si Kerker generation
        ker     2.00
   Si   ca
         0
    3    2
    3    0      2.00
    3    1      2.00
      1.80      1.80      0.00       0.0       0.0       0.0

#23456789012345678901234567890123456789012345678901234567890   Ruler
\end{verbatim}


This completes the discussion of the more common features of the input
file. See the Appendix~\ref{sec:directives} for more advanced options.

\section{APPENDIX: INPUT FILE DIRECTIVES}
\label{sec:directives}

The fixed format can be a source of desperation for the beginner, and
its rigidity means that it is not easy to add new items to the
input. For this purpose, the program takes another route: several
variables can be entered in a specially flexible format by means of
{\sl directives} at the top of the file. For example

\begin{verbatim}
%define NEW_CC
.... rest of the input file
\end{verbatim}

would signal that we want to use a new core-correction scheme.

There are two kinds of directives, with syntax:
\begin{verbatim}
%VARIABLE=value
%define NAME
\end{verbatim}

In the first case we assign the value {\tt value} to the variable {\tt
 VARIABLE}. The program can look at the value via a special
 subroutine call.

The second form is a bit more abstract, but can be understood as
assigning a special ``existence'' value of {\tt 1} to the variable
{\tt NAME}.  Again, the program can check for the existence of the
variable via a special subroutine call.

Currently, the program understands the following {\tt NAME}s:

\begin{itemize}
\item {\tt COMPAT\_UCB:}  Revert to the standard circa 1990 UCB values. Note   
                   that these correspond to the first released version 
                   of Jose Luis Martins code, not to the old Froyen        
                   version.
     (The defaults are: use a denser grid up to larger radii.
              Use a larger value for the pseudopotential
cutoff point.  Use the Soler-Balbas XC package.)  
                                                                          
\item  {\tt NEW\_CC:} New core-correction scheme
\item {\tt OLD\_CC:}  Old core-correction scheme  (see Sect.~\ref{sec:cc})
 
\item {\tt USE\_OLD\_EXCORR:} Use the old exchange-correlation package.

\item {\tt NO\_PS\_CUTOFFS:} Avoid cutting off the tails of the
pseudopotentials. Currently, a simple exponential tapering function is
used, which introduces a discontinuity in the first derivative of the
ionic pseudopotential.

\item {\tt FREE\_FORMAT\_RC\_INPUT:} Use free-format for the input of
the cutoff radii and the specification of the core-correction
parameters. This is useful for externally-driven runs of ATOM. In this
case the user should make sure that all six values (four rc's plus
the two cc parameters) are present in the input line.

\end{itemize}


%\addcontentsline{toc}{section}{Index}
%\printindex

\end{document}






