\documentclass{article}

\usepackage[a4paper]{geometry}

\newcommand\siesta{\textsc{SIESTA}}
\newcommand\tsiesta{\textsc{TranSIESTA}}

\newcommand{\fdf}[1]{\textbf{#1}}
\newcommand{\opt}[1]{\textbf{#1}}
\newcommand{\code}[1]{\texttt{#1}}

\title{Compatibility issues across versions}
\author{\siesta}

\begin{document}

\maketitle

\clearpage

This list describes the compatibility issues when using different versions of \siesta.

\begin{description}
  \setlength\itemsep{1pt}
  \setlength\topsep{1pt}

  \item[4.1]%
  \begin{itemize}
    \item %
    \tsiesta\ has been fully redeveloped. The entire algorithm has been revised and
    substantial improvements on performance and numerical precision has been achieved. One
    cannot compare, directly, the results although the transmissions should be (roughly)
    equivalent.

    \item %
    \fdf{DM.MixSCF1} is now defaulted to \fdf{true}.

    \item %
    Nearly all flags regarding the mixing options have changed name. However, the old
    flags are still read (and used). In future versions they may be fully removed.

    \begin{itemize}
      \item All mixing routines are re-written. This means that one cannot compare with
      prior versions in terms of the convergence path. Sometimes they will differ due to numerics.
    \end{itemize}

    \item \fdf{NumberOfSpecies} and \fdf{NumberOfAtoms} are now defaulted to \# of lines
    in their respective blocks. If they are specified, however, they will be honoured.

  \end{itemize}
  

  \item[4.0.2, 4.1] %
  Internal tabulated values. We have increased the tabulated grid quantities related to
  orbital basis functions. This change \emph{increases} precision for Siesta. To compare
  with prior versions of Siesta set \texttt{integer, parameter :: NRTAB = 128} in
  \texttt{Src/new\_matel.f}.

  \item[\emph{any} --- 4.0-b2] The following compatibility issues should be remarked when
  comparing with any later version of \siesta.

  \begin{itemize}
    \item \emph{Electric field and dipole correction for slab calculations}

    Older versions applied an incorrect dipole correction when also using an
    electric field (old behaviour may be recovered by forcing \code{SlabDipoleCorrection .false.}). 

    Older versions over-estimated the energy contribution from the dipole correction by a
    factor of 2 (old behaviour cannot be recovered).

  \item\emph{Changes in the default precision from grid magnitudes}

    The grid functions (charge densities, potentials, etc) were in
    single precision by default in the 3.X versions, but are in double
    precision by default for post-3.X versions. The \code{phi}  array that
    holds the values of the basis orbitals on the real-space grid is kept in
    single precision. Please take this into account if you compare the
    results with those of siesta-3.X runs. See the manual in both
    versions for more information.

  \item\emph{Changes in the geometry used for the analysis of the
    electronic structure}

  The update of the atomic coordinates after computing the forces on
  the atoms no longer affects the analysis of the electronic
  structure at the end of a run. To do the analysis using the actual
  coordinates used in the computation of the electronic structure, the
  coordinates are now reset to those values in \code{siesta\_analysis}.  The old
  behavior can be recovered by using by setting the backward
  compatibility switch \opt{compat-pre-v4-dynamics} to 'true'.

  This change will slightly alter the output of most calculations and the
  detailed results of any post-processing. Keep this in mind if you
  need to maintain coherency within a project.


  \item\emph{Changes in the handling of the density-matrix and
    hamiltonian after convergence of the scf cycle}

    To guarantee that the DM and H used in the electronic-structure
    analysis correspond to the latest electronic structure computed,
    there should be no mixing at the end of the scf loop. This
    behavior, controlled by the value of \opt{scf-mix-after-convergence}, is
    now the default. The use of DM\_out instead of DM\_mixed provides
    also a truly variational total-energy and more consistent
    forces. If the scf cycle is carried out mixing H, H(DM\_in) is
    recovered by re-setting H to Hold in \code{siesta\_forces} at the end of
    the cycle.

    Also, to preserve the original H for analysis it is necessary to
    avoid overwriting it in \code{post\_scf\_work}. This is now done by
    default, except if the option \opt{scf-recompute-h-after-scf} is
    'true'.

    There is another switch for pre-v4.0 compatibility in this
    context: If \opt{compat-pre-v4-dm-h} is 'true', the
    \opt{scf-mix-after-convergence} and \opt{scf-recompute-h-after-scf}
    options are activated too.

    This change will slightly alter the output of most calculations
    and the detailed results of any post-processing. Keep this in mind
    if you need to maintain coherency within a project.
    
  \end{itemize}

\end{description}


\end{document}
