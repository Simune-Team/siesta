% Manual for the TBtrans program
%
% To generate the printed version:
%
% pdflatex tbtrans
% pdflatex tbtrans
% splitidx tbtrans.idx (optional if you want a split index)
% makeindex tbtrans    (optional if you have a current siesta.ind)
% pdflatex tbtrans
%
%

% Include settings appropriate for tbtrans
%
% This TeX file includes the default environments for creating fdf blocks
% in the documentation of SIESTA.
%
% These utility functions have been implemented by:
%  Nick R. Papior, 2016 (nickpapior <at> gmail.com)
%
\makeatletter

% Here we define some of the commonly used variables across the document.
\RequirePackage{expl3}
\RequirePackage{xparse}
% One requires the ec font installation
\RequirePackage{textcomp}
\RequirePackage{xcolor}
\RequirePackage{fancyvrb}

% Include standard packages
\input tex/packages.tex

% Include default commands/macros for the siesta manual
\input tex/variables.tex

% This includes how the indexes of the manual should be handled
\input tex/index.tex

% Include default commands for creating output
\input tex/output.tex

% Enable the specific environments for fdf-keys used throughout the documentation
\input tex/fdf.tex

\makeatother


%%% Local Variables:
%%% mode: latex
%%% TeX-master: "../siesta"
%%% End:


% Local command for the software version for printing
% \unskip helps when space/newline chars are present in the
% version.info file.
\providecommand\softwareversion{\input{../version.info}\unskip}

% Title (note that \input above will fail in the title, we however
% don't care since it is only meaningful for final publications where
% direct text is used.)
\title{TBtrans manual \softwareversion}

% Set the date
\date{\today}

% Specify Authors
\author{%
    Nick R. Papior%
}

% Ensure the information is written in the PDF (see tex/setup.tex)
\setpdfmetadata

\begin{document}

% TITLE PAGE --------------------------------------------------------------

\begin{titlepage}

\begin{center}

\vspace{1cm}
\ifdeveloper
 {\Huge \textsc{D e v e l o p e r' s \, \, G u i d e}}
\else
 {\Huge \textsc{U s e r' s \, \, G u i d e}}
\fi

\vspace{1cm}
\hrulefill
\vspace{1cm}

{\Huge \textbf{T B t r a n s \, \, \softwareversion}}

\vspace{1cm}
\hrulefill
\vspace{0.5cm}

{\Large \printdate}

\vspace{1.5cm}
{\Large \url{https://launchpad.net/siesta}}

\end{center}

\end{titlepage}

% END TITLE PAGE --------------------------------------------------------------

\newpage

\section*{Contributors to \tbtrans}
\addcontentsline{toc}{section}{Contributors to \tbtrans}

\tbtrans\ is Copyright \copyright\ 2016-2018 by Nick R. Papior. The
original \tbtrans\ code was implemented by Mads Brandbyge, Jose
L. Mozos, Jeremy Taylor, Pablo Ordejon and Kurt Stokbro. The current
\tbtrans\ is implemented by the following contributors:

\vspace{0.5cm}

\hbox{ \hskip 0.1cm
\begin{tabular}{ll}

  Nick R\"ubner Papior &
  \textit{Technical University of Denmark} \\ \\

\end{tabular}
}


\tableofcontents

\newpage

\section{Introduction}

\textit{This Reference Manual contains descriptions of all the input,
    output and execution features of \tbtrans, but is not a tutorial
    introduction to the program.}

\tbtrans\ (Tight-Binding transport) is a generic computer program
which calculates transport and other physical quantities using the
Green function formalism. 
%
It is a stand-alone program which allows \emph{extreme} scale
tight-binding calculations. 
\begin{itemize}
  
  \item%
  It uses the basic non-equilibrium Green function formalism and
  allows extensive customizability and analysis forms.

  \item%
  \tbtrans\ may be given any type of local-orbital Hamiltonian and
  calculate transport properties of arbitrary geometries and/or number
  of electrodes. 

  \item%
  The \phtrans\ variant may be compiled to obtain thermal (phonon)
  transport using the same Green function formalism and \emph{all} the
  same functionalities as those presented in this manual.

\end{itemize}
As \tbtrans\ output has changed to the flexible NetCDF-4 format you
are highly encouraged to use the \sisl\cite{sisl} toolsuite which has
nearly all the necessary tools available to perform advanced
analysis. If used, please cite \sisl\ appropriately.

A list of the currently implemented features are:
\begin{itemize}

  \item Density of states (orbital resolved)
  \begin{itemize}
    \item Green function DOS
    \item Scattering DOS
  \end{itemize}

  \item Hamiltonian interpolation at different voltages

  \item Selective wide-band limit of the electrode(s)

  \item Transmission eigenvalues

  \item Bulk electrode density of state and transmission (directly
  from the electrode Hamiltonian)

  \item Projected transmission of eigenstates

  \item Orbital resolved ``bond-currents'' which may subsequently be
  analyzed to yield actual bond-currents

  \item Density matrices using the Green function and/or the
  spectral density
  
  \item COOP and COHP curves using the Green function and/or the
  spectral density.
  
\end{itemize}


\vspace{0.5cm}
{\large \textbf{References:} }

\begin{itemize}

  \item%
  Description of the \tbtrans\ and \tsiesta\ code in the $N$ terminal
  generic implementation \cite{Papior2017}.

  \item%
  \sisl\ is a data analysis/extraction utility for \tbtrans\ which
  enables easy access to the data stored in the output NetCDF-4 file
  \cite{sisl}.

\end{itemize}



\section{Compilation}

\tbtrans\ may be compiled in the \shell{Util/TS/TBtrans} directory.

To compile \tbtrans\ simply go to the directory and type:
\begin{shellexample}
  $ make
\end{shellexample}
%$
This will default to use the \file{arch.make} file in the \shell{Obj}
directory. To use a different directory you may do:
\begin{shellexample}
  $ make OBJDIR=AnotherObjDir
\end{shellexample}
%$

\tbtrans\ is tightly intertwined with the \siesta\ source to reduce
code duplication.

Please see the \siesta\ manual for installing NetCDF easily.

\subsection{The \texttt{arch.make}\ file}
\label{sec:arch-make}

The compilation is done using a \shell{Makefile} that is provided with
the code\index{Makefile}. This \shell{Makefile} will generate the
executable for any of several architectures, with a minimum of tuning
required from the user and encapsulated in a separate file called
\file{arch.make}.

\tbtrans\ relies on the following libraries
\begin{description}
  \item[BLAS] %
  it is recommended to use a high-performance library
  (\href{https://github.com/xianyi/OpenBLAS}{OpenBLAS} or the MKL
  library from Intel)
  
  \begin{itemize}
    \item If you use your *nix distribution package manager to install
    BLAS you are bound to have a poor performance. Please try and use
    performance libraries.
  \end{itemize}

  To add BLAS to the \file{arch.make} file you need to add the
  required linker flags to the \shell{LIBS} variable in the
  \file{arch.make} file.

  Example variables
\begin{shellexample}
  # OpenBLAS:
  LIBS += -L/opt/openblas/lib -lopenblas
  # or for MKL
  LIBS += -L/opt/intel/.../mkl/lib/intel64 -lmkl_blas95_lp64 ...
\end{shellexample}

  \item[LAPACK]%
  it is recommended to use a high-performance library
  (\href{https://github.com/xianyi/OpenBLAS}{OpenBLAS}\footnote{OpenBLAS
      enables the inclusion of the LAPACK routines. This is advised.}
  or the MKL library from Intel)

  Example variables
\begin{shellexample}
  # OpenBLAS (OpenBLAS will default to build in LAPACK 3.6)
  LIBS += -L/opt/openblas/lib -lopenblas
  # or for MKL
  LIBS += -L/opt/intel/.../mkl/lib/intel64 -lmkl_lapack95_lp64 ...
\end{shellexample}

\end{description}

The above are the minimally required libraries. 

Highly encouraged libraries
\begin{description}
  \item[\href{https://www.unidata.ucar.edu/software/netcdf}{NetCDF}] %
  Note that it should a NetCDF4 compliant compiled
  library\footnote{Remark that a NetCDF-3 compliant library is not
      sufficient for \tbtrans.}. This library is required for a
  multitude of advanced analysis methods such as orbital resolved DOS,
  bond-currents, $\delta \mathbf H$, eigenstate projections, etc.

  To use this library add these variables to your \file{arch.make}
  file
\begin{shellexample}
  COMP_LIBS += libncdf.a
  FPPFLAGS += -DCDF -DNCDF -DNCDF_4
  LIBS += -lnetcdff -lnetcdf -lhdf5_fortran -lhdf5 -lz
\end{shellexample}

  If you have compiled NetCDF4 with parallel IO you may benefit from
  parallel IO by adding this compilation flag:
\begin{shellexample}
  FPPFLAGS += -DNCDF_PARALLEL
\end{shellexample}

  To easily install \program{NetCDF} please see the installation file:
  \shell{Docs/install\_netcdf4.bash}.
  
\end{description}

Importantly, \tbtrans\ is compatible with hybrid parallelism using MPI
and OpenMP or either of them alone. 

\paragraph{MPI}
To compile using MPI add this to your \file{arch.make} file
\begin{shellexample}
 FPPFLAGS += -DMPI
\end{shellexample}

\paragraph{OpenMP}
To compile using OpenMP add this to your \file{arch.make} file
\begin{shellexample}
 FFLAGS += -fopenmp
 LIBS += -fopenmp
\end{shellexample}
change the corresponding flag according to your compiler.

\paragraph{Running Hybrid parallel \tbtrans}
%
Running \tbtrans\ using hybrid parallelism is difficult due to the
complexity of controlling the threads and processors.

To achieve good performance one \emph{must} ensure that the threads
and processors are not oversubscribe and are not overlapping.
For instance if using OpenMPI $\ge1.8.4$ one may run \tbtrans\
using this command:
\begin{shellexample}
 mpirun -np $((PBS_NP/OMP_NUM_THREADS)) \
   -x OMP_NUM_THREADS \
   -x OMP_PROC_BIND=true \
   --map-by ppr:1:socket:pe=$OMP_NUM_THREADS
\end{shellexample}
where \shell{PBS\_NP} is the total number of processors,
\shell{OMP\_NUM\_THREADS} is the number of threads per processor. The
above command assumes using 1 MPI processor per socket with each
socket having \shell{OMP\_NUM\_THREADS} cores.

\subsubsection{BLAS GEMM3M kernel}

Several modern BLAS implementations allow the use of GEMM3M kernels
for complex linear algebra. They should provide a performance
enhancement for large matrices. To use these routines add this to your
\file{arch.make} file:
\begin{shellexample}
  FPPFLAGS += -DUSE_GEMM3M
\end{shellexample}
Note that OpenMP threaded BLAS libraries are known to fail with the
GEMM3M kernels. 


\subsubsection{Intel MKL libraries}

The MKL libraries are very efficient, but may be difficult to obtain a
correct linking. Here is a short tutorial for linking the MKL BLAS and
LAPACK libraries correctly.

In the following assume that \shell{MKL\_ROOT} points to the root of
the MKL installation directory, for instance one may install MKL into
\shell{/opt/intel/mkl}:
\begin{shellexample}
  MKL_ROOT = /opt/intel/mkl
\end{shellexample}
where \shell{MKL\_ROOT/lib/intel64} contains the libraries.

The linking depends on the used compiler:
\begin{description}

  \item[Intel compiler] %
  \index{compile!Intel}

  The MKL libraries are parallelized using threads and you may also
  enable threads in \tbtrans:
  \begin{description}
    \item[No threading] \mbox{}%

\begin{shellexample}
  LIBS += -L$(MKL_ROOT)/lib/intel64 -lmkl_lapack95_lp64
  -lmkl_blas95_lp64 -lmkl_intel_lp64 -lmkl_core -lmkl_sequential
\end{shellexample}
%$
    \item[OpenMP threading] \mbox{}%
    \index{OpenMP!Intel}

\begin{shellexample}
  LIBS += -openmp -L$(MKL_ROOT)/lib/intel64 -lmkl_lapack95_lp64
  -lmkl_blas95_lp64 -lmkl_intel_lp64 -lmkl_core -lmkl_intel_thread
\end{shellexample}
  %$  
  \end{description}

  \item[GNU compiler] %
  \index{compile!GNU}
  The MKL libraries are parallelized using threads and you may also
  enable threads in \tbtrans:
  \begin{description}
    \item[No threading] \mbox{}%

\begin{shellexample}
  LIBS += -L$(MKL_ROOT)/lib/intel64 -lmkl_lapack95_lp64
  -lmkl_blas95_lp64 -lmkl_gf_lp64 -lmkl_core -lmkl_sequential
\end{shellexample}
%$
    \item[OpenMP threading] \mbox{}%
    \index{OpenMP!GNU}

\begin{shellexample}
  LIBS += -fopenmp -L$(MKL_ROOT)/lib/intel64 -lmkl_lapack95_lp64
  -lmkl_blas95_lp64 -lmkl_gf_lp64 -lmkl_core -lmkl_gnu_thread
\end{shellexample}
  %$  
  \end{description}
  
\end{description}

\section{Execution of the Program}

\tbtrans\ should be called with an input file which defines what
\emph{it should do}. This may either be piped or simply added on the
input line. The latter method is preferred as one may use flags for
the executable.
\begin{shellexample}
  $ tbtrans < RUN.fdf
  $ tbtrans RUN.fdf
\end{shellexample}
Note that if \tbtrans\ is compiled with MPI support one may call it
like 
\begin{shellexample}
  $ mpirun -np 4 tbtrans RUN.fdf
\end{shellexample}
%$
for $4$ MPI-processors.

\tbtrans\ has these optional flags:
\begin{description}
  \item[\fdf*{-h}] print a help instruction

  \item[\fdf*{-out}] specify where all output should be written to
  (instead of STDOUT)
  
  \item[\fdf*{-L}] override \fdf{SystemLabel} flag

  \item[\fdf*{-V}] override \fdf{TBT.Voltage} flag. To denote the
  unit do as this example: \fdf*{-V 0.2:eV} which sets the voltage to
  $0.2\,\mathrm{eV}$.

  \item[\fdf*{-D}] override \fdf{TBT.Directory} flag, all output
  of \tbtrans\ will be put in the corresponding folder (it will be
  created if non-existing)

  \item[\fdf*{-HS}] specify the \fdf{TBT.HS} variable, quickly
  override the used Hamiltonian

  \item[\fdf*{-fdf}] specify any given fdf flag on the command line,
  example \fdf*{-fdf TBT.Voltage:0.2:eV}

\end{description}
Note that for all flags one may use ``:'' as a replacement for `` '',
although one may use quotation marks when having a space in the argument.



\section{fdf-flags}


Although \tbtrans\ is a fully independent Green function transport
code, it is hard-wired with the \tsiesta\ \fdflib\ flags and
options. If you are familiar with \tsiesta\ and its input flags, then
the use of \tbtrans\ should be easy.

All fdf-flags for \tbtrans\ are defaulted to their equivalent
\tsiesta\ flag. Thus if you are using \tsiesta\ as a back-end you
should generally not change any flags. For instance \fdf{TBT.Voltage}
defaults to \fdf*{TS.Voltage} if not supplied.

\begin{fdfentry}{SystemLabel}[string]<siesta>

  The label defining this calculation. All relevant output will be
  prefixed with the \fdf*{SystemLabel}.

  One may start several \tbtrans\ calculations in the same directory
  if they have different labels.

\end{fdfentry}

\begin{fdfentry}{TBT.Voltage}[energy]<$0\,\mathrm{eV}$>
  
  Define the applied bias in the scattering region. 

\end{fdfentry}

\begin{fdfentry}{TBT.Directory}[directory]<./>

  Define the output directory of files from \tbtrans. This allow
  execution of several \tbtrans\ instances in the same folder and
  writing their result to different, say, sub-folders. It is
  particularly useful for interpolation of Hamiltonian's and for
  testing purposes.
  
\end{fdfentry}

\begin{fdfentry}{TBT.Verbosity}[integer]<5>

  Specify how much information \tbtrans\ will print-out (range 0-10).

  For smaller numbers, less information will be printed, and for
  larger values, more information is printed.
  
\end{fdfentry}

\begin{fdfentry}{TBT.Progress}[real]<5.>

  \tbtrans\ prints out an estimated time of completion (ETA) for the
  calculation. By default this is printed out every 5\% of the
  total loops ($k$-point $\times$ energy loops). Setting this to 0
  will print out after every energy loop.
  
\end{fdfentry}



\subsection{Define electronic structure}


\begin{fdfentry}{TBT.HS}[file]<\fdf*{<SystemLabel>}.TSHS>

  Define the Hamiltonian file which contains information regarding the
  Hamiltonian and geometry.

\end{fdfentry}

\begin{fdfentry}{TBT.HS.Files}[block]
  
  A list of files which each contain the Hamiltonian for the same
  geometry at different bias'. 
  % 
  Each line has three entries, 1) the \fdf{TBT.HS} file, 2) the value
  of the bias applied, 3) the unit of the bias.

  \note if this is existing it will assume that you will perform an
  interpolation of the Hamiltonians to the corresponding bias
  (\fdf{TBT.Voltage}). 

\end{fdfentry}

\begin{fdfentry}{TBT.HS.Interp}[string]<spline|linear>
  \fdfdepend{TBT.HS.Files}

  Interpolate all files defined in \fdf{TBT.HS.Files} to the
  corresponding applied bias.

  Generally \fdf*{spline} produces the best interpolated values and
  its use is encouraged. The linear interpolation scheme is mainly
  used for comparison to the \fdf*{spline}. If they are very different
  from each other then one may be required to perform additional
  self-consistent calculations at the specific bias due to large
  changes in the electronic structure.

\end{fdfentry}

Say you have calculated the SCF solution of a certain system at 5
different applied bias':
\begin{fdfexample}
  %block TBT.HS.Files
    ../V0/siesta.TSHS     0.  eV
    ../V-0.5/siesta.TSHS -0.5 eV
    ../V0.5/siesta.TSHS   0.5 eV
    ../V-1.0/siesta.TSHS -1.0 eV
    ../V1.0/siesta.TSHS   1.0 eV
  %endblock
\end{fdfexample}
and you wish to calculate the interpolated transmissions and currents
at steps of $0.1\,\mathrm{eV}$, then you may use this simple loop
\begin{shellexample}
  for V in `seq -1.5 0.1 1.5` ; do
     tbtrans -V $V:eV -D V$V RUN.fdf
  done
\end{shellexample}
which at each execution of \tbtrans\ interpolates the Hamiltonian to
the corresponding applied bias and store all output files in the
\shell{V\$V} folder.

\subsubsection{Changing the electronic structure via
    \texorpdfstring{$\delta$}{d} elements}

\newcommand\dH{\delta\mathbf H}
\newcommand\dSE{\delta\mathbf\Sigma}

The electronic structure may be altered by changing the Hamiltonian
elements via a simple additive term
\begin{equation}
  \mathbf H \leftarrow \mathbf H + \dH + \dSE,
\end{equation}
which allows easy changes to the electronic structure or adding
additional terms such as imaginary self-energies. One may also use it
to add magnetic fields etc.

\tbtrans\ uses a distinction between $\dH$ and $\dSE$ only via the
orbital current calculation. I.e. $\dH$ enters the equations for
calculating the orbital current, whereas $\dSE$ does not. Otherwise
the two $\delta$-terms are \emph{completely} identical. In the
following discussion we will use the term $\delta$ to be either $\dH$
or $\dSE$.

To use this feature at $k$ points it is important to know that phases
in \tbtrans\ are defined using the lattice vectors (and \emph{not}
inter-atomic distances)
\begin{equation}
  \mathbf H_k = \mathbf H \cdot e^{i k \cdot \mathbf R}.
  \label{eq:hk-phase}
\end{equation}
\tbtrans\ will add the phases on \emph{all} elements of $\delta$ via
Eq.~\eqref{eq:hk-phase}. To counter these phases one may simply
multiply $\delta$ with the negative phase ($-i$). Note that phases are
only added on super cell elements, \emph{not} unit cell elements.


\begin{fdfentry}{TBT.dH}[file]

  Denote a file which contains the $\dH$ information.

  \note that the terms defined in this file are added to the
  Hamiltonian when calculating the orbital currents, if your terms are
  not a Hamiltonian change, then consider using \fdf{TBT.dSE} instead.

  This file
  \emph{must} adhere to these file format notations and is required to
  be supplied in a NetCDF4 format

  \begin{shellexample}
netcdf file.dH {
dimensions:
	one = 1 ;
	n_s = 9 ;
	xyz = 3 ;
	no_u = 900 ;
	spin = 1 ;
variables:
	int nsc(xyz) ;
		nsc:info = "Number of supercells in each unit-cell direction" ;

group: LEVEL-1 {
  dimensions:
  	nnzs = 2670 ;
  variables:
  	int n_col(no_u) ;
  		n_col:info = "Number of non-zero elements per row" ;
  	int list_col(nnzs) ;
  		list_col:info = "Supercell column indices in the sparse format" ;
  	int isc_off(n_s, xyz) ;
  		isc_off:info = "Index of supercell coordinates" ;
  	double Redelta(spin, nnzs) ;
  		Redelta:info = "Real part of delta" ;
  		Redelta:unit = "Ry" ;
  	double Imdelta(spin, nnzs) ;
  		Imdelta:info = "Imaginary part of delta" ;
  		Imdelta:unit = "Ry" ;
  } // group LEVEL-1

group: LEVEL-2 {
  dimensions:
  	nkpt = UNLIMITED ;
  	nnzs = 2670 ;
  variables:
  	double kpt(nkpt, xyz) ;
  		kpt:info = "k-points for delta values" ;
  		kpt:unit = "b**-1" ;
  	... n_col list_col isc_off ...
  	double delta(nkpt, spin, nnzs) ;
  		delta:info = "delta" ;
  		delta:unit = "Ry" ;
  } // group LEVEL-2

group: LEVEL-3 {
  dimensions:
  	ne = UNLIMITED ;
  	nnzs = 2670 ;
  variables:
  	double E(ne) ;
  		E:info = "Energy points for delta values" ;
  		E:unit = "Ry" ;
  	... n_col list_col isc_off ...
  	double delta(ne, spin, nnzs) ;
  		delta:info = "delta" ;
  		delta:unit = "Ry" ;
  } // group LEVEL-3

group: LEVEL-4 {
  dimensions:
  	nkpt = UNLIMITED ;
  	ne = UNLIMITED ;
  	nnzs = 2670 ;
  variables:
  	double kpt(nkpt, xyz) ;
  		kpt:info = "k-points for delta values" ;
  		kpt:unit = "b**-1" ;
  	double E(ne) ;
  		E:info = "Energy points for delta values" ;
  		E:unit = "Ry" ;
  	... n_col list_col isc_off ...
  	double delta(nkpt, ne, spin, nnzs) ;
  		delta:info = "delta" ;
  		delta:unit = "Ry" ;
  } // group LEVEL-4
}
  \end{shellexample}

  This example file shows how the file should be formatted. Note that
  one may either define the Hamiltonian as \shell{delta} or as
  \shell{Redelta} and \shell{Imdelta}. The former is defining $\delta$
  as a real quantity while the latter makes it an imaginary $\delta$.

  The levels are defined because they have precedence from each other,
  if the energy point and $k$ point is found in LEVEL-4 it will use
  this, if not, it will check for the energy point in LEVEL-3, and so
  on. 

\end{fdfentry}

The remaining options are only applicable if \fdf{TBT.dH} has been
set. 

\begin{fdflogicalT}{TBT.dH!Parallel}

  Whether the $\delta\mathbf H$ file should be read in parallel. If
  your architecture supports parallel IO it is beneficial to do so. 
  %
  \tbtrans\ performs a basic check whether parallel IO may be
  possible, if it cannot assert this it will be turned off.
  
\end{fdflogicalT}

\begin{fdfentry}{TBT.dSE}[file]

  File has same format as specified for \fdf{TBT.dH}.

  The only difference between a $\dH$ and $\dSE$ file is that the
  terms in $\dSE$ does \emph{not} enter the calculation of the
  bond-currents, whereas $\dH$ does.

\end{fdfentry}

\subsection{Determine calculated physical quantities}
\label{sec:physical}

\tbtrans\ can calculate a large variety of physical
quantities. By default it will only calculate the transmission between
the electrodes. Calculating as few quantities as possible will
increase throughput, while requesting many quantities will result in
much longer run-times.

You are heavily encouraged to compile \tbtrans\ with NetCDF4 support,
see Sec.~\ref{sec:arch-make}, as quantities will be orbital resolved. 

If \tbtrans\ has been compiled with NetCDF4 support, one may extract
the projected DOS from the \sysfile{TBT.nc} using \sisl\ (or manual
scripting). The calculated DOS can \emph{only} be extracted from the
atoms in the device region (atoms in block
\fdf{TBT.Atoms!Device}). Hence the \fdf{TBT.Atoms!Device} block is
\emph{extremely} important when conducting detailed DOS analysis. 
For instance if the input file has this:
\begin{fdfexample}
  %block TBT.Atoms.Device 
    atom [20 -- 40]
  %endblock
\end{fdfexample}
one may extract the PDOS on a subset of atoms using this \sisl\
command
\begin{shellexample}
  sdata siesta.TBT.nc --atom 20-30 --dos --ados Left --out dos_20-30.dat
  sdata siesta.TBT.nc --atom 20-30[1-3] --dos --ados Left --out dos_20-30_1-3.dat
\end{shellexample}
where the former is the total PDOS on atoms $20$ through $30$, and the
latter is the PDOS on orbitals 1, 2 and 3 on atoms $20$ through
$30$. It thus is extremely easy to extract different PDOS once the
calculation has completed. 

\begin{fdflogicalF}{TBT.T!Bulk}

  Calculate the bulk (pristine) electrode transmission if
  \fdftrue.

  This generates \sysfile{BTRANS\_<>} and \sysfile{AVBTRANS\_<>}.
  
  \note implicitly enables \fdf{TBT.DOS!Elecs} if \fdftrue.

\end{fdflogicalF}

\begin{fdflogicalF}{TBT.DOS!Elecs}

  Calculate the bulk (pristine) electrode DOS if
  \fdftrue. 

  This generates \sysfile{BDOS\_<>} and \sysfile{AVBDOS\_<>}.

  \note implicitly enables \fdf{TBT.T!Bulk} if \fdftrue.

\end{fdflogicalF}


\begin{fdflogicalF}{TBT.DOS!Gf}
  \fdfdepend{TBT.Atoms!Device}

  Calculate the DOS from the Green function on the atoms in the device
  region. 

  \note this flag should only be used if there are bound states in the
  scattering region (or if one wish to uncover whether there are bound
  states). Due to internal algorithms the DOS from the Green function
  is computationally more demanding than using \fdf{TBT.DOS!A} and
  \fdf{TBT.DOS!A.All}.

  This generates \sysfile{DOS} and \sysfile{AVDOS}.

  See \fdf{TBT.Atoms!Device.Connect}.

  In case any of \fdf{TBT.DM!Gf}, \fdf{TBT.COOP!Gf} or
  \fdf{TBT.COHP!Gf} is \fdftrue\ this flag will be set to \fdftrue\ as
  well.

\end{fdflogicalF}

\begin{fdflogicalF}{TBT.DOS!A}
  \fdfdepend{TBT.Atoms!Device}

  Calculate the DOS from the spectral function. This will not
  calculate the DOS from the last electrode (last in the list
  \fdf{TBT.Elecs}), see \fdf{TBT.DOS!A.All}. 

  This generates \sysfile{ADOS\_<>} and \sysfile{AVADOS\_<>}.

  See \fdf{TBT.Atoms!Device.Connect}.

  In case any of \fdf{TBT.Current!Orb}, \fdf{TBT.DM!A}, \fdf{TBT.COOP!A} or
  \fdf{TBT.COHP!A} is \fdftrue\ this flag will be set to \fdftrue\ as
  well.

\end{fdflogicalF}

\begin{fdflogicalF}{TBT.DOS!A.All}
  \fdfdepend{TBT.Atoms!Device}

  Calculate the DOS from the spectral function and do so with
  \emph{all} electrodes.

  This additionally generates \sysfile{ADOS\_<>} and
  \sysfile{AVADOS\_<>} for the last electrode in \fdf{TBT.Elecs}.

  \note if \fdftrue, this implicitly sets \fdf{TBT.DOS!A} to \fdftrue.

\end{fdflogicalF}

Setting the flags \fdf{TBT.DOS!Gf} and \fdf{TBT.DOS!A.All} to
\fdftrue\ enables the estimation of bound states in the scattering
region via this simple expression
\begin{equation}
  \rho_{\mathrm{bound-states}} = \rho_{\mathbf G} - 
  \sum_i \rho_{\mathbf A_i},
\end{equation}
where the sum is over all electrodes, $\mathbf G$ and $\mathbf A_i$ are
the Green and spectral function, respectively. Note that typically
$\rho_{\mathrm{bound-states}}=0$.

The below two options enables the calculation of the energy resolved
density matrices. In effect they may be used to construct
$\mathrm{LDOS}(E)$ profiles using \sisl.

\begin{fdflogicalF}{TBT.DM!Gf}
  \fdfdepend{TBT.Atoms!Device}

  Calculate the energy and $k$-resolved density matrix for the Green
  function. The density matrix may be used to construct 
  real-space LDOS profiles.
  
\end{fdflogicalF}

\begin{fdflogicalF}{TBT.DM!A}
  \fdfdepend{TBT.Atoms!Device}

  Calculate the energy and $k$-resolved density matrix for the
  electrode spectral functions. The density matrix may be used to
  construct real-space LDOS profiles.
  
\end{fdflogicalF}


In addition to the DOS analysis of the Green and spectral functions,
the Crystal Orbital Overlap Population and Crystal Orbital Hamilton
Population may also be calculated. These are only available if
\tbtrans\ is compiled with NetCDF-4 support.

\begin{fdflogicalF}{TBT.COOP!Gf}
  \fdfdepend{TBT.Atoms!Device,TBT.DOS!Gf}

  Calculate COOP from the Green function in the device region.

  The COOP curve is calculated as:
  \begin{equation}
    \label{eq:COOP-Gf}
    \mathrm{COOP}_{\mu\nu} = \frac{-1}\pi\Im[\mathbf G_{\mu\nu} \SO_{\nu\mu}].
  \end{equation}
  The COOP curves are orbital, energy and $k$-resolved and they may
  thus result in very large output files.

  \note Untested!

\end{fdflogicalF}

\begin{fdflogicalF}{TBT.COOP!A}
  \fdfdepend{TBT.Atoms!Device,TBT.DOS!A.All}

  Calculate COOP from the spectral function in the device region.

  The COOP curve is calculated as:
  \begin{equation}
    \label{eq:COOP-A}
    \mathrm{COOP}_{\mu\nu} = \frac{1}{2\pi}\Re[\mathbf A_{\mu\nu} \SO_{\nu\mu}].
  \end{equation}
  The COOP curves are orbital, energy and $k$-resolved and they may
  thus result in very large output files.

  \note Untested!

\end{fdflogicalF}

\begin{fdflogicalF}{TBT.COHP!Gf}
  \fdfdepend{TBT.Atoms!Device}

  Calculate COHP from the Green function in the device region.

  The COHP curve is calculated as:
  \begin{equation}
    \label{eq:COHP-Gf}
    \mathrm{COHP}_{\mu\nu} = \frac{-1}\pi\Im[\mathbf G_{\mu\nu} \Ham_{\nu\mu}].
  \end{equation}
  The COHP curves are orbital, energy and $k$-resolved and they may
  thus result in very large output files.

  \note Untested!

\end{fdflogicalF}

\begin{fdflogicalF}{TBT.COHP!A}
  \fdfdepend{TBT.Atoms!Device,TBT.DOS!A.All}

  Calculate COHP from the spectral function in the device region.

  The COHP curve is calculated as:
  \begin{equation}
    \label{eq:COHP-A}
    \mathrm{COHP}_{\mu\nu} = \frac{1}{2\pi}\Re[\mathbf A_{\mu\nu} \Ham_{\nu\mu}].
  \end{equation}
  The COHP curves are orbital, energy and $k$-resolved and they may
  thus result in very large output files.

  \note Untested!

\end{fdflogicalF}



\begin{fdfentry}{TBT.T!Eig}[integer]<0>

  Specify how many of the transmission eigenvalues will be
  calculated. 

  This generates \sysfile{TEIG\_<1>\_<2>} and
  \sysfile{AVTEIG\_<1>\_<2>}, possibly \sysfile{CEIG\_<1>} and
  \sysfile{AVCEIG\_<1>}. The former is for two different electrodes
  $i\neq j$, while the latter is for electrode $i=j$.
  
  \note if you specify a number of eigenvalues above the available
  number of eigenvalues, \tbtrans\ will automatically truncate it to a
  reasonable number.

  \note The transmission eigenvalues for $N>2$ systems is not fully
  understood and the transmission eigenvalues calculated in \tbtrans\ is
  done by diagonalizing this sub-matrix:
  \begin{equation}
    \mathbf G \boldsymbol \Gamma_i \mathbf G^\dagger \boldsymbol \Gamma_j.
  \end{equation}

\end{fdfentry}

\begin{fdflogicalF}{TBT.T!All}

  By default \tbtrans\ only calculates transmissions in \emph{one}
  direction because time-reversal symmetry makes $T_{ij}=T_{ji}$. If
  one wishes to assert this, or if time-reversal symmetry does not
  apply for your system, one may set this to \fdftrue\ to explicitly
  calculate all transmissions.

  This additionally generates \sysfile{TRANS\_<1>\_<2>} and
  \sysfile{AVTRANS\_<1>\_<2>} for all electrode combinations (and the
  equivalent eigenvalue files if \fdf{TBT.T!Eig} is \fdftrue.

\end{fdflogicalF}

\begin{fdflogicalF}{TBT.T!Out}
  
  The total transmission out of any electrode\footnote{In
      $N>2$-electrode calculations one \emph{cannot} use this quantity
      to calculate the total current out of an electrode.} may easily
  be calculated using only the scattering matrix of the origin
  electrode and the scattering region Green function.
  %
  This enables the calculation of these equations
  \begin{align}
    i\Tr[(\mathbf G - \mathbf G^\dagger)\boldsymbol \Gamma_j],
    \label{eq:G.Gamma}
    \\
    \Tr[\mathbf G \boldsymbol \Gamma_j \mathbf G^\dagger\boldsymbol
    \Gamma_j].
    \label{eq:G.Gamma.G.Gamma}
  \end{align}
  The total transmission out of electrode $j$ may then be calculated
  as
  \begin{equation}
    \label{eq:T-out}
    T_j = i\Tr[(\mathbf G - \mathbf G^\dagger)\boldsymbol \Gamma_j] 
    -
    \Tr[\mathbf G \boldsymbol \Gamma_j \mathbf G^\dagger\boldsymbol \Gamma_j].
  \end{equation}

  This generates two sets of files: \sysfile{CORR\_<>} and
  \sysfile{TRANS\_<1>\_<1>} which corresponds to equations
  Eqs.~\eqref{eq:G.Gamma} and \eqref{eq:G.Gamma.G.Gamma},
  respectively. To calculate $T_j$ subtract the two files
  according to Eq.~\eqref{eq:T-out}.
  
\end{fdflogicalF}

\begin{fdflogicalF}{TBT.Current!Orb}
  \fdfdepend{TBT.Atoms!Device,TBT.DOS!A}

  Whether the orbital currents will be calculated and stored.  These
  will be stored in a sparse matrix format corresponding to the
  \siesta\ sparse format with only the device atoms in the sparse
  pattern.
  
  Orbital currents are implemented as:
  \begin{equation}
    J_{\alpha \beta}(E) = i [
    (\mathbf H_{\beta\alpha} - E\mathbf S_{\beta\alpha}) \mathbf A_{\alpha\beta}(E)
    - 
    (\mathbf H_{\alpha\beta} - E\mathbf S_{\alpha\beta}) \mathbf A_{\beta\alpha}(E)],
  \end{equation}
  where we have left out the pre-factor ($e/\hbar$)
  intentionally. \sisl\ may be used to analyze the orbital currents
  and enables easy transformation of orbital currents to bond currents
  and activity currents\cite{Papior2017}.

  \note this requires \tbtrans\ to be compiled with NetCDF-4 support,
  see Sec.~\ref{sec:arch-make}.
  
\end{fdflogicalF}

\begin{fdfentry}{TBT.Spin}[integer]<\nonvalue{all}>

  If the Hamiltonian is a polarized calculation one my define the
  index of the spin to be calculated.

  This allows one to simultaneously calculate the spin-up and
  spin-down transmissions, for instance
\begin{shellexample}
  $ tbtrans -fdf TBT.Spin:1 -D UP RUN.fdf &
  $ tbtrans -fdf TBT.Spin:2 -D DOWN RUN.fdf &
\end{shellexample}
  which will create two folders \program{UP} and \program{DOWN} and
  output the relevant physical quantities in the respective folders.
  
\end{fdfentry}

\begin{fdflogicalT}{TBT.Symmetry!TimeReversal}

  Whether the Hamiltonian and the calculation should use time-reversal
  symmetry. 
  Currently this only affects $\mathbf k$-point sampling calculations
  by not removing any symmetry $\mathbf k$-points.

  If one has $\mathbf k$-point sampling and wishes to use
  \fdf{TBT.Current!Orb} this should be \fdffalse.
  
\end{fdflogicalT}


\subsubsection{Device region} 

The scattering region (and thus device region) is formally consisting
of all atoms besides the electrodes. However, when calculating the
transmission this choice is very inefficient. Thus to heavily increase
throughput one may define a smaller device region consisting of a
subset of atoms in the scattering region.

The choice of atoms \emph{must} separate each electrode from each
other. \tbtrans\ will stop if this is not enforced.

Remark that the physical quantities such as DOS, spectral DOS, orbital
currents may only be calculated in the selected device region. 

\begin{fdfentry}{TBT.Atoms!Device}[block/list]%
  <\nonvalue{all but electrodes}>

  This flag may either be a block, or a list.

  A block with each line denoting the atoms that consists of the
  device region.
  %
  \begin{fdfexample}
    %block TBT.Atoms.Device
       atom [ 10 -- 20 ]
       atom [ 30 -- 40 ]
       # Atoms removed from the device region
       # Even though they are specified in other
       # lines
       not-atom [ 15, 35]
    % endblock
    # Or equivalently as a list   
    TBT.Atoms.Device [10 -- 14, 16 -- 20, 30 -- 34, 36 -- 40]
  \end{fdfexample}
  will limit the device region to atoms [10--14, 16--20, 30--34, 36--40].
  
\end{fdfentry}

\begin{fdflogicalF}{TBT.Atoms!Device.Connect}

  Setting this to \fdftrue\ will \emph{extend} the device region to
  also include atoms that the input device atoms has matrix
  elements between. This may be important when using non-orthogonal
  basis sets as one can ensure the full overlap matrix on the selected
  device atoms.
  
  \note this parameter should be set to \fdftrue\ in case accurate DOS
  calculations are required on the specified device atoms (if using a
  non-orthogonal basis set).

\end{fdflogicalF}

\begin{fdfentry}{TBT.Atoms!Buffer}[block/list]%

  A block with each line denoting the atoms that are disregarded in
  the Green function calculation.
  % 
  For self-consistent calculations it may be required to introduce
  buffer atoms which are removed from the SCF cycle. In such cases
  these atoms should also be removed from the transport calculation.
  %
  \begin{fdfexample}
    %block TBT.Atoms.Buffer
       atom [ 1 -- 5 ]
    %endblock
    # Or equivalently as a list   
    TBT.Atoms.Buffer [1 -- 5]
  \end{fdfexample}
  will remove atoms [1--5] from the calculation.
  
\end{fdfentry}


\subsubsection{Brillouin zone}

\tbtrans\ allows calculating physical quantities via averaging in the
Brillouin zone. 

\begin{fdfentry}{TBT.k}[list/block]<\nonvalue{kgrid\_Monkhorst\_Pack}>
  
  Specify how to perform Brillouin zone integrations.

  This may be given as a list like this:
  \begin{fdfexample}
    TBT.k [A B C]
  \end{fdfexample}
  where each integer corresponds to the diagonal elements of the
  Monkhorst-Pack grid. I.e. 
  \begin{fdfexample}
    TBT.k [10 10 1]
    %block TBT.k
      10  0 0 0.
       0 10 0 0.
       0  0 1 0.
    %endblock 
  \end{fdfexample}
  are equivalent.


  If you supply this flag as a block the following options are available:
  \begin{fdfoptions}

    \option[path]%
    \fdfindex{TBT.k!path}%
    
    Define a Brillouin zone path\footnote{Much like \fdf*{BandLines}
        in \siesta.} where the $k$-points are equi-spaced.
    %
    It may be best described using this example:
    \begin{fdfexample}
  path 10
    from 0.  0.  0.
    to   0.5 0.5 0.
  path 20
    from 0.25 0.25 0.
    to   0.0  0.5  0.
    \end{fdfexample}

    This will create $k$-points starting from the $\Gamma$-point and
    move to the Brillouin zone boundary at [$1/2$, $1/2$, $0$] with
    spacing to have 10 points.

    There is no requirement that the \fdf*{path}s are
    connected and one may specify as many paths as wanted.

    \begin{fdfoptions}

      \option[even-path]%
      It is generally advised to add this flag in the blog
      (somewhere) if one wants equi-distance $k$-spacings in the
      Brillouin zone. This flag sums up the total number of $k$-points
      on the total path and then calculates the exact number of
      required points required on each path to have the same $\delta
      k$ in each path.

    \end{fdfoptions}

    \note if any one \fdf*{path} is found in the block the options
    (explained below) are ignored.


    \option[diagonal|diag]%
    \fdfindex{TBT.k!diagonal}%

    Specify the number of $k$ points in each unit-cell direction

    \fdf*{diagonal 3 3 1} will use 3 $k$ points along the first and
    second lattice vectors and only one along the third lattice
    vector.


    \option[displacement|displ]%
    \fdfindex{TBT.k!displacement}%

    Specify the displacement of the Brillouin zone $k$ points along
    each lattice vector. This input is similar to \fdf*{diagonal} but
    requires real input.

    \fdf*{displacement 0.5  0.25 0.} will displace the first and
    second $k$ origin to [$1/2$, $1/4$, $0$].


    \option[size]%
    \fdfindex{TBT.k!size}%

    This reduces the sampled Brillouin zone to only the fractional
    size of each lattice vector direction.

    This may be used to only sample $k$-points in a reduced Brillouin
    zone which for instance is useful if one wishes to sample the
    Dirac point in graphene in an energy range of $-0.5\,\mathrm{eV}$
    -- $0.5\,\mathrm{eV}$. 

    \fdf*{size 0.5 1. 1.} will reduce the sampled $k$ points along the
    first reciprocal lattice to be in the range ]$-1/4$, $1/4$], while
    the other directions are still sampled ]$-1/2$, $1/2$].

    \note expert use only.

    \option[list]%
    \fdfindex{TBT.k!list}%

    Explicitly specify the sampled $k$-points and (optionally) the
    associated weights.
    \begin{fdfexample}
  list 2
    0.  0.  0.  0.5
    0.5 0.5 0.
    \end{fdfexample}
    where the integer on the \fdf*{list} line specifies the number of
    lines that contains $k$ points. Each line \emph{must} be created
    with $3$ reals which define the $k$ point in units of the
    reciprocal lattice vectors (]$-1/2$--$1/2$]). 

    An optional 4th value denote the associated weight which is
    defaulted to $1/N$ where $N$ is the total number of $k$ points.

    \note if this is found it will neglect the other input options
    (except \fdf*{path}).

    \option[method]%
    \fdfindex{TBT.k!method}%

    Define how the $k$-points should be created in the Brillouin zone.

    Currently these options are available (\fdf*{Monkhorst-Pack} being
    the default)
    \begin{fdfoptions}

      \option[Monkhorst-Pack|MP]%
      \fdfindex{TBT.k!method!Monkhorst-Pack}%

      Use the regular Monkhorst-Pack sampling (equi-spaced) with
      simple linear weights.


      \option[Gauss-Legendre]%
      \fdfindex{TBT.k!method!Gauss-Legendre}%

      Use the Gauss-Legendre quadrature and weights for constructing
      the $k$ points and weights. These $k$ points are not
      equi-spaced and puts more weight to the $\Gamma$ point.


      \option[Simpson-mix]%
      \fdfindex{TBT.k!method!Simpson-mix}%
      
      Use the Newton-Cotes method (Simpson, degree 3) which uses equi-spaced
      points but non-uniform weights.


      \option[Boole-mix]%
      \fdfindex{TBT.k!method!Boole-mix}%

      Use the Newton-Cotes method (Boole, degree 5) which uses equi-spaced
      points but non-uniform weights.

    \end{fdfoptions}

    \option[\fdfvalue*{siesta-method}]%

    One may also use the typical \fdf*{kgrid\_Monkhorst\_Pack} method
    of input as done in \siesta. This is a $3\times3$ block such as:
    \begin{fdfexample}
  10  0  0 0.
   0 15  0 0.
   0  0  1 0.
    \end{fdfexample}
    which uses $10$, $15$ and $1$ $k$-points along the 1st, 2nd and
    3rd reciprocal lattice vectors. And with $0$ displacement.

    \note it is recommended to use the \fdf*{diagonal} option unless
    off-diagonal $k$ points are needed.

  \end{fdfoptions}
  
\end{fdfentry}


\subsubsection{Energy grid}

\tbtrans\ uses a default energy reference as the Fermi level in the
corresponding \tsiesta\ calculation. I.e. the equilibrium Fermi
level. Thus one should be aware when using a shifted bias window that
the calculated properties shifts according to the applied bias. For
example; if one performs two equivalent 2-terminal calculations A)
with $\mu_L=V$, $\mu_R=0$ and the other B) with $\mu_L=V/2$,
$\mu_R=-V/2$ then the calculated properties are equivalent if one
shifts the energy spectrum of A) by $E \to E - V/2$. Any 2-terminal
calculation is recommended to be setup with $\mu_L=V/2$ and
$\mu_R=-V/2$ due to the fixed energy reference, $E_R = 0$.

The Green function is calculated at explicit energies and does not
rely on diagonalization routines to retrieve the eigenspectrum. This
is due to the smearing of states from the coupling with the
semi-infinite electrodes.

It is thus important to define an energy grid for analysis of the DOS
and transmission.

\begin{fdfentry}{TBT.Contours!Eta}[energy]<$0\,\mathrm{eV}$>
  
  The imaginary ($\eta$) part of the Green function in the device
  region. Note that the electrodes imaginary part may be controlled
  via \fdf{TBT.Elecs!Eta}.

  This value controls the smearing of the DOS on the energy
  axis. Generally will the electrode $\eta$ value contribute to a
  smearing in the device region, while in certain situations an
  imaginary $\eta$ is required in the device region.
  
\end{fdfentry}

\begin{fdfentry}{TBT.Contours}[block]<\emph{see note further down}>

  Each line in this block corresponds to a specific contour.
  Enabling several lines of input allows to create regions of the
  energy grid which has a high density and ranges of energies with
  lower density. Also it allows to bypass energy ranges where the DOS
  is zero in for instance a semi-conductor.  

  See \fdf{TBT.Contour!<>} for details on specifying the energy contour.
  
\end{fdfentry}

\begin{fdfentry}{TBT.Contour!<>}[block]

  Specify a contour named \fdf*{<>} with options within the block.

  The names \fdf*{<>} are taken from the \fdf{TBT.Contours} block.

  The format of this block is made up of at least $3$ lines, in the
  following order of appearance.

  \begin{fdfoptions}

    \option[from \emph{a} to \emph{b}]%
    \fdfindex{TBT.Contour!<>!from}%

    Define the integration range on the energy axis.
    Thus \emph{a} and \emph{b} are energies.


    \option[points|delta|file]%
    \fdfindex{TBT.Contour!<>!points}%
    \fdfindex{TBT.Contour!<>!delta}%
    \fdfindex{TBT.Contour!<>!file}%

    Define the number of integration points/energy separation.
    If specifying the number of points an integer should be supplied.

    If specifying the separation between consecutive points an energy
    should be supplied (e.g. \fdf*{0.01 eV}).

    Optionally one may specify a file which contains the energy points
    and their weights.

    This file has the same formatting as the \sysfile{TBT.CC} output
    with some optional inputs. Below is an example input file.
    \begin{shellexample}
# There are 2 different input options:
# 1. Re[E] Im[E] W  (optional unit)
# 2. Re[E] W        (optional unit) (imaginary part will be device Eta)
# If the unit is specified on any line, all subsequent lines will use
# the specified unit. Default unit is eV!
# Empty lines and lines starting with # will be ignored.
-0.5  0.1    # E = -0.5 eV, weight (for integrating current) of 0.1 eV
-0.01 0.1 Ry # E = -0.01 Ry and weight 0.1 Ry
-0.02 0.1    # E = -0.02 Ry (above unit continue) and weight 0.1 Ry
-0.2  0.1 eV # E = -0.2 eV and weight 0.1 eV
-0.2 1. 0.1  # E = -0.2 eV and 1. eV eta and weight 0.1 eV
\end{shellexample}

    If the file specified is \sysfile{TBT.CC} the same energy points
    will be used. Note that the resulting \sysfile{TBT.nc} file does
    not store the energies as complex numbers, thus one cannot
    subsequently extract the $\eta$ value used for the individual
    energy points.

    \note for \phtrans\ the energies will be squared internally to be
    in correct units, hence the units should still be $\mathrm{eV}$.

    \option[method]%
    \fdfindex{TBT.Contour!<>!method}%

    Specify the numerical method used to conduct the integration. Here
    a number of different numerical integration schemes are accessible

    \begin{fdfoptions}
      \option[mid|mid-rule]%
      Use the mid-rule for integration.

      \option[simpson|simpson-mix]%
      Use the composite Simpson $3/8$ rule (three point Newton-Cotes).

      \option[boole|boole-mix]%
      Use the composite Booles rule (five point Newton-Cotes).
 
      \option[G-legendre]%
      Gauss-Legendre quadrature.

      \option[tanh-sinh]%
      Tanh-Sinh quadrature.

      \note has \fdf*{opt precision <>}.

      \option[user]%
      User defined input via a file.

    \end{fdfoptions}

    \option[opt]%
    \fdfindex{TBT.Contour!<>!opt}%

    Specify additional options for the \fdf*{method}. Only a selected
    subset of the methods have additional options.

  \end{fdfoptions}

\end{fdfentry}

By default the \tbtrans\ energy grid is defined as
\begin{fdfexample}
  TBT.Contours.Eta 0. eV
  %block TBT.Contours
    line
  %endblock
  %block TBT.Contour.line
     from -2. eV to 2. eV
      delta 0.01 eV
        method mid-rule
  %endblock
\end{fdfexample}

An example of input using a file (note that regular contour setups may
be used together with file-inputs)
\begin{fdfexample}
  TBT.Contours.Eta 0. eV
  %block TBT.Contours
    file
  %endblock
  %block TBT.Contour.file
     from 2. eV to 2.5 eV
      file my_energies
  %endblock
\end{fdfexample}
Note that the energy specifications are necessary (due to internal
bookkeeping).

\subsection{Chemical potentials}


For $N$ electrodes there will also be $N_\mu$ chemical
potentials. They are defined via blocks similar to \fdf{TBT.Elecs}.
If no bias is applied \tbtrans\ will default to a single chemical
potential with the chemical potential in equilibrium. In this case you
need not specify any chemical potentials.

By default \tbtrans\ creates a single chemical potential with the
chemical potential equal to the device Fermi-level. Hence, performing
non-bias calculations does not require one to specify these blocks.

\begin{fdfentry}{TBT.ChemPots}[block]
  
  Each line denotes a new chemical potential which may is further
  defined in the \fdf{TBT.ChemPot!<>} block.
  
\end{fdfentry}

\begin{fdfentry}{TBT.ChemPot!<>}[block]

  Each line defines a setting for the chemical potential named
  \fdf*{<>}.

  \begin{fdfoptions}
    
    \option[chemical-shift|mu]%
    \fdfindex{TBT.ChemPot!<>!chemical-shift}%
    \fdfindex{TBT.ChemPot!<>!mu}%

    Define the chemical shift (an energy) for this chemical
    potential. One may specify the shift in terms of the applied bias
    using \fdf*{V/<integer>} instead of explicitly typing the energy.

    \option[ElectronicTemperature|Temp|kT]%
    \fdfindex{TBT.ChemPot.<>!ElectronicTemperature}%
    \fdfindex{TBT.ChemPot.<>!Temp}%
    \fdfindex{TBT.ChemPot.<>!kT}%

    Specify the electronic temperature (as an energy or in
    Kelvin). This defaults to \fdf*{TS.ElectronicTemperature}.

    One may specify this in units of \fdf*{TS.ElectronicTemperature} by
    using the unit \fdf*{kT}.

  \end{fdfoptions}

  It is important to realize that the parameterization of the voltage
  into the chemical potentials enables one to have a \emph{single}
  input file which is never required to be changed, even when changing
  the applied bias.

\end{fdfentry}

These options complicate the input sequence for regular $2$ electrode
which is unfortunate. 



\subsection{Electrode configuration}

The electrodes are defining the semi-infinite region that is coupled
to the scattering region.

\tbtrans\ is a fully $N$ electrode calculator. Thus the input for such
setups is rather complicated.

\tbtrans\ defaults to read the \tsiesta\ electrodes and as such one
may replace \fdf*{TBT} by \fdf*{TS} and \tbtrans\ will still
work. However, the \fdf*{TBT} has precedence. 

If there is only $1$ chemical potential all electrodes will default
to use this chemical potential, thus for non-bias calculations there
is no need to specify the chemical potential
(\fdf{TBT.Elec.<>!chemical-potential}).  

\begin{fdfentry}{TBT.Elecs}[block]

  Each line denote an electrode which may be queried in
  \fdf{TBT.Elec.<>} for its setup.
  
\end{fdfentry}

\begin{fdfentry}{TBT.Elec.<>}[block]

  Each line represents a setting for electrode \fdf*{<>}.
  There are a few lines that \emph{must} be present, \fdf*{HS},
  \fdf*{semi-inf-dir}, \fdf*{electrode-pos}, \fdf*{chem-pot} (only if
  \fdf{TBT!Voltage} is not $0$).

  If there are some settings that you only want to take effect in
  \tbtrans\ calculations you can prefix the option with
  \fdf*{tbt.Eta}, e.g. which will only be used for the \tbtrans\
  calculations. Note that \emph{all} \fdf*{tbt.*} options must be
  located at the end of the block. This may be particularly useful
  with respect to the Green function file options.

  \begin{fdfoptions}

    \option[HS]%
    \fdfindex*{TBT!Elec.<>!HS}%
    The electronic structure information from the initial
    electrode calculation. This file retains the geometrical
    information as well as the Hamiltonian, overlap matrix and the
    Fermi-level of the electrode.
    %
    This is a file-path and the electrode \sysfile{TSHS} need not be
    located in the simulation folder.

    \tbtrans\ also reads NetCDF4 files which contain the electronic
    structure. This may be created using \sisl.

    \option[semi-inf-direction|semi-inf-dir|semi-inf]%
    \fdfindex*{TBT.Elec.<>!semi-inf-direction}%
    The semi-infinite direction of the electrode with respect to the
    electrode unit-cell. 

    \note this has nothing to do with the scattering region unit cell,
    \tbtrans\ will figure out the alignment of the electrode unit-cell
    and the scattering region unit-cell.

    \option[chemical-potential|chem-pot|mu]%
    \fdfindex*{TBT.Elec.<>!chemical-potential}%
    The chemical potential that is associated with this
    electrode. This is a string that should be present in the
    \fdf{TBT.ChemPots} block in case there is a bias applied in the
    calculation. 

    \option[electrode-position|elec-pos]%
    \fdfindex*{TBT.Elec.<>!electrode-position}%
    The index of the electrode in the scattering region.
    This may be given by either \fdf*{elec-pos <idx>}, which refers to
    the first atomic index of the electrode residing at index
    \fdf*{<idx>}. Else the electrode position may be given via
    \fdf*{elec-pos end <idx>} where the last index of the electrode
    will be located at \fdf*{<idx>}.

    \option[used-atoms]%
    \fdfindex*{TBT.Elec.<>!used-atoms}%
    Number of atoms from the electrode calculation that is used in the
    scattering region as electrode. This may be useful when the
    periodicity of the electrodes forces extensive electrodes in the
    semi-infinite direction.

    \note do not set this if you use all atoms in the electrode.

    \option[Bulk]%
    \fdfindex*{TBT.Elec.<>!Bulk}%
    Logical controlling whether the Hamiltonian of the electrode
    region in the scattering region is enforced \emph{bulk} or whether
    the Hamiltonian is taken from the scattering region elements.

    \option[tbt.Gf/Gf]%
    \fdfindex*{TBT.Elec.<>!Gf}%
    String with filename of the surface Green function data. This may
    be used to place a common surface Green function file in a top
    directory which may then be used in all calculations using the
    same electrode and the same contour. 
    %
    If many calculations are performed this will heavily increase
    performance at the cost of disk-space.

    
    \option[Eta]%
    \fdfindex*{TBT.Elec.<>!Eta}%
    \fdfdepend{TBT.Elecs!Eta}%
    Control the imaginary energy ($\eta$) of the surface Green
    function for this electrode.

    \note if this energy is negative the complex value associated with
    the contour is used. This is particularly useful when providing a
    user-defined contour. Ensure that all imaginary values are larger
    than $0$ as otherwise \tbtrans\ may seg-fault.

    \note for \phtrans\ calculations you are highly encouraged to
    change this value since the default ($1\,\mathrm{meV}$) is very
    low.

    \option[Accuracy]%
    \fdfindex*{TBT.Elec.<>!Accuracy}%
    \fdfdepend{TBT.Elecs!Accuracy}%
    Control the convergence accuracy required for the self-energy
    calculation when using the Lopez-Sanchez, Lopez-Sanchez iterative
    scheme.

    \note advanced use \emph{only}.

    \option[Bloch]%
    \fdfindex*{TBT.Elec.<>!Bloch}%
    $3$ integers are present on this line which each denote the number
    of times bigger the scattering region electrode is compared to the
    electrode, in each lattice direction. Remark that these expansion
    coefficients are with regard to the electrode unit-cell.
    This is denoted ``Bloch'' because it is an expansion based on
    Bloch waves.

    Please see \emph{Matching electrode coordinates: basic rules} in
    the \siesta\ manual for details.

    \option[Bloch-A/a1|B/a2|C/a3]%
    \fdfindex{TBT.Elec.<>!Bloch}%
    Specific Bloch expansions in each of the electrode unit-cell
    direction. See \fdf*{Bloch} for details.

    \option[pre-expand]%
    \fdfindex*{TBT.Elec.<>!pre-expand}%
    String denoting how the expansion of the surface Green function
    file will be performed. This only affects the Green function file
    if \fdf*{Bloch} is larger than 1. By default the Green function
    file will contain the fully expanded surface Green function, but
    not Hamiltonian and overlap matrices (\fdf*{Green}). One may
    reduce the file size by setting this to \fdf*{Green} which only
    expands the surface Green function. Finally \fdf*{none} may be
    passed to reduce the file size to the bare minimum.
    %
    For performance reasons \fdf*{all} is preferred. 

    \option[out-of-core]%
    \fdfindex*{TBT.Elec.<>!out-of-core}%
    If \fdftrue\ the GF files are created which contain the surface
    Green function. Setting this to \fdftrue\ may be advantageous when
    performing many calculations using the same $k$ and energy grid
    using the same electrode. In those case this will heavily increase
    throughput. 
    % 
    If \fdffalse\ (default) the surface Green function will be
    calculated when needed. 

    \note simultaneous calculations may read the same GF file.

    \option[Gf-Reuse]%
    \fdfindex*{TBT.Elec.<>!Gf-Reuse}%
    \fdfdepend{TBT.Elec.<>!out-of-core}%
    Logical deciding whether the surface Green function file should be
    re-used or deleted.
    %
    If this is \fdffalse\ the surface Green function file is deleted
    and re-created upon start.

    \option[check-kgrid]%
    \fdfindex*{TBT.Elec.<>!check-kgrid}%
    For $N$ electrode calculations the $\mathbf k$ mesh will sometimes
    not be equivalent for the electrodes and the device region
    calculations. However, \tbtrans\ requires that the device and
    electrode $\mathbf k$ samplings are commensurate. This flag
    controls whether this check is enforced. 

    \note only use if fully aware of the implications (for
    tight-binding calculations this may safely be set to \fdffalse).

  \end{fdfoptions}
  
\end{fdfentry}

There are several flags which are globally controlling the variables
for the electrodes (with \fdf{TBT.Elec.<>} taking precedence).

\begin{fdflogicalT}{TBT.Elecs!Bulk}

  This globally controls how the Hamiltonian is treated in all
  electrodes. 
  %
  See \fdf{TBT.Elec.<>!Bulk}.
  
\end{fdflogicalT}

\begin{fdfentry}{TBT.Elecs!Eta}[energy]<$1\,\mathrm{meV}$>
  
  Globally control the imaginary energy ($\eta$) used for the surface
  Green function calculation. This $\eta$ value is \emph{not} used in
  the device region.
  %
  See \fdf{TBT.Elec.<>!Eta} for extended details on the usage of this
  flag.
  
\end{fdfentry}

\begin{fdfentry}{TBT.Elecs!Accuracy}[energy]<$10^{-13}\,\mathrm{eV}$>
  
  Globally control the accuracy required for convergence of the self-energy.
  %
  See \fdf{TBT!Elec.<>!Accuracy}.
  
\end{fdfentry}

\begin{fdflogicalF}{TBT.Elecs!Neglect.Principal}

  If this is \fdffalse\ \tbtrans\ dies if there are connections beyond
  the principal cell.

  \note set this to \fdftrue\ with care, non-physical results may
  arise. Use at your own risk!

\end{fdflogicalF}  

\begin{fdflogicalF}{TBT.Elecs!Out-of-core}

  This enables reusing the self-energies by storing them on-disk
  (\fdftrue). The surface Green function files may be large files but
  heavily increases throughput if one performs several transport
  calculations using the same electrodes.

  You are encouraged to set this to \fdftrue\ to reduce computations. 
  %
  See \fdf{TBT.Elec.<>!out-of-core}.

  Currently this option is not compatible with \fdf{TBT.T.Bulk} and
  \fdf{TBT.DOS.Elecs}, and the bulk transmission and bulk DOS will not
  be calculated if this option is \fdftrue.
  
\end{fdflogicalF}

\begin{fdflogicalT}{TBT.Elecs!Gf.Reuse}
  \fdfdepend{TBT.Elecs!Out-of-core}
  
  Globally control whether the surface Green function files should
  be re-used (\fdftrue) or re-created (\fdffalse).
  %
  See \fdf{TBT.Elec.<>!Gf-Reuse}.
  
\end{fdflogicalT}


\begin{fdfentry}{TBT!Elecs!Coord.EPS}[length]<$10^{-4}\,\mathrm{Bohr}$>

  When using Bloch expansion of the self-energies one may experience
  difficulties in obtaining perfectly aligned electrode coordinates.

  This parameter controls how strict the criteria for equivalent
  atomic coordinates is. If \tbtrans\ crashes due to mismatch between
  the electrode atomic coordinates and the scattering region
  calculation, one may increase this criteria. This should only be
  done if one is sure that the atomic coordinates are almost similar
  and that the difference in electronic structures of the two may be
  negligible.
  
\end{fdfentry}


\subsubsection{Principal layer interactions} %
\index{electrode!principal layer}%

It is \emph{extremely} important that the electrodes only interact
with one neighboring supercell due to the self-energy
calculation. \tbtrans\ will print out a block as this
\begin{fdfexample}
 <> principal cell is perfect!
\end{fdfexample}
if the electrode is correctly setup and it only interacts with its
neighboring supercell.
%
In case the electrode is erroneously setup, something similar to the
following will be shown in the output file.
\begin{fdfexample}
 <> principal cell is extending out with 96 elements:
    Atom 1 connects with atom 3
    Orbital 8 connects with orbital 26
    Hamiltonian value: |H(8,6587)|@R=-2 =  0.651E-13 eV
    Overlap          :  S(8,6587)|@R=-2 =   0.00    
\end{fdfexample}
It is imperative that you have a \emph{perfect} electrode as otherwise
nonphysical results will occur.


\subsection{Calculation settings}

The calculation time is currently governed by two things:
\begin{enumerate}
  
  \item%
  the size of the device region, 

  \item%
  and by the partitioning of the block-tri-diagonal matrix.
  
\end{enumerate}
The first may be controlled via \fdf{TBT.Atoms!Device}. If one is only
interested in transmission coefficients this flag is encouraged to
select the minimum number of atoms that will successfully run the
calculation. Please see the flag entry for further details.

Secondly there is, currently, no way to determine the most optimal
block-partitioning of a banded matrix and \tbtrans\ allows several
algorithms to determine an optimal partitioning scheme. The following
flag controls the partitioning for the device region.


\begin{fdflogicalF}{TBT!Analyze}

  As the pivoting algorithm \emph{highly} influences the performance
  and throughput of the transport calculation it is crucial to select
  the best performing algorithm available. This option tells \tbtrans\
  to analyze the pivoting table for nearly all the implemented
  algorithms and print-out information about them.

  \note we advice users to \emph{always} run an analyzation step prior
  to actual calculation and select the \emph{best} BTD format. This
  analyzing step is very fast and can be performed on small
  work-station computers, even on systems of $\gg10,000$ orbitals.

  To run the analyzing step you may do:
  \begin{shellexample}
    tbtrans -fdf TBT.Analyze RUN.fdf > analyze.out
  \end{shellexample}
  note that there is little gain on using MPI and it should complete
  within a few minutes, no matter the number of orbitals.

  Choosing the best one may be difficult. Generally one should choose
  the pivoting scheme that uses the least amount of memory. However,
  one should also choose the method with the largest block-size being
  as small as possible. As an example:
  \begin{shellexample}
TBT.BTD.Pivot.Device atom+GPS
...
    BTD partitions (7): 
     [ 2984, 2776, 192, 192, 1639, 4050, 105 ]
    BTD matrix block size [max] / [average]: 4050 /   1705.429
    BTD matrix elements in % of full matrix:   47.88707 %

TBT.BTD.Pivot.Device atom+GGPS
...
    BTD partitions (6): 
     [ 2880, 2916, 174, 174, 2884, 2910 ]
    BTD matrix block size [max] / [average]: 2916 /   1989.667
    BTD matrix elements in % of full matrix:   48.62867 %

  \end{shellexample}
  Although the GPS method uses the least amount of memory, the GGPS
  will likely perform better as the largest block in GPS is $4050$
  vs. $2916$ for the GGPS method. 

\end{fdflogicalF}


\begin{fdfentry}{TBT.BTD!Pivot.Device}[string]<atom-\nonvalue{largest
      overlapping electrode}>

  Decide on the partitioning for the BTD matrix. One may denote either
  \fdf*{atom+} or \fdf*{orb+} as a prefix which does the analysis on
  the atomic sparsity pattern or the full orbital sparsity pattern,
  respectively. If neither are used it will default to \fdf*{atom+}.

  \begin{fdfoptions}

    \option[<elec-name>|CG-<elec-name>]%
    The partitioning will be a connectivity graph starting from the
    electrode denoted by the name. This name \emph{must} be found in
    the \fdf{TBT!Elecs} block. One can append more than one electrode
    to simultaneously start from more than 1 electrode. This may be
    necessary for multi-terminal calculations.

    \note One may append an optional setting \fdf*{front} or
    \fdf*{fan} which makes the connectivity graph to consider the
    geometric front of the atoms. For extreme scale simulations or
    tight-binding calculations with constrictions this may improve the
    BTD matrix substantially because it splits the unit-cell into
    segments of equal width.

    \option[rev-CM] %
    Use the reverse Cuthill-McKee for pivoting the matrix elements to
    reduce bandwidth. One may omit \fdf*{rev-} to use the standard
    Cuthill-McKee algorithm (not recommended).

    This pivoting scheme depends on the initial starting
    electrodes, append \fdf*{+<elec-name>} to start the Cuthill-McKee
    algorithm from the specified electrode.

    \option[GPS] %
    Use the Gibbs-Poole-Stockmeyer algorithm for reducing the
    bandwidth.

    \option[GGPS] %
    Use the generalized Gibbs-Poole-Stockmeyer algorithm for reducing
    the bandwidth.

    \option[PCG] %
    Use the perphiral connectivity graph algorithm for reducing the
    bandwidth.

    This pivoting scheme \emph{may} depend on the initial starting
    electrode(s), append \fdf*{+<elec-name>} to initialize the PCG
    algorithm from the specified electrode.

  \end{fdfoptions}

  Examples are
  \begin{fdfexample}
    TBT.BTD.Pivot.Device atom+GGPS
    TBT.BTD.Pivot.Device GGPS
    TBT.BTD.Pivot.Device orb+GGPS
    TBT.BTD.Pivot.Device orb+PCG
    TBT.BTD.Pivot.Device orb+PCG+Left
    TBT.BTD.Pivot.Device orb+rev-CM+Right
  \end{fdfexample}
  where the first two are equivalent. The 3rd and 4th are more heavily
  on analysis and will typically not improve the bandwidth reduction.

\end{fdfentry}

\begin{fdfentry}{TBT.BTD!Pivot.Elec.<>}[string]<atom-\nonvalue{<>}>
  \fdfdepend{TBT.Atoms!Device}

  If \fdf{TBT.Atoms!Device} has been set to a reduced region the
  electrode self-energies must be \emph{down-folded} through the atoms
  not part of the device-region. In this case these down-fold regions
  can also be considered a BTD matrix which may be optimized
  separately from the device region BTD matrix.

  This option have all available options as described in
  \fdf{TBT.BTD!Pivot.Device} but one will generally find the best
  pivoting scheme by using the default (\fdf*{atom-<>}) which is the
  atomic connectivity graph from the electrode it-self.

  It may be advantageous to use \fdf*{atom-<>-front} for very large
  tight-binding calculations where the device region is chosen far
  from this electrode and normal to the electrode-plane. 

\end{fdfentry}

\begin{fdfentry}{TBT.BTD!Spectral}[string]<propagation|column>

  Method used for calculating the spectral function ($\mathbf A_i$).
  For 4 or more electrodes the \fdf*{column} option is the default
  while \fdf*{propagation} is the default for less electrodes.

  \note this option may heavily influence performance. Test for a
  single $k$-point and single energy point to figure out the
  implications of using one over the other.
  
\end{fdfentry}

\begin{fdflogicalF}{TBT.BTD!Pivot.Graphviz}

  Create Graphviz\footnote{\url{www.graphviz.org}} compatible input
  files for the pivoting tables for all electrodes,
  \sysfile{TBT.<elec>.gv}, and the device, \sysfile{TBT.gv}.

  These files may be processed by Graphviz display commands
  \program{neato} etc.

  \begin{shellexample}
    neato -x <>
    neato -x -Tpdf <> -o graph.pdf
    neato -x -Tpng <> -o graph.png
  \end{shellexample}
  
\end{fdflogicalF}


\subsection{Input/Output}

\tbtrans\ IO is mainly relying on the NetCDF4 library and full
capability is only achieved if compiled with this library.

Several fdf-flags control how \tbtrans\ performs IO.


\begin{fdfentry}{TBT.CDF!Precision}[string]<single|float|double>

  Specify the precision used for storing the quantities in the
  NetCDF4.

  \fdf*{single} takes half the disk-space as \fdf*{double} and will
  generally retain a sufficient precision of the quantities. 

  \fdf*{single} and \fdf*{float} are equivalent.

  \note all calculations are performed using \fdf*{double} so this is
  \emph{only} a storage precision.
  
\end{fdfentry}

\begin{fdfentry}{TBT.CDF!DOS.Precision}[string]<{<\fdf{TBT.CDF!Precision}>}>

  Specify the precision used for storing DOS in NetCDF4.

\end{fdfentry}

\begin{fdfentry}{TBT.CDF!T.Precision}[string]<{<\fdf{TBT.CDF!Precision}>}>

  Specify the precision used for storing transmission function in NetCDF4.

\end{fdfentry}

\begin{fdfentry}{TBT.CDF!T.Eig.Precision}[string]<{<\fdf{TBT.CDF!Precision}>}>

  Specify the precision used for storing transmission eigenvalues in NetCDF4.
  
\end{fdfentry}

\begin{fdfentry}{TBT.CDF!Current.Precision}[string]<{<\fdf{TBT.CDF!Precision}>}>

  Specify the precision used for storing orbital current in NetCDF4.

  \note This is heavily advised to be in single precision as this may
  easily use large amounts of disk-space if in double precision.
  
\end{fdfentry}

\begin{fdfentry}{TBT.CDF!DM.Precision}[string]<{<\fdf{TBT.CDF!Precision}>}>

  Specify the precision used for storing density matrices in NetCDF4.

  \note This is heavily advised to be in single precision as this may
  easily use large amounts of disk-space if in double precision.
  
\end{fdfentry}

\begin{fdfentry}{TBT.CDF!COOP.Precision}[string]<{<\fdf{TBT.CDF!Precision}>}>

  Specify the precision used for storing COOP and COHP curves in NetCDF4.

  \note This is heavily advised to be in single precision as this may
  easily use large amounts of disk-space if in double precision.
  
\end{fdfentry}

\begin{fdfentry}{TBT.CDF!Compress}[integer]<0>

  Specify whether the NetCDF4 files stored will be compressed. This
  may heavily reduce disk-utilization at the cost of some performance.

  This number must be between 0 (no compression) and 9 (maximum
  compression). A higher compression is more time consuming and a good
  compromise between speed and compression is 3.

  \note one may subsequently to a \tbtrans\ compilation compress a
  NetCDF4 file using:
\begin{shellexample}
   nccopy -d 3 siesta.TBT.nc newsiesta.TBT.nc
 \end{shellexample}

 \note one \emph{can not} do parallel I/O together with compression.
  
\end{fdfentry}

\begin{fdflogicalF}{TBT.CDF!MPI}

  Whether the IO is performed in parallel. If using a large amount of
  MPI processors this may increase performance.

  \note the actual performance increase is \emph{very} dependent on
  your hardware support for parallel IO.

  \note this automatically sets the compression to 0 (one cannot
  compress and perform parallel IO).

\end{fdflogicalF}

\subsubsection{Self-energy}
\label{sec:self-energy}

\tbtrans\ enables the storage of the self-energies from the electrodes
in selected regions. I.e. in a two electrode setup the self-energies
may be ``down-folded'' to a region of interest (say molecule etc.) and
then saved.

This feature enables one to easily use self-energies in Python for
subsequent analysis etc. It is only available if compiled against
NetCDF4. 

\begin{fdflogicalF}{TBT!SelfEnergy.Save}

  Store the self-energies of the electrodes. The self-energies are
  first down-folded into the device region (see
  \fdf{TBT.Atoms!Device}).

\end{fdflogicalF}

\begin{fdflogicalF}{TBT!SelfEnergy.Save.Mean}

  If \fdftrue\ the down-folded self-energies will be $k$-averaged
  after \tbtrans\ has finished.
  
\end{fdflogicalF}


\begin{fdflogicalF}{TBT!SelfEnergy.Only}
  
  If \fdftrue\ this will \emph{only} calculate and store the
  down-folded self-energies. No physical quantities will be calculated
  and \tbtrans\ will quit. 
  
\end{fdflogicalF}


\begin{fdfentry}{TBT.CDF!SelfEnergy.Precision}[string]<{<\fdf{TBT.CDF!Precision}>}>
  \fdfdepend{TBT.CDF!Precision}

  Specify the precision used for storing the self-energies in NetCDF4.

\end{fdfentry}


\begin{fdfentry}{TBT.CDF!SelfEnergy.Compress}[integer]%
  <{<\fdf{TBT.CDF!Compress}>}>
  \fdfdepend{TBT.CDF!Compress}

  Specify the compression of the self-energies in NetCDF4.
  
\end{fdfentry}


\begin{fdflogicalF}{TBT.CDF!SelfEnergy.MPI}%
  \fdfdepend{TBT.CDF!MPI}

  If \fdftrue\ \tbtrans\ will use MPI when writing the NetCDF files
  containing the downfolded self-energies.

\end{fdflogicalF}


\subsubsection{Projected transmissions}

The transmission through a scattering region is determined by the
electrodes band-structure and the energy levels for the scattering
part. In for instance molecular electronics it is often useful to
determine which molecular orbitals are responsible for the
transmission as well as knowing their hybridization with the substrate
(electrodes).

\tbtrans\ implements an advanced projection method which splits the
transmission and DOS into eigenstate projectors.

In the following we concentrate on a $2$ terminal device while it may
be used for $N$ electrode calculations. 
%
One important aspect of projection is that the self-energies that are
to be projected \emph{must} be fully located on the projection
region. \tbtrans\ will die if this is not enforced.
%
A projection can \emph{only} be performed if the down-folding of the
self-energies for the projected electrode is fully encapsulated in the
device region (\fdf{TBT.Atoms!Device}). I.e. one should reduce the
device region such that any couplings from the electrodes only couple
into the projection region. Generally for the most simple projections
the device region should be equivalent to the projection region in
case there is only one projection region.

These projections should not be confused with local DOS which may be
obtained if compiled with the NetCDF4 library and via the use of
\sisl, see Sec.~\ref{sec:physical}.

\note if the \fdf{TBT.Projs} block is defined, then the
\fdf{TBT.Projs!T} block is required in the input unless
\fdf{TBT.Projs!Init} is \fdftrue.


\begin{fdfentry}{TBT.Projs}[block]

  List of molecular projections used:
\begin{fdfexample}
  %block TBT.Projs
    M-L
    M-R
  %endblock
\end{fdfexample}

  This tells \tbtrans\ that two projections will exist. Each
  projection setup will be read in \fdf{TBT.Proj!<>}.

  There is no limit to the number of projection molecules.
  
\end{fdfentry}

\begin{fdfentry}{TBT.Proj!<>}[block]

  Block that designates a molecular projection by the names specified
  in the \fdf{TBT.Projs} block.

  This block determines how each projection is interpreted, it
  consists of several options defined below:
  \begin{fdfoptions}
    
    \option[atom]%
    \fdfindex{TBT.Proj!<>!atom}%
    \fdfindex{TBT.Proj!<>!position}%
    
    There may be several \fdf*{atom} lines. The full set of atomic
    indices will be used as a sub-space for the Hamiltonian. 
    %
    The atoms may be defined via these variants
    \begin{fdfoptions}
      
      \option[{atom \emph{A} [\emph{B} [\emph{C} [\dots]]]}]%
      A sequence of atomic indices which are used for the projection.

      % Generic input (compatible with the <= 4.0)
      \option[{atom from \emph{A} to \emph{B} [step \emph{s}]}]%
      Here atoms \emph{A} up to and including \emph{B} are
      used.
      %
      If \fdf*{step <s>} is given, the range
      \emph{A}:\emph{B} will be taken in steps of \emph{s}.

      \begin{fdfexample}
        atom from 3 to 10 step 2
      \end{fdfexample}
      will add atoms 3, 5, 7 and 9.

      \option[{atom from \emph{A} plus/minus \emph{B} [step
          \emph{s}]}]%
      Atoms \emph{A} up to and including $\emph{A}+\emph{B}-1$
      are added to the projection.
      %
      If \fdf*{step <s>} is given, the range
      \emph{A}:$\emph{A}+\emph{B}-1$ will be taken in steps of
      \emph{s}.

      % Generic input (compatible with the <= 4.0)
      \option[atom {[<A>, \emph{B} -\mbox{}- \emph{C} [step
          \emph{s}], \emph{D}]}]%
      Equivalent to \fdf*{from \dots to} specification, however in a
      shorter variant. Note that the list may contain arbitrary number
      of ranges and/or individual indices.

      \begin{fdfexample}
        atom [2, 3 -- 10 step 2, 6]
      \end{fdfexample}
      will add atoms 2, 3, 5, 7, 9 and 6.

    \end{fdfoptions}


    \option[Gamma]%
    \fdfindex{TBT.Proj!<>!Gamma}%
    
    Logical variable which determines whether the projectors are the
    $\Gamma$-point projectors, or the $k$ resolved ones. For
    $\Gamma$-only calculations this has no effect.
    % 
    If the eigenstates are non-dispersive in the Brillouin zone there
    should be no difference between \fdftrue\ or \fdffalse.

    % I do not think this is a problem. |>e^{i\theta} * e^{-i\theta}<|
    % == |><|
    % \beware{Diagonalisation of a $\kk$-point Hamiltonian does not
    % necessarily retain the
    % correct phases, i.e. $\ket{\psi}$ and $\ket{\psi}e^{i\theta}$
    % are both solutions,
    % whereas the Green functions will always be set up in the
    % $\ket{\psi}$ basis. Hence
    % doing $\kk$-point resolved need not be correct. This is not a
    % problem for the $\Gamma$
    % point as we force the basis to be a real basis.}

    \note it is \emph{very} important to know the dispersion and
    possible band-crossings of the eigenstates if this option is
    \fdffalse. For band-crossings one must manually perform the
    projections for the $k$-points in a stringent manner as the order
    of eigenstates are not retained.


    \option[proj <P-name>]
    \fdfindex{TBT.Proj!<>!proj}%

    Allows to define a projection based on the eigenstates for the
    current molecule.

    The \fdf*{<P-name>} designates the name associated with this
    projection.

    It is parsed like this, in the following $0$ is the Fermi level
    (HOMO $=-1$, LUMO $=1$):
    \begin{fdfoptions}

      \option[level from <E1> to <E2>] %
      Energy eigenstates \fdf*{E1} and \fdf*{E2} will be part of the
      molecular orbitals that constitute this projection

      \option[level from <E> plus <N>] %
      Energy eigenstates between \fdf*{E} and $\fdf*{E}+N-1$ will be
      part of the molecular orbitals that constitute this projection

      \option[level from <E> minus <N>] %
      Energy eigenstates between \fdf*{E} and $\fdf*{E}-N+1$ will be
      part of the molecular orbitals that constitute this projection

      \option[level <E1> <E2> ... <En>] %
      All eigenstates specified will be part of the molecular orbitals
      that constitute this projection

      \option[level {[ <list> ]}] %
      A comma-separated list specification.

      \option[end] %
      All gathered eigenstates so far will constitute the projection
      named \fdf*{<P-name>}
      
    \end{fdfoptions}

    Note that level $0$ refers to the Fermi level, it will be silently
    removed as it is not an eigenstate, so you do not need to think
    about it.

    You can specify named projection blocks as many times as you want.

    To conclude the full projection block here is an example
    describing three different projections for the left molecule in
    \begin{fdfexample}
 %block TBT.Proj.M-L
   # We have 2 atoms on this molecule
   atom from 5 plus 2
   # We only do a Gamma projection
   Gamma .true.
   # We will utilise three different projections on 
   # this molecule
   proj HOMO
    level -1
   end 
   proj LUMO
    level 1
   end
   proj H-plus-L
    level from -1 to 1
   end
 %endblock
    \end{fdfexample}

    Similarly for the right molecule we do
    \begin{fdfexample}
 %block TBT.Proj.M-R
   # We have 2 atoms on this molecule
   atom from 8 plus 2
   # We only do a Gamma projection
   Gamma .true.
   # We will utilise three different projections on 
   # this molecule
   proj HOMO
    level -1
   end 
   proj LUMO
    level 1
   end
   proj H-plus-L
    level from -1 to 1
   end
 %endblock
    \end{fdfexample}
   
  \end{fdfoptions}
 
\end{fdfentry}


\begin{fdflogicalF}{TBT.Proj!<>!States}%

  Save all states for the projection. The saved quantity can be
  post-processed to decipher the locality of each projection.

  In the NetCDF file there will be two variables: \code{state} and
  \code{states} where the former will contain $\SO^{1/2}\ket{i}$,
  while the latter will contain $\ket{i}$.

  \emph{Needed if you wish to select specific molecular orbitals
      dependent on the nature of the molecular orbital.}
  
\end{fdflogicalF}


\begin{fdfentry}{TBT.CDF!Proj.Compress}[integer]<{<\fdf{TBT.CDF!Compress}>}>

  Allows a different compression for the projection file. The
  projection file is typically larger than the default output file, so
  compression of them separately might be needed.
  
\end{fdfentry}

\begin{fdflogicalF}{TBT.Projs!Init}

  Whether \tbtrans\ will only create the projection tables and then
  quit.

  As \tbtrans\ allows to re-use the projection file the user can
  choose to stop right after creation. Specifically it will allow one
  to swap projection states with other projection states. This can be
  useful when bias is applied and the hybridisation ``destroys'' the
  molecule Hamiltonian. After initialising the projection tables the
  user can manually swap them with those calculated at zero bias, thus
  retaining the same projection tables for different bias'.

  Note that for spin calculations you need to utilise the
  \fdf{TBT.Spin} flag to initialise both projection files (spin UP
  \emph{and} spin DOWN) before proceeding with the calculation.
  
\end{fdflogicalF}

\begin{fdflogicalF}{TBT.Projs!Debug}

  Print out additional information regarding the projections. It will
  print out assertion lines orthogonality.

  \emph{Possibly not useful for other than the developers.}
  
\end{fdflogicalF}

\begin{fdfentry}{TBT.Projs!T}[block]%

  As you might specify \emph{many} molecular projections to
  investigate a lot of details of the system it seems perilous to
  always calculate all allowed transmission permutations.

  Instead the user has to supply the permutations of transport that is
  calculated. This block will let the user decide which to calculate
  and which to not.

  In the following \fdf*{Left}(L)/\fdf*{Right}(R) corresponds to
  $T=\Tr[\mathbf G\boldsymbol\Gamma_L\mathbf
  G^\dagger\boldsymbol\Gamma_R]$
  where \fdf*{Left}, \fdf*{Right} are found in the \fdf{TBT!Elecs}
  block.

  \begin{fdfoptions}

    \option[from <proj-L> to]%
    Projects $\boldsymbol\Gamma_L$ on to the \fdf*{<projection>}
    before doing the R projections.

    The R projections are constructed in the following lines until
    \fdf*{end} is seen.

    \begin{fdfoptions}
      
      \option[<proj-R>]%
      Projects $\boldsymbol\Gamma_R$ on to the \fdf*{<projection>}
      which then calculates the transmission
    \end{fdfoptions}
  
  \end{fdfoptions}

  Each projection can be represented in three different ways:

  \begin{fdfoptions}
    
    \option[<elec>] %
    Makes no projection on the scattering matrix

    \option[<elec>.<name>] %
    Makes all permutations of the projections attached to the molecule
    named \fdf*{<name>}
  
    \option[<elec>.<name>.<P-name>] %
    Projects the named projection \fdf*{<P-name>} from molecule
    \fdf*{<name>} onto electrode \fdf*{<elec>}

  \end{fdfoptions}

  An example input for projection two molecules could be:
  \begin{fdfexample}
 %block TBT.Projs.T
   from Left.M-L.HOMO to
     Right.M-R
     Right
   end
   from Left.M-L.LUMO to
     Right.M-R.LUMO
   end
 %endblock
  \end{fdfexample}
  which will be equivalent to the more verbose
  \begin{fdfexample}
 %block TBT.Projs.T
   from Left.M-L.HOMO to
     Right.M-R.HOMO
     Right.M-R.LUMO
     Right.M-R.H-plus-L
     Right
   end
   from Left.M-L.LUMO to
     Right.M-R.LUMO
   end
 %endblock
  \end{fdfexample}
  
\end{fdfentry}

\bgroup
\def\Gf{\mathbf G}
\def\Gam{\boldsymbol\Gamma}

This will calculate the transport using all these equations
\begin{align}
  \label{eq:T:proj:1}
  T_{\ket{H_1},\ket{H_2}} &=\Tr\big[\Gf\kb{H_1}\Gam_L\kb{H_1}\Gf^\dagger\kb{H_2}\Gam_R\kb{H_2}\big]
  \\
  \label{eq:T:proj:2}
  T_{\ket{H_1},\ket{L_2}} &=\Tr\big[\Gf\kb{H_1}\Gam_L\kb{H_1}\Gf^\dagger\kb{L_2}\Gam_R\kb{L_2}\big]
  \\
  \label{eq:T:two:one}
  T_{\ket{H_1},\ket{H_2}+\ket{L_2}}
  &=\Tr\Big[\Gf\kb{H_1}\Gam_L\kb{H_1}\Gf^\dagger\big(\kb{H_2}+\kb{L_2}\big)\Gam_R\big(\kb{H_2}+\kb{L_2}\big)\Big]
  \\
  \label{eq:T:single:one}
  T_{\ket{H_1},R} &=\Tr\Big[\Gf\kb{H_1}\Gam_L\kb{H_1}\Gf^\dagger\Gam_R\Big]
  \\
  \label{eq:T:proj:5}
  T_{\ket{L_1},\ket{L_2}} &=\Tr\big[\Gf\kb{L_1}\Gam_L\kb{L_1}\Gf^\dagger\kb{L_2}\Gam_R\kb{L_2}\big]
\end{align}
\egroup %
Notice that \ref{eq:T:two:one} is equivalent to \ref{eq:T:single:one}
in our two state model.

Note that removing an explicit named projection allows easy creation
of all available permutations of the projection states associated with
the molecule.


\begin{fdflogicalF}{TBT.Projs!Only}

  Whether \tbtrans\ will not calculate non-projected transmissions. If
  you are only interested in the projection transmissions and/or have
  already calculated the non-projected transmissions you can use this
  option.
  
\end{fdflogicalF}

\begin{fdflogicalF}{TBT.Projs!DOS.A}%

  Save the spectral density of states for the projections. In case you
  have set \fdf{TBT.DOS!A} this will default to that flag.

  In case any of \fdf{TBT.Projs!Current.Orb}, \fdf{TBT.Projs!DM.A},
  \fdf{TBT.Projs!COOP.A} or \fdf{TBT.Projs!COHP.A} is \fdftrue\ this
  flag will be set to \fdftrue\ as well.

\end{fdflogicalF}

\begin{fdflogicalF}{TBT.Projs!Current.Orb}%

  Will calculate and save the orbital current for the device with the
  projections.

  The orbital current will be saved in the same sparsity pattern as
  the cut-out device region sparsity pattern. 
  
\end{fdflogicalF}

\begin{fdflogicalF}{TBT.Projs!DM.A}
  \fdfdepend{TBT.Atoms!Device}

  Calculate the energy and $k$-resolved density matrix for the
  projected spectral functions. The density matrix may be used to
  construct real-space LDOS profiles.
  
\end{fdflogicalF}

\begin{fdflogicalF}{TBT.Projs!COOP.A}
  \fdfdepend{TBT.Atoms!Device}

  Calculate COOP from the projected spectral function in the device
  region.

\end{fdflogicalF}

\begin{fdflogicalF}{TBT.Projs!COHP.A}
  \fdfdepend{TBT.Atoms!Device}

  Calculate COHP from the projected spectral function in the device
  region.

\end{fdflogicalF}


\begin{fdflogicalF}{TBT.Projs!T.All}%

  Same as \fdf{TBT.T!All}, but for projections. If differing
  projections are performed on the scattering states the transmission
  will not be reversible. You can turn on all projection operations
  using this flag.

\end{fdflogicalF}

\begin{fdflogicalF}{TBT.Projs!T.Out}%

  Same as \fdf{TBT.T!Out} for projections. 
  
\end{fdflogicalF}



\subsubsection{NetCDF4 support}

\tbtrans\ stores all relevant physical quantities in the
\sysfile{TBT.nc} file which retains orbital resolved DOS, orbital
currents, transmissions, transmission eigenvalues, etc. One may use
\sisl\ to easily analyze and extract quantities from this file using
Python. 

These files are created if NetCDF4 support is enabled
\begin{description}
  \def\fillsee#1{\hfill see \fdf{#1}}

  \item[\sysfile{TBT.nc}] %

  File which contain nearly everything calculated in \tbtrans.
  The structure of this file is a natural tree structure to
  accommodate $N$ electrode output.

  \item[\sysfile{TBT.SE.nc}] \fillsee{TBT!SelfEnergy.Save}%

  Down-folded self-energies are stored in this file. 

  \item[\sysfile{TBT.Proj.nc}] \fillsee{TBT.Projs}%

  Stores projected DOS, transmission and/or orbital currents. Using
  projections for large $k$ and energy sampling will create very large
  files. Ensure that you have a large amount of disk-space available.

  \item[\sysfile{DOS}] \fillsee{TBT.DOS!Gf}%

  The $k$ resolved density of states from the Green function. 

  \item[\sysfile{AVDOS}] \fillsee{TBT.DOS!Gf}%

  The $k$ averaged density of states from the Green function. 

  \item[\sysfile{ADOS\_<>}] \fillsee{TBT.DOS!A}%

  The $k$ resolved density of states from the electrode name \fdf*{<>}.

  \item[\sysfile{AVADOS\_<>}] \fillsee{TBT.DOS!A}%

  The $k$ averaged density of states from the electrode name \fdf*{<>}.

  \item[\sysfile{TRANS\_<1>\_<2>}] \mbox{}%

  The $k$ resolved transmission from \fdf*{<1>} to \fdf*{<2>}. 

  \item[\sysfile{AVTRANS\_<1>\_<2>}] \mbox{}%

  The $k$ averaged transmission from \fdf*{<1>} to \fdf*{<2>}. 

  \item[\sysfile{CORR\_<1>}] \fillsee{TBT.T!Out}%

  The $k$ resolved correction to the transmission for \fdf*{<1>}.

  \item[\sysfile{AVCORR\_<1>}] \fillsee{TBT.T!Out}%

  The $k$ averaged correction to the transmission for \fdf*{<1>}.

  \item[\sysfile{TEIG\_<1>\_<2>}] \fillsee{TBT.T!Eig}%

  The $k$ resolved transmission eigenvalues from \fdf*{<1>} to
  \fdf*{<2>}. 

  \item[\sysfile{AVTEIG\_<1>\_<2>}] \fillsee{TBT.T!Eig}%

  The $k$ averaged transmission eigenvalues from \fdf*{<1>} to
  \fdf*{<2>}. 

  \item[\sysfile{CEIG\_<1>}] \fillsee{TBT.T!Out}%

  The $k$ resolved correction eigenvalues for \fdf*{<1>}.

  \item[\sysfile{AVCEIG\_<1>}] \fillsee{TBT.T!Out}%

  The $k$ averaged correction eigenvalues for \fdf*{<1>}.

  \item[\sysfile{BDOS\_<>}] \fillsee{TBT.DOS!Elecs}/\fdf{TBT.T!Bulk}%

  The $k$ resolved bulk density of states of electrode \fdf*{<>}.

  \item[\sysfile{AVBDOS\_<>}] \fillsee{TBT.DOS!Elecs}/\fdf{TBT.T!Bulk}%

  The $k$ averaged bulk density of states of electrode \fdf*{<>}.

  \item[\sysfile{BTRANS\_<>}] \fillsee{TBT.DOS!Elecs}/\fdf{TBT.T!Bulk}%

  The $k$ resolved bulk transmission of electrode \fdf*{<>}.
  
  \item[\sysfile{AVBTRANS\_<>}] \fillsee{TBT.DOS!Elecs}/\fdf{TBT.T!Bulk}%

  The $k$ averaged bulk transmission of electrode \fdf*{<>}.

\end{description}

All the above files will only be created if \tbtrans\ was successfully
executed and their respective options was enabled.

\subsubsection{No NetCDF4 support}

In case \tbtrans\ is not compiled with NetCDF4 support \tbtrans\ is
heavily limited in functionality and subsequent analysis. Particularly
the DOS quantities are not orbital resolved. Also none of the
quantities will be $k$ averaged, this is required to be done
externally. 

The following files are created:
\begin{description}
  \def\fillsee#1{\hfill see \fdf{#1}}

  \item[\sysfile{DOS}] \fillsee{TBT.DOS!Gf}%

  The $k$ resolved density of states from the Green function. 

  \item[\sysfile{ADOS\_<>}] \fillsee{TBT.DOS!A}%

  The $k$ resolved density of states from the electrode name \fdf*{<>}.

  \item[\sysfile{TRANS\_<1>\_<2>}] \mbox{}%

  The $k$ resolved transmission from \fdf*{<1>} to \fdf*{<2>}. 

  \item[\sysfile{TEIG\_<1>\_<2>}] \fillsee{TBT.T!Eig}%

  The $k$ resolved transmission eigenvalues from \fdf*{<1>} to
  \fdf*{<2>}. 

  \item[\sysfile{BDOS\_<>}] \fillsee{TBT.DOS!Elecs}/\fdf{TBT.T!Bulk}%

  The $k$ resolved bulk density of states of electrode \fdf*{<>}.

  \item[\sysfile{BTRANS\_<>}] \fillsee{TBT.DOS!Elecs}/\fdf{TBT.T!Bulk}%

  The $k$ resolved bulk transmission of electrode \fdf*{<>}.
  
\end{description}


\clearpage
\addcontentsline{toc}{section}{Bibliography}
\bibliographystyle{plainnat}
\bibliography{siesta}

% Indices
\clearpage
\addcontentsline{toc}{section}{Index}
\printindex

\printindex[sfiles]
\printindex[sfdf]


\end{document}




%%% Local Variables:
%%% mode: latex
%%% ispell-local-dictionary: "american"
%%% fill-column: 70
%%% TeX-master: t
%%% End:
