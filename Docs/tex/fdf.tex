%
% This TeX file includes the default environments for creating fdf blocks
% in the documentation of SIESTA.
%
% These utility functions have been implemented by:
%  Nick R. Papior, 2016 (nickpapior <at> gmail.com)
%

% Create a new output file for the short-manual.
% This file will contain all the short descriptions of
% the fdf-keys.

% create the correct fdf mode (for the fdf-environment)
\let\fdf@mode\leavevmode

% Create font for fdf-flags
% If this command is changed it changes all 
% |#1| and \fdf{#1} commands.
\NewDocumentCommand\fontfdf{ m }
{%
    \textbf{#1}%
}

% Shorthand command for printing the fdf-key
% correctly, AND adding it to the index.
% Note that | | is equivalent to this.
\NewDocumentCommand\fdf{ s m }
{%
    \fontfdf{#2}%
    \IfBooleanF{#1}{%
        % Create index
        \sindex[sfdf]{#2}%
    }%
}

% Define the skip of a new fdf key
\newcommand\fdfentrybody{\parskip2pt}

\NewDocumentCommand\fdfabstract{ m }{\textlangle#1\textrangle}

\def\fdfentryline@None{None}
% Create the entry for the fdf-key
%
% #1: the name of the fdf-key
% #2: (optional) the type of the input value (string, true/false, physical quantity)
% #3: <> the default value of the fdf-key
\NewDocumentCommand\fdfentryline{ m o D<>{None}}
{ %
    \itemsep=0pt%
    \parskip=0pt%
    \bgroup%
    \raggedright\item%
    % Create fdf-label (with indexing)
    \strut{\fdf{#1}%
        \space%
        \bgroup%
        \def\tmp@None{#3}%
        \ifx\tmp@None\fdfentryline@None%
          \let\tmp@None\fdfabstract%
        \else%
          \let\tmp@None\relax%
        \fi%
        \strut{\color{red!50!black}\tmp@None{\fdf*{#3}}}%
        \egroup %
    }
    \IfValueT{#2}%
    {%
        \hfill\strut{\textit{(#2)}}%
    } %
    \par%
    \egroup%
    \topsep=0pt%
    % Create fdf label
    \fdfentrybody%
}


% Create new fdf input
% This finalizes the call with \fdfenttryline
% which does the actual parsing of the arguments.
% This environment only sets up the enviroment
% to typeset things in.
\NewDocumentEnvironment{fdfentry}{ }
{   %
    % Create correct box-setups
    \list{%
    }{ %
        % Define list
        \leftmargin=2em\itemindent-\leftmargin %
        \def\makelabel##1{\hss##1} %
    }%
    % Create entry line of the argument
    \fdfentryline %
}
{%
    \endlist%
}


% Simple environment which extends 
%   \begin{verbatim}
% 
%   \end{verbatim}
% It is simply intended as placeholder to easily change
% the future behaviour of the environment.
\DefineVerbatimEnvironment{fdfexample}{Verbatim}{xleftmargin=2em}


%%% Local Variables:
%%% mode: latex
%%% TeX-master: "../siesta"
%%% End:
