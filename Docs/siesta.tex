% Manual for the SIESTA program
%
% To generate the printed version:
%
% latex siesta
% makeindex siesta    (Optional if you have a current siesta.ind)
% latex siesta
% [ dvips siesta ]
%
%
\documentclass[11pt]{article}
\usepackage{makeidx}
\usepackage{url}

\tolerance 10000
\textheight 22cm
\textwidth 16cm
\oddsidemargin 1mm
\topmargin -15mm

\makeindex

\parindent=0cm
\baselineskip=14pt
\parskip 5pt

\begin{document}

% TITLE PAGE --------------------------------------------------------------

\begin{titlepage}

\begin{center}

\vspace{1cm}
{\Huge {\sc U s e r' s \, \, G u i d e}}

\vspace{1cm}
\hrulefill
\vspace{1cm}

{\Huge {\bf S I E S T A \, \, 2.1-devel}}

\vspace{1cm}
\hrulefill
\vspace{1cm}

{\Large Dec 13, 2006 }

\vspace{1.5cm}

\hbox{ \hskip 1.5cm
\begin{tabular}{ll}

{\Large Emilio Artacho} &
   \Large{\it University of Cambridge} \\ \\

{\Large Julian D. Gale} &
   \Large{\it Curtin University of Technology, Perth} \\ \\

{\Large Alberto Garc\'{\i}a} &
   \Large{\it Institut de Ci\`encia de Materials, CSIC, Barcelona} \\ \\

{\Large Javier Junquera} &
   \Large{\it Universidad de Cantabria, Santander} \\ \\

{\Large Richard M. Martin} &
   \Large{\it University of Illinois at Urbana-Champaign} \\ \\

{\Large Pablo Ordej\'on} &
   \Large{\it Institut de Ci\`encia de Materials, CSIC, Barcelona} \\ \\

{\Large Daniel S\'anchez-Portal} &
   \Large{\it Unidad de F\'{\i}sica de Materiales,} \\ 
                                 &
   \Large{\it Centro Mixto CSIC-UPV/EHU, San Sebasti\'an} \\ \\

{\Large Jos\'e M. Soler} &
   \Large{\it Universidad Aut\'onoma de Madrid}

\end{tabular}
}

\vspace{0.5cm}
{\Large {\tt http://www.uam.es/siesta} }

\vspace{2mm}
{\Large \tt siesta@uam.es}

\vspace{1cm}
Copyright \copyright\  Fundaci\'on General Universidad Aut\'onoma de Madrid:
E.Artacho, J.D.Gale, A.Garc\'{\i}a, J.Junquera, P.Ordej\'on,
D.S\'anchez-Portal and J.M.Soler, 1996-2006

\end{center}

\end{titlepage}

% END TITLE PAGE --------------------------------------------------------------

\tableofcontents

\newpage



\section{INTRODUCTION}

{\sc Siesta}\index{Siesta@{\sc Siesta}} (Spanish Initiative for 
Electronic Simulations with
Thousands of Atoms) is both a method and its computer program implementation,
to perform electronic structure calculations and {\it ab initio} molecular 
dynamics simulations of molecules and solids. Its main characteristics are:
\begin{itemize}
\item
It uses the standard Kohn-Sham selfconsistent density functional
method in the local density (LDA-LSD) or generalized gradient (GGA)
approximations.
\item
It uses norm-conserving pseudopotentials in its fully nonlocal
(Kleinman-Bylander) form.
\item
It uses atomic orbitals as basis set, allowing unlimited multiple-zeta
and angular momenta, polarization and off-site orbitals. The radial
shape of every orbital is numerical and any shape can be used and provided 
by the user, with the only condition that it has to be of finite support,
i.e., it has to be strictly zero beyond a user-provided distance from the 
corresponding nucleus.
Finite-support basis sets are the key for calculating the Hamiltonian 
and overlap matrices in $O(N)$ operations.
\item
Projects the electron wavefunctions and density onto a real-space
grid in order to calculate the Hartree and exchange-correlation
potentials and their matrix elements.
\item
Besides the standard Rayleigh-Ritz eigenstate method, it allows
the use of localized linear combinations of the occupied orbitals
(valence-bond or Wannier-like functions), making the computer
time and memory scale linearly with the number of atoms.
Simulations with several hundred atoms are feasible with
modest workstations.
\item
It is written in Fortran 90 and memory is allocated dynamically.  
\item
It may be compiled for serial or parallel execution (under MPI).
(Note: This feature might not be available in all distributions.)

\end{itemize}

It routinely provides:
\begin{itemize}
\item Total and partial energies.
\item Atomic forces.
\item Stress tensor.
\item Electric dipole moment.
\item Atomic, orbital and bond populations (Mulliken).
\item Electron density.
\end{itemize}

And also (though not all options are compatible):
\begin{itemize}
\item Geometry relaxation, fixed or variable cell.
\item Constant-temperature molecular dynamics (Nose thermostat).
\item Variable cell dynamics (Parrinello-Rahman).
\item Spin polarized calculations (collinear or not).
\item k-sampling of the Brillouin zone.
\item Local and orbital-projected density of states.
\item Band structure.
\end{itemize}

{\large {\bf References:} }

\begin{itemize}

\item
``Unconstrained minimization approach for electronic computations
that scales linearly with system size"
P. Ordej\'on, D. A. Drabold, M. P. Grumbach and R. M. Martin, 
Phys. Rev. B {\bf 48}, 14646 (1993); 
``Linear system-size methods for electronic-structure calculations"
Phys. Rev. B {\bf 51} 1456 (1995), and references therein.

Description of the order-{\it N} eigensolvers
implemented in this code.

\item
``Self-consistent order-$N$ density-functional 
calculations for very large systems"
P. Ordej\'on, E. Artacho and J. M. Soler,
Phys. Rev. B {\bf 53}, 10441, (1996).

Description of a previous version of this methodology.

\item
``Density functional method for very large systems with LCAO basis sets"
D. S\'anchez-Portal, P. Ordej\'on, E. Artacho and J. M. Soler,
Int. J. Quantum Chem., {\bf 65}, 453 (1997).

Description of the present method and code.

\item
``Linear-scaling ab-initio calculations for large and complex systems"
E. Artacho, D. S\'anchez-Portal, P. Ordej\'on, A. Garc\'{\i}a and
J. M. Soler, Phys. Stat. Sol. (b) {\bf 215}, 809 (1999).

Description of the numerical atomic orbitals (NAOs) most commonly 
used in the code, and brief review of applications as of March 1999.

\item
``Numerical atomic orbitals for linear-scaling calculations"
J. Junquera, O. Paz, D. S\'anchez-Portal, and E. Artacho, Phys. Rev. B
 {\bf 64}, 235111, (2001).

Improved, soft-confined NAOs.

\item
``The {\sc Siesta} method for ab initio order-$N$ materials simulation"
J. M. Soler, E. Artacho, J.D. Gale, A. Garc\'{\i}a, J. Junquera, P. Ordej\'on,
and D. S\'anchez-Portal, J. Phys.: Condens. Matter {\bf 14}, 2745-2779 (2002) 

Extensive description of the {\sc Siesta} method.

\item 
``Computing the properties of materials from first principles 
with  {\sc Siesta}", D. S\'anchez-Portal, P. Ordej\'on, 
and E. Canadell, Structure and Bonding {\bf 113},
103-170 (2004).

Extensive review of applications as of summer 2003.
\end{itemize}

For more information you can visit the web page 
{\tt http://www.uam.es/siesta}.

The following is a short description of the compilation procedures
and of the datafile format for the {\sc Siesta} code.

\section{VERSION UPDATE}
If you have a previous version of {\sc Siesta}, the update is simply
replacing the old {\tt siesta} directory tree with the new one, saving
the {\tt arch.make} file that you built to compile {\sc Siesta} for
your architecture (the format of this file has changed slightly, but
you should be able to translate the immportan fields, such as library
locations and compiler switches, to the new version).
You also have the option of using the new {\tt configure} script
(see below) to see whether the automatically generated {\tt arch.make} file
provides anything new or interesting for your setup.
  If you have working files within the old {\sc Siesta} tree, including
pseudopotential etc., you will have to fish them out. That is why we recommend
working directories outside the package.

\section{QUICK START}

\subsection{Compilation}
Unpack the {\sc Siesta} distribution. Go to the {\tt Src} directory, 
where the source code resides together with the Makefile. You will
need a file called {\tt arch.make} to suit your particular computer
setup. The command {\tt ./configure}  will start an automatic scan of 
your system and try to build
an {\tt arch.make} for you. Please note that the configure script
might need some help in order to find your Fortran compiler, and that
the created arch.make may not be optimal, mostly in regard to compiler
switches, but the process should provide a reasonable working
file. Type {\tt ./configure --help} to see the flags understood by the
script, and take a look at the {\tt Src/Confs} subdirectory for some
examples of their explicit use. You can also create your own {\tt arch.make}
by looking at the examples in the {\tt Src/Sys} subdirectory.
If you intend to create a parallel version of {\sc
Siesta}, make sure you have all the extra support libraries ({\tt MPI,
scalapack, blacs...}). Type {\tt make}. The executable should work
for any job (This is not exactly true, since some of the parameters in
the atomic routines are still hardwired (see {\tt Src/atmparams.f}),
but those would seldom need to be changed.)

\subsection{Running the program}

A fast way to test your installation of {\sc Siesta} and get a feeling
for the workings of the program is implemented in directory
{\tt Tests}\index{Tests}. In it you can find several subdirectories
with pre-packaged FDF files and pseudopotential references. Everything
is automated: after compiling {\sc Siesta} you can just go into any
subdirectory and type {\tt make}. The program does its work in
subdirectory {\tt work}, and there you can find all the resulting
files. For convenience, the output file is copied to the parent
directory. A collection of reference output files can be found in {\tt
  Tests/Reference}. Please note that small numerical and
formatting differences are to be expected, depending on the compiler.

Other examples are provided in the {\tt Examples} directory. This
directory contains basically the {\tt .fdf} files and the
pseudopotential generation input files. Since at some point you will
have to generate your own pseudopotentials and run your own jobs, we
describe here the whole process by means of the simple example of the
water-molecule. It is advisable to create independent directories for
each job, so that everything is clean and neat, and out of the {\tt
siesta} directory, so that one can easily update version by replacing
the whole {\tt siesta} tree. Go to your favorite working directory
and:

{\tt \$ mkdir h2o}

{\tt \$ cd h2o}

{\tt \$ cp $\sim$/siesta/Examples/H20/h2o.fdf .}

\noindent
We need to generate the required pseudopotentials
\index{pseudopotential!example generation}
(We are going to streamline this process for this time, but
you must realize that this is a tricky business that you 
must master before using {\sc Siesta} responsibly. Every 
pseudopotential must be thoroughly checked before use. Please refer to
the {\sc ATOM} program manual in {\tt $\sim$/siesta/Pseudo/atom/Docs}
for details regarding what follows.)

{\tt \$ cd $\sim$/siesta/Pseudo/atom}

{\tt \$ make}

\noindent
Now the pseudopotential-generation program, called {\tt atm}, 
should be compiled (you might want to change the definition of
the compiler in the makefile).

{\tt \$ cd Tutorial/O}

{\tt \$ cat O.tm2.inp}

\noindent
This is the input file, for the oxygen pseudopotential, 
that we have prepared for you. 
It is in a standard (but obscure) format that
you will need to understand in the future:
\begin{verbatim}
------------------------------------------------------------
   pg      Oxygen
        tm2  2.0
 n=O  c=ca 
       0.0       0.0       0.0       0.0       0.0       0.0
    1    4
    2    0     2.00      0.00
    2    1     4.00      0.00
    3    2     0.00      0.00
    4    3     0.00      0.00
   1.15     1.15     1.15     1.15
------------------------------------------------------------
\end{verbatim}

To generate the pseudopotential do the following;

{\tt \$ sh ../pg.sh O.tm2.inp}

\noindent
Now there should be a new subdirectory called O.tm2 (O for oxygen)
and {\tt O.tm2.vps} (unformatted) and {\tt O.tm2.psf} (ASCII) files.

{\tt \$ cp O.tm2.psf $\sim$/whateveryourworkingdir/h2o/O.psf}

\noindent
copies the generated pseudopotential file to your working directory. 
(The unformatted and ASCII files are functionally equivalent, but
the latter is more transportable and easier to look at, if you so
desire.) The same could be repeated for the pseudopotential for H,
but you may as well copy {\tt H.psf} from {\tt siesta/Examples/Vps/}
to your {\tt h2o} working directory.

\noindent
Now you are ready to run the program:

{\tt siesta < h2o.fdf | tee h2o.out}

\noindent
(If you are running the parallel version you should use some other 
invocation, such as {\tt mpirun -np 2 siesta ...}, but we cannot 
go into that here.)

After a successful run of the program, you should have many
files in your directory including the following:
\begin{itemize}

\item out.fdf\index{out.fdf}
 (contains all the data used, explicit or default-ed) 
\item O.ion and H.ion\index{species.ionl@{\it species.}ion}
 (complete information about the basis and KB projectors)
\item h2o.XV\index{Systemlabel.XV@{\it Systemlabel.}XV}
 (contains the final positions and velocities)
\item h2o.STRUCT\_OUT
\index{Systemlabel.STRUCT\_OUT@{{\it Systemlabel}.STRUCT\_OUT}}
 (contains the final cell vectors and positions in
 ``crystallographic'' format)
\item h2o.DM\index{Systemlabel.DM@{\it Systemlabel.}DM}
 (contains the density matrix to allow a restart)
\item h2o.ANI\index{Systemlabel.ANI@{\it Systemlabel.}.ANI}
 (contains the coordinates of every MD step, in this case only one)
\item h2o.FA\index{Systemlabel.FA@{\it Systemlabel.}.FA}
 (contains the forces on the atoms)
\item h2o.EIG\index{Systemlabel.EIG@{\it Systemlabel.}.EIG}
 (contains the eigenvalues of the Kohn-Sham Hamiltonian)
\item h2o.out\index{Systemlabel.out@{\it Systemlabel.}out}
 (standard output)
\item h2o.xml\index{Systemlabel.xml@{\it Systemlabel.}xml}
 (XML marked-up output)
\end{itemize}

The Systemlabel.out is the standard output of the program, that you
have already seen passing on the screen. Have a look at it
and refer to the output-explanation section if necessary.
You may also want to look at the out.fdf file to see all
the default values that siesta has chosen for you, before
studying the input-explanation section and start changing them.

Now look at the other data files in {\tt Examples}
(all with an .fdf suffix) choose one and repeat the process for it.


\section{PSEUDOPOTENTIAL HANDLING}

\index{pseudopotential!generation}

The atomic pseudopotentials are stored either in binary files (with
extension {\tt .vps}) or in ASCII files (with extension {\tt .psf}),
and are read at the beginning of the execution, for each species
defined in the input file. The data files must be named {\tt *.vps}
(or {\tt *.psf}), where {\tt *} is the label of the chemical species
(see the {\bf ChemicalSpeciesLabel}\index{ChemicalSpeciesLabel@{\bf
ChemicalSpeciesLabel}} descriptor below).

These files are generated by the ATOM program (read {\tt
siesta/Pseudo/atom/README} for more complete authorship and copyright
acknowledgements).  It is included (with permission) in
siesta/Pseudo/atom. Remember that {\bf all pseudopotentials should be
thoroughly tested} before using them. We refer you to the standard
literature on pseudopotentials and to the {\sc ATOM} manual {\tt
siesta/Pseudo/atom/atom.tex}.


\section{ATOMIC-ORBITAL BASES IMPLEMENTED IN SIESTA}

The main advantage of atomic orbitals is their efficiency (fewer orbitals 
needed per electron for similar precision) 
and their main disadvantage is the lack of systematics for optimal 
convergence, an issue that quantum chemists have been working on for
many years. They have also clearly shown that there
is no limitation on precision intrinsic to LCAO.
This section provides some information about how basis sets can be
generated for {\sc Siesta}.

It is important to stress at this point that neither the {\sc Siesta} 
method nor the program
are bound to the use of any particular kind of atomic orbitals. The
user can feed into {\sc Siesta} the atomic basis set he/she choses by
means of radial tables (see {\bf User.Basis} below), the
only limitations being: $(i)$ the functions have to be atomic-like (radial 
functions mutiplied by spherical harmonics), and $(ii)$ they have to be
of finite support, i.e., each orbital becomes strictly zero beyond some
cutoff radius chosen by the user.

Most users, however, do not have their own basis sets. For these users
we have devised some schemes to generate reasonable basis sets within
the program. These bases depend on several parameters per atomic
species that are for the user to choose, and can be important for both
quality and efficiency. A description of these bases and some
performance tests can be found in

\noindent
``Numerical atomic orbitals for linear-scaling calculations"
J. Junquera, O. Paz, D. S\'anchez-Portal, and E. Artacho, Phys. Rev. B
{\bf 64} 235111, (2001)

An important point here is that the basis set selection is a 
variational problem and, therefore, minimizing the energy with respect
to any parameters defining the basis is an ``ab initio" way to 
define them.

We have also devised a quite simple and systematic way of generating 
basis sets based on specifying only one main parameter (the energy shift)
besides the basis size. It does not offer the best NAO results one can get 
for a given basis size but it has the important advantages mentioned above. 
More about it in:

\noindent
``Linear-scaling ab-initio calculations for large and complex systems"
E. Artacho, D. S\'anchez-Portal, P. Ordej\'on, A. Garc\'{\i}a and
J. M. Soler, Phys. Stat. Sol. (b) {\bf 215}, 809 (1999).

In addition to {\sc Siesta} we provide the program {\sc
Gen-basis}\index{Gen-basis@{\sc Gen-basis}}, which reads {\sc
Siesta}'s input and generates basis files for later use. {\sc
Gen-basis} is compiled automatically at the same time as {\sc
Siesta}. It should be run from the {\tt Tutorials/Bases} directory, 
using the {\tt gen-basis.sh} script. It is limited to a single species.

In the following we give some clues on the basics of the basis sets
that {\sc Siesta} generates. 
  The starting point is always the solution of Kohn-Sham's Hamiltonian
for the isolated pseudo-atoms, solved in a radial grid,
with the same approximations as for the solid or molecule 
(the same exchange-correlation functional and  pseudopotential),
plus some way of confinement (see below).
  We describe in the following three main features of a
basis set of atomic orbitals: size, range, and radial shape.

\subsection{Size: number of orbitals per atom}

  Following the nomenclature of Quantum Chemistry, we establish
a hierarchy of basis sets, from single-$\zeta$ to multiple-$\zeta$ 
with polarization and diffuse orbitals, covering from quick calculations
of low quality to high precision, as high as the finest obtained in
Quantum Chemistry. 
  A single-$\zeta$ (also called minimal) basis set (SZ in the following)
has one single radial function per angular momentum channel, and only for 
those angular momenta with substantial electronic population in the valence of
the free atom.
  It offers quick calculations and some insight on qualitative trends 
in the chemical bonding and other properties. 
  It remains too rigid, however, for more quantitative calculations
requiring both radial and angular flexibilization.

  Starting by the radial flexibilization of SZ, a better basis is obtained 
by adding a second function per channel: double-$\zeta$ (DZ).
  In Quantum Chemistry, the {\it split valence} scheme
is widely used: starting from the expansion in Gaussians of one atomic 
orbital, the most contracted gaussians are used to define the first
orbital of the double-$\zeta$ and the most extended ones for the second.
  For strictly localized functions there was a first proposal
of using the excited states of the confined atoms, but it would work only 
for tight confinement (see {\bf PAO.BasisType} {\tt nodes} below).
  This construction was proposed and tested in D. S\'anchez-Portal 
{\it et al.}, J. Phys.: Condens. Matter {\bf 8}, 3859-3880 (1996).
 
  We found that the basis set convergence is slow, requiring high levels
of multiple-$\zeta$ to achieve what other schemes do at the double-$\zeta$
level. 
  This scheme is related with the basis sets used in the OpenMX project
[see T. Ozaki, Phys. Rev. B {\bf 67}, 155108 (2003); T. Ozaki and H. Kino, 
Phys. Rev. B {\bf 69}, 195113 (2004)]. 

  We then proposed an extension of the split valence idea of Quantum Chemistry 
to strictly localized NAO which has become the standard and has been used
quite succesfully in many systems (see {\bf PAO.BasisType} {\tt split} below).
  It is based on the idea of suplementing the first $\zeta$ with, instead of
a gaussian, a numerical orbital that reproduces the tail of the original PAO 
outside a matching radius $r_{m}$, and continues smoothly towards the origin as 
$r^l(a-br^2)$, with $a$ and $b$ ensuring continuity and differenciability 
at $r_{m}$. 
  Within exactly the same
Hilbert space, the second orbital can be chosen to be the difference between
the smooth one and the original PAO, which gives a basis orbital strictly
confined within the matching radius $r_{m}$ (smaller than the
original PAO!) continuously differenciable throughout. 

  Extra parameters have thus appeared: one $r_m$ per orbital to be doubled. 
The user can again introduce them by hand (see {\bf PAO.Basis} below).
Alternatively, all the $r_m$'s can be defined at once by specifying
the value of the tail of the original PAO beyond $r_m$, the so-called
split norm. Variational optimization
of this split norm performed on different systems
shows a very general and stable performance for values around
15\% (except for the $\sim 50\%$ for hydrogen). 
  It generalizes to multiple-$\zeta$ trivially by adding an additional
matching radius per new zeta.

  Angular flexibility is obtained by adding shells of higher angular 
momentum.
  Ways to generate these so-called polarization orbitals have been
described in the literature for Gaussians. 
  For NAOs there are two ways for {\sc Siesta} and {\sc Genbasis} to
generate them: $(i)$ Use atomic PAO's of higher angular momentum with
suitable confinement, and $(ii)$ solve the pseudoatom in the presence
of an electric field and obtain the $l+1$ orbitals from the perturbation
of the $l$ orbitals by the field.

  Finally, the method allows the inclusion of offsite orbitals (not centered
around any specific atom). The orbitals again can be of any shape, including
atomic orbitals as if an atom would be there (useful for calculating the
counterpoise correction for basis-set superposition errors). 
Bessel functions for any radius and any excitation level can also be 
added anywhere to the basis set.

\subsection{Range: cutoff radii of orbitals}

  Strictly localized orbitals (zero beyond a cutoff radius) are used
in order to obtain sparse Hamiltonian and overlap matrices for linear 
scaling. One cutoff radius per angular momentum channel has to be
given for each species. 
  A balanced and systematic starting point for defining all the different 
radii is achieved by giving one single parameter, the energy shift, i.e., 
the energy raise suffered by the orbital when confined.
Allowing for system and physical-quantity variablity, as a rule of thumb
$\Delta E_{\small \rm PAO} \approx 100$ meV gives 
typical precisions within the accuracy of current GGA functionals.
The user can, nevertheless, change the cutoff radii at will.

\subsection{Shape}

  Within the pseudopotential framework it is important to keep 
the consistency between the pseudopotential and
the form of the pseudoatomic orbitals in the core region.
  The shape of the orbitals at larger radii depends on the
cutoff radius (see above) and on the way the localization 
is enforced.

  The first proposal (and quite a standard among {\sc Siesta}
users) uses an infinite square-well potential.
  It was oroginally proposed and has been widely and succesfully used 
by Otto Sankey and collaborators, for minimal bases within 
the ab initio tight-binding scheme, using the {\sc Fireball } program, 
but also for more flexible bases using the methodology of {\sc Siesta}.
  This scheme has the disadavantage, however, of generating 
orbitals with a discontinuous derivative at $r_c$. 
  This discontinuity is more pronounced for smaller $r_c$'s and
tends to disappear for long enough values of this cutoff.
  It does remain, however, appreciable for sensible values of
$r_c$ for those orbitals that would be very wide in the free atom.
  It is surprising how small an effect such kink produces in the
total energy of condensed systems.
  It is, on the other hand, a problem for forces and stresses,
especially if they are calculated using a (coarse) finite 
three-dimensional grid.

  Another problem of this scheme is related to its defining the
basis considering the free atoms. 
  Free atoms can present extremely
extended orbitals, their extension being, besides problematic,
of no practical use for the calculation in condensed systems:
the electrons far away from the atom can be described by the
basis functions of other atoms.
 
  A traditional scheme to deal with this is the one based on the radial 
scaling of the orbitals by suitable scale factors.
  In addition to very basic bonding arguments, it is
soundly based on restoring virial's theorem for finite bases, in the case
of coulombic potentials (all-electron calculations).
  The use of pseudopotentials limits its applicability, allowing only for
extremely small deviations from unity ($\sim 1\%$) in the scale factors 
obtained variationally (with the exception of hydrogen that can contract 
up to 25\%). This possiblity is available to the user.

  Another way of dealing with that problem and that of the kink at the
same time is adding a soft confinement potential to the atomic Hamiltonian 
used to generate the basis orbitals: it smoothens
the kink and contracts the orbital as suited. Two additional parameters
are introduced for the purpose, which can be defined again variationally.
  The confining potential  is flat (zero) in the core region, starts off at
some internal radius $r_i$ with all derivatives continuous
and diverges at $r_c$ ensuring the strict localization there.
  It is
\begin{equation}
  V(r) = V_{\rm o} { e^{- { {r_c - r_i} \over {r - r_i} } } \over {r_c -r} } 
\end{equation}
and both $r_i$ and $V_{\rm o}$ can be given to {\sc Siesta} together
with $r_c$ in the input (see {\bf PAO.Basis} below).

  Finally, the shape of an orbital is also changed by the ionic character 
of the atom. 
  Orbitals in cations tend to shrink, and they swell in anions.
  Introducing a $\delta Q$ in the basis-generating free-atom calculations
gives orbitals better adapted to ionic situations in the condensed
systems.

  More information about basis sets can be found in the proposed
  literature.

The directory {\tt Tutorials/Bases} in the main {\sc Siesta
  distribution} contains some tutorial material for the generation of
basis sets and KB projectors.

\section{COMPILING THE PROGRAM}

The compilation of the program is done using a {\tt Makefile}
that is provided with the code.\index{Makefile}
This {\tt Makefile} will generate the executable for any of several
architectures, with a  minimum of tuning required from the
user in a separate file called {\tt arch.make} to reside in
the {\tt Src/} directory. 
The instructions are in directory {\tt siesta/Src/Sys}, where 
there are also a number of {\tt .make} files
already prepared for several architectures and
operating sistems.\index{platforms}\index{architectures}
If none of these fit your needs, you will have to prepare
one on your own. 
The command

\noindent 
{\tt \$ ./configure} 

\noindent
 will start an automatic scan of your system and try to build
an {\tt arch.make} for you. Please note that the configure script
might need some help in order to find your Fortran compiler, and that
the created arch.make may not be optimal, mostly in regard to compiler
switches, but the process should provide a reasonable working
file. Type {\tt ./configure --help} to see the flags understood by the
script, and take a look at the {\tt Src/Confs} subdirectory for some
examples of their explicit use. You can fine tune {\tt arch.make}
by looking at the examples in the {\tt Src/Sys} subdirectory.
If you intend to create a parallel version of {\sc
Siesta}, make sure you have all the extra support libraries ({\tt MPI,
scalapack, blacs...}). 

After {\tt arch.make} is ready, ype {\tt make}. The executable should work
for any job (This is not exactly true, since some of the parameters in
the atomic routines are still hardwired (see {\tt Src/atmparams.f}) ,
but those would seldom need to be changed.)

\section{INPUT DATA FILE}

\subsection{The Flexible Data Format (FDF)}\index{FDF}

The main input file,\index{input file}
which is read as the standard input (unit 5),
contains all the physical data of the system and the parameters of 
the simulation to be performed.
This file is written in a special format called FDF, developed by 
Alberto Garc\'{\i}a and Jos\'e M. Soler. This format allows data to be 
given in any order, or to be omitted in favor of default values.
Refer to documentation in $\sim$/siesta/Src/fdf for details.
Here we offer a glimpse of it through the following rules:

\begin{itemize}

\item[$\bullet$] The FDF syntax is a 'data label' followed by its value.
Values that are not specified in the datafile are assigned
a default value. 

\item[$\bullet$] FDF labels are case insensitive, and characters - \_ . 
in a data label are ignored. Thus, LatticeConstant and
lattice\_constant represent the same label.

\item[$\bullet$] All text following the \# character is taken as comment.

\item[$\bullet$] Logical values can be specified as T, true, .true.,
yes, F, false, .false., no. Blank is also equivalent to true.

\item[$\bullet$] Character strings should {\bf not} be in apostrophes.

\item[$\bullet$] Real values which represent a physical magnitude must be
followed by its units. Look at function fdf\_convfac in
file $\sim$/siesta/Src/fdf/fdf.f for the units that are currently supported.
It is important to include a decimal point in a real number to distinguish
it from an integer, in order to prevent ambiguities when mixing the types
on the same input line.

\item[$\bullet$] Complex data structures are called blocks and are
placed between `\%block label'\index{block@\%block} and a `\%endblock label' 
(without the quotes).

\item[$\bullet$] You may `include' other FDF files and redirect the search
for a particular data label to another file. 
If a data label appears more than once, its first appearance
is used.

\end{itemize}

\noindent
These are some examples:

\begin{verbatim}
           SystemName      Water molecule  # This is a comment 
           SystemLabel     h2o
           SpinPolarized        yes 
           SaveRho                 
           NumberOfAtoms         64  
           LatticeConstant       5.42 Ang  
           %block LatticeVectors  
                    1.000  0.000  0.000 
                    0.000  1.000  0.000 
                    0.000  0.000  1.000
           %endblock LatticeVectors  
           KgridCutoff < BZ_sampling.fdf 

           # Reading the coordinates from a file 
           %block AtomicCoordinatesAndAtomicSpecies < coordinates.data 

           # Even reading more FDF information from somewhere else
           %include mydefaults.fdf  
\end{verbatim}

Note that there is a lot of information that can be 
passed to {\sc Siesta} in the input file via fdf tags
(see most of the manual). Almost all of the tags are
optional: {\sc Siesta} will assign a default if a given tag
is not found when needed (see {\tt out.fdf}).
The only tags that are mandatory in any input file are
{\bf NumberOfSpecies}, \index{NumberOfSpecies@{\bf NumberOfSpecies}}
{\bf NumberOfAtoms}, \index{NumberOfAtoms@{\bf NumberOfAtoms}}
and {\bf ChemicalSpeciesLabel} 
\index{ChemicalSpeciesLabel@{\bf ChemicalSpeciesLabel}}
in addition to introducing the atomic positions, either through
{\bf AtomicCoordinatesAndAtomicSpecies}, 
\index{AtomicCoordinatesAndAtomicSpecies@{\bf 
AtomicCoordinatesAndAtomicSpecies}}
or via {\bf Zmatrix}. \index{Zmatrix@{\bf Zmatrix}}

Here follows a description of the variables that you can
define in your {\sc Siesta} input file, with their data types and
default values.

\vspace{5pt}
\subsection{General system descriptors}

\begin{description}
\itemsep 10pt
\parsep 0pt

\item[{\bf SystemName}] ({\it string}): 
\index{Systemname@{\bf SystemName}}
A string of one or several words containing a descriptive
name of the system (max. 150 characters). 

{\it Default value:} blank

\item[{\bf SystemLabel}] ({\it string}): 
\index{SystemLabel@{\bf SystemLabel}}
A {\bf single} word (max. 20 characters {\bf without blanks})
containing a nickname of the system, used to name output files.

{\it Default value:} {\tt siesta}\index{siesta}

\item[{\bf NumberOfSpecies}] ({\it integer}): 
\index{NumberOfSpecies@{\bf NumberOfSpecies}}
Number of different atomic species in the simulation. 
Atoms of the same species, but with a different 
pseudopotential or basis set are counted as different species.

{\it Default value:} There is no default. You must supply this variable.

\item[{\bf NumberOfAtoms}] ({\it integer}): 
\index{NumberOfAtoms@{\bf NumberOfAtoms}}
Number of atoms in the simulation.

{\it Default value:} There is no default. You must supply this variable.

\item[{\bf ChemicalSpeciesLabel}] ({\it data block}): 
\index{ChemicalSpeciesLabel@{\bf ChemicalSpeciesLabel}} 
It specifies the different chemical species\index{species} that are present, 
assigning them a number for further identification. 
{\sc Siesta} recognizes the different atoms by the given atomic number.

\begin{verbatim}
         %block Chemical_Species_label
            1   6   C 
            2  14   Si
            3  14   Si_surface
         %endblock Chemical_Species_label
\end{verbatim}

The first number in a line is the species number, it is followed by the
atomic number, and then by the desired label. This label will be used
to identify corresponding files, namely, pseudopotential file, user basis
file, basis output file, and local pseudopotential output file.

This construction allows you to have atoms of the same species but with
different basis or pseudopotential, for example.

Negative atomic numbers are used for {\it ghost} atoms\index{ghost atoms} 
(see {\bf PAO.basis}).

Atomic numbers over 200 are used to represent {\it synthetic atoms}
\index{synthetic atoms} (created for example as a ``mixture'' of two
real ones for a ``virtual crystal'' (VCA)\index{VCA} calculation). In
this special case a new 'SyntheticAtoms' block
\index{SyntheticAtoms@{\bf SyntheticAtoms}}  must be present to give
{\sc Siesta} information about the ``ground state'' of the synthetic
atom.

\begin{verbatim}
         %block Chemical_Species_label
            1   201 ON-0.50000
         %endblock Chemical_Species_label
         %block SyntheticAtoms
         1               # Species index
         2 2 3 4         # n numbers for valence states  with l=0,1,2,3
         2.0 3.5 0.0 0.0 # occupations of valence states with l=0,1,2,3
         %endblock SyntheticAtoms
\end{verbatim}

Pseudopotentials for synthetic atoms can be created using the {\tt
  mixps} program\index{mixps program} in the main source directory
  ({\tt make mixps}). See also the {\tt Util/VCA} directory.

{\it Use:} This block is mandatory.

{\it Default:} There is no default. You must supply this block.

\item[{\bf PhononLabels}] ({\it data block}): 
\index{PhononLabels@{\bf PhononLabels}} 
It provides the mapping\index{Phonon program@{\sc Phonon program}} 
between the species number and those used by the {\sc
Phonon} program. Note that chemically identical elements might be
assigned different labels if they are not related by symmetry.

\begin{verbatim}
         %block PhononLabels
            1   A   Mg
            2   B   O
         %endblock PhononLabels
\end{verbatim}

The species number is followed by the {\sc
Phonon} program label and by the chemical symbol.

{\it Use:} This block is mandatory if {\tt MD.TypeOfRun} is {\tt Phonon}.

{\it Default:} No default. 



\item[{\bf AtomicMass}] ({\it data block}): 
\index{AtomicMass@{\bf AtomicMass}} 
It allows the user to introduce
the atomic masses of the different species used in the calculation, useful
for the dynamics with isotopes,\index{isotopes} for example. If
a species index is not found within the block, the natural mass for the 
corresponding atomic number is assumed. If the block is absent all masses
are the natural ones. One line per species with the species index (integer)
and the desired mass (real). The order is not important. If there is no 
integer and/or no real numbers within the line, the line is disregarded.

\begin{verbatim}
         %block AtomicMass
            3  21.5
            1  3.2 
         %endblock AtomicMass
\end{verbatim}

{\it Default:} (Block absent or empty) Natural masses assumed. For 
{\it ghost} atoms (i.e. floating orbitals), a default of 1.d30 a.u. is 
assigned.


\item[{\bf NetCharge}] ({\it real}): 
\index{NetCharge@{\bf NetCharge}}\index{Charge of the system}
Specify the net charge of the system (in units of $|e|$). 
For charged systems, the energy converges very slowly
versus cell size. For molecules or atoms, a Madelung
correction term is applied to the energy to make it converge
much faster with cell size (this is done only if
the cell is SC, FCC or BCC). For other cells, or for 
periodic systems (chains, slabs or bulk), this energy
correction term can not be applied, and the user is warned
by the program.   It is not advised to do charged systems
other than atoms and molecules in SC, FCC or BCC cells,
unless you know what you are doing.

{\it Use:} 
For example, the F$^-$ ion would have {\bf NetCharge} = {\tt -1},
and the Na$^+$ ion would have {\bf NetCharge} = {\tt 1}.
Fractional charges can also be used.

{\it Default value:} {\tt 0.0} 


\end{description}


\vspace{5pt}
\subsection{Basis definition}

\noindent 
The format for the input of the basis described in this manual
has experienced important changes respect to 
previous versions of the program. Although old fashioned 
input files are
still readable by {\sc Siesta}, we highly recommend the use of this
new format, which allows a much more flexible input.

\begin{description}
\itemsep 10pt
\parsep 0pt
\item[{\bf User.Basis}] ({\it logical}): 
\index{User.Basis@{\bf User.Basis}}\index{basis!User basis} 

If true, the basis, KB projector, and other information is read from
files {\it Atomlabel}{\tt .ion}, where {\it Atomlabel} is the atomic
species label specified in block {\it ChemicalSpeciesLabel}. These
files can be generated by a previous {\sc Siesta} run or (one by one) by the
standalone program {\sc Gen-basis}.\index{Gen-basis program@{\sc
Gen-basis}}\index{basis!Gen-basis standalone program} No pseudopotential
files are necessary.

\item[{\bf User.Basis.NetCDF}] ({\it logical}): 
\index{User.Basis.NetCDF@{\bf User.Basis.NetCDF}}
\index{basis!User basis (NetCDF format)} 
\index{NetCDF format} 

If true, the basis, KB projector, and other information is read from
NetCDF files {\it Atomlabel}{\tt .ion.nc}, where {\it Atomlabel} is
the atomic label specified in block {\it ChemicalSpeciesLabel}. These
files can be generated by a previous {\sc Siesta} run or by the
standalone program {\sc Gen-basis}.\index{Gen-basis program@{\sc
Gen-basis}}\index{basis!Gen-basis standalone program} No pseudopotential
files are necessary.


\item[{\bf PAO.BasisType}] ({\it string}): 
\index{PAO.BasisType@{\bf PAO.BasisType}}
\index{basis!PAO} 

The kind of basis to be generated is chosen. All are based on
finite-range pseudo-atomic orbitals\index{finite-range pseudo-atomic
orbitals} [PAO's of Sankey and Niklewsky, PRB 40, 3979 (1989)] The
original PAO's were described only for minimal bases. {\sc Siesta}
generates extended bases (multiple-$\zeta$,\index{multiple-$\zeta$}
polarization,\index{polarization orbitals} and diffuse
orbitals\index{diffuse orbitals} applying different schemes of choice:

\begin{itemize}

\item[-] Generalization of the PAO's: uses the excited orbitals of the
finite-range pseudo-atomic problem, both for multiple-$\zeta$ and for
polarization [see S\'anchez-Portal, Artacho, and Soler, JPCM {\bf 8},
3859 (1996)]. Adequate for short-range orbitals.

\item[-] Multiple-$\zeta$ in the spirit of split valence,\index{split
valence} decomposing the original PAO in several pieces of different
range, either defining more (and smaller) confining radii, or
introducing gaussians\index{gaussians} from known bases (Huzinaga's
book).
\end{itemize}

\noindent
All the remaining options
give the same minimal basis\index{minimal basis}. 
The different options and their FDF descriptors are the following:

\begin{itemize}

\item {\tt split:} Split-valence scheme for multiple-zeta.
The split is based on different radii. 

\item {\tt splitgauss:}\index{splitgauss@{\tt splitgauss}}
Same as {\tt split} but using gaussian functions
$e^{-(x/\alpha_i)^2}$. The gaussian widths $\alpha_i$ are read instead 
of the scale factors (see below). There is no cutting algorithm, so that
a large enough $r_c$ should be defined for the gaussian to have decayed
sufficiently.

\item {\tt nodes:}\index{nodes@{\tt nodes}} Generalized PAO's.

\item {\tt nonodes:}\index{nonodes@{\tt nonodes}} 
The original PAO's are used, multiple-zeta is generated
by changing the scale-factors, instead of using the excited orbitals. 

\end{itemize}

\noindent
Note that, for the {\tt split} and {\tt nodes} cases
the whole basis can be generated by {\sc Siesta} with no further information
required. {\sc Siesta} will use default values as defined in the following 
({\bf PAO.BasisSize},
{\bf PAO.EnergyShift}, and {\bf PAO.SplitNorm}, see below).

{\it Default value:} {\tt split}


\item[{\bf PAO.BasisSize}] ({\it string}): 
\index{PAO.BasisSize@{\bf PAO.BasisSize}}\index{basis!PAO} 
It defines usual basis sizes. It has effect only if there is no 
block {\bf PAO.Basis} present.

\begin{itemize}

\item {\tt SZ}\index{SZ@{\tt SZ}} or {\tt MINIMAL}:\index{MINIMAL@{\tt
MINIMAL}} minimal or single-$\zeta$
basis.\index{basis!minimal}\index{single-$\zeta$}

\item {\tt DZ}:\index{DZ@{\tt DZ}} Double zeta basis, in the scheme
defined by {\bf PAO.BasisType}.

\item {\tt SZP}:\index{SZP@{\tt SZP}} Single-zeta basis plus polarization 
orbitals.

\item {\tt DZP}\index{DZP@{\tt DZP}} or {\tt
STANDARD}:\index{STANDARD@{\tt STANDARD}} Like {\tt DZ} plus
polarization orbitals.  Polarization orbitals are constructed from
perturbation theory,\index{perturbative polarization} and they are
defined so they have\index{basis!polarization} the minimum angular
momentum $l$ such that there are not occupied orbitals with the same
$l$ in the valence shell of the ground-state atomic
configuration. They polarize the corresponding $l-1$ shell.

{\bf Note}: The ground-state atomic configuration used internally
by {\sc Siesta} is defined in the source file {\tt Src/periodic\_table.f}. 
For some elements (e.g., Pd), the configuration might not be the
standard one.\index{Ground-state atomic configuration}.

\end{itemize}

{\it Default value:} {\tt STANDARD}

\item[{\bf PAO.BasisSizes}]({\it data block}): 
\index{PAO.BasisSizes@{\bf PAO.BasisSizes}}\index{basis!PAO}
Block which allows to specify a different value of the variable 
PAO.BasisSize for each species. 
\begin{verbatim}
          %block    PAO.BasisSizes
               Si      DZ
               H       DZP
               O       SZP 
          %endblock PAO.BasisSizes
\end{verbatim} 


\item[{\bf PAO.EnergyShift}] ({\it real energy}): A standard for 
orbital-confining cutoff radii. It is the excitation energy
of the PAO's due to the confinement to a finite-range. It offers a
general procedure for defining the confining radii of the original
(first-zeta) PAO's for all the species guaranteeing the compensation 
of the basis. It has only effect when the block
{\bf PAO.Basis} is not present or when the radii
specified in that block are zero for the first zeta.

{\it Use:} It has to be positive.

{\it Default value:} {\tt 0.02 Ry}


\item[{\bf PAO.SplitNorm}] ({\it real}): 
\index{PAO.SplitNorm@{\bf PAO.SplitNorm}}\index{basis!split valence} 
A standard to define default sensible
radii for the split-valence type of basis. It gives the amount of norm that
the second-$\zeta$ split-off piece has to carry. The split radius is defined
accordingly. If multiple-$\zeta$\index{multiple-$\zeta$} 
is used, the corresponding radii are obtained
by imposing smaller fractions of the SplitNorm (1/2, 1/4 ...) value as
norm carried by the higher zetas. It has only effect when the block
{\bf PAO.Basis} is not present or when the radii
specified in that block are zero for zetas higher than one.

{\it Default value:} {\tt 0.15} (sensible values range between 0.05 and 0.5).

\item[{\bf PAO.SplitNormH}] ({\it real}):
\index{PAO.SplitNormH@{\bf PAO.SplitNormH}}\index{basis!split valence
  for H}
This option is as per {\bf PAO.SplitNorm} but allows a separate
default to be
specified for hydrogen which typically needs larger values than those
for other
elements.


\item[{\bf PAO.SoftDefault}] ({\it boolean}):
\index{PAO.SoftDefault@{\bf PAO.SoftDefault}}\index{basis!default soft
  confinement}
If set to true then this option causes soft confinement to be the
default form
of potential during orbital generation. The default potential and
inner radius
are set by the commands given below.

{\it Default value:} {\tt .false.}

\item[{\bf PAO.SoftInnerRadius}] ({\it real}):
\index{PAO.SoftInnerRadius@{\bf
    PAO.SoftInnerRadius}}\index{basis!default soft confinement radius}
For default soft confinement, the inner radius is set at a fraction of
the outer
confinement radius determined by the energy shift. This option
controls the fraction
of the confinement radius to be used.

{\it Default value:} {\tt 0.9}

\item[{\bf PAO.SoftPotential}] ({\it real}):
\index{PAO.SoftPotential@{\bf PAO.SoftPotential}}\index{basis!default
  soft confinement potential}
For default soft confinement, this option controls the value of the
potential used
for all orbitals.

{\it Default value:} {\tt 40.0 Ry}


\item[{\bf PS.lmax}]  ({\it data block}): 
\index{PS.lmax@{\bf PS.lmax}}
Block with the maximum angular momentum of the Kleinman-Bylander 
projectors,\index{Kleinman-Bylander projectors} {\tt lmxkb}. 
This information is optional. If the block 
is absent, or for a species which is not mentioned inside 
it, {\sc Siesta} will take {\tt lmxkb(is) = lmxo(is) + 1}, where {\tt lmxo(is)}
is the maximum angular momentum of the basis orbitals of species {\tt is}.
\begin{verbatim}
         %block Ps.lmax
              Al_adatom   3
              H           1
              O           2
         %endblock Ps.lmax
\end{verbatim}
{\it Default:} (Block absent or empty). Maximum angular momentum 
of the basis orbitals plus one.
\noindent

\item[{\bf PS.KBprojectors}] ({\it data block}):
\index{PS.KBprojectors@{\bf PS.KBprojectors} }
This block provides information about the number of Kleinman-Bylander
projectors per angular momentum, and for each species, that will used
in the calculation. This block is optional.
If the block is absent, or for species not mentioned in it, only 
one projector will be used for each angular momentum. The projectors
will be constructed using the eigenfunctions of the respective
pseudopotentials. 


This block allows to specify the number of projector for each l, and also
the reference energies of the wavefunctions used to build them.
The specification of the reference energies is optional. If these 
energies are not given, the program will use the eigenfunctions with an
increasing number of nodes (if there is not bound state with
the corresponding number of nodes, the ``eigenstates" are taken to be just
functions which are made zero at very long distance of the nucleus).
The units for the energy can be optionally specified, if not, the 
program will assumed that are given in Rydbergs. 
The data provided in this block must be consistent with those 
read from the block {\bf PS.lmax}. 

\begin{verbatim}
         %block PS.KBprojectors
             Si  3
              2   1 
             -0.9     eV
              0   2
             -0.5  -1.0d4 Hartree
              1   2
             Ga  1
              1  3
             -1.0  1.0d5 -6.0
         %endblock PS.KBprojectors
\end{verbatim}

The reading is done this way (those variables in brackets are optional,
therefore they are only read if
present):

\begin{verbatim}
    From is = 1 to  nspecies
         read: label(is), l_shells(is)
         From lsh=1 to l_shells(is)
              read: l, nkbl(l,is)
              read: {erefKB(izeta,il,is)}, from ikb = 1 to nkbl(l,is), {units}
\end{verbatim}

When a very high energy, higher that 1000 Ry, is specified, the
default is taken instead.  On the other hand, very low (negative)
energies, lower than -1000 Ry, are used to indicate that the energy
derivative of the last state must be used. For example, in the example
given above, two projectors will be used for the {\it s}
pseudopotential of Si. One generated using a reference energy of -0.5
Hartree, and the second one using the energy derivative of this
state. For the {\it p} pseudopotential of Ga, three projectors will be
used.  The second one will be constructed from an automatically
generated wavefunction with one node, and the other projectors from
states at -1.0 and -6.0 Rydberg.

The analysis looking for possible {\it ghost} states is only performed
when a single projector is used.  Using several projectors some
attention should be paid to the ``KB cosine" (kbcos), given in the
output of the program.  The KB cosine gives the value of the overlap
between the reference state and the projector generated from it.  If
these numbers are very small ( $<$ 0.01, for example) for {\bf all}
the projectors of some angular momentum, one can have problems related
with the presence of ghost states.

{\it Default:} (Block absent or empty). Only one KB projector,
constructed from the nodeless eigenfunction, used for each angular
momentum.
\noindent

     
\item[{\bf PAO.Basis}] ({\it data block}): \index{PAO.Basis@{\bf
PAO.Basis}\index{basis!PAO}} 
Block with data to define explicitly the
basis to be used.  It allows the definition by hand of all the
parameters that are used to construct the atomic basis. There is no
need to enter information for all the species present in the
calculation. The basis\index{basis!PAO} for the species not mentioned in
this block will be generated automatically using the parameters {\bf
PAO.BasisSize}, {\bf PAO.BasisType}, {\bf PAO.EnergyShift}, {\bf
PAO.SplitNorm} (or {\bf PAO.SplitNormH}), and the soft-confinement
defaults, if used (See {\bf PAO.SoftDefault}). 

Some parameters can be set to zero, or
left out completely.  In these cases the values will be generated from the
magnitudes defined above, or from the appropriate default values. For
example, the radii\index{cutoff radius} will be obtained from {\bf
PAO.EnergyShift} or from {\bf PAO.SplitNorm} if they are zero; the
scale factors will be put to 1 if they are zero or not given in the
input.  An example block for a two-species calculation (H and O) is
the following ({\tt opt} means optional):

\begin{verbatim}
%block PAO.Basis     # Define Basis set
O    2  nodes  1.0   # Label, l_shells, type (opt), ionic_charge (opt)
 n=2 0 2  E 50.0 2.5 # n (opt if not using semicore levels),l,Nzeta,Softconf(opt
     3.50  3.50      #     rc(izeta=1,Nzeta)(Bohr)
     0.95  1.00      #     scaleFactor(izeta=1,Nzeta) (opt)
     1 1  P 2        # l, Nzeta, PolOrb (opt), NzetaPol (opt)
     3.50            #     rc(izeta=1,Nzeta)(Bohr)
H    1               # Label, l_shells, type (opt), ionic_charge (opt)
     0 2 S 0.2       # l, Nzeta, Per-shell split norm parameter
     5.00  0.00      #     rc(izeta=1,Nzeta)(Bohr)
%endblock PAO.Basis
\end{verbatim}

\noindent
The reading is done this way (those variables in brackets are
optional, therefore they are only read if present) (See 
the routines in {\tt Src/basis\_specs.f} for detailed information):

\begin{verbatim}
    From js = 1 to  nspecies 
       read: label(is), l_shells(is), { type(is) }, { ionic_charge(is) }
       From lsh=1 to l_shells(is)
        read: 
         { n }, l(lsh), nzls(lsh,is), { PolOrb(l+1) }, { NzetaPol(l+1) },
         {SplitNormfFlag(lsh,is)}, {SplitNormValue(lsh,is)}
         {SoftConfFlag(lsh,is)}, {PrefactorSoft(lsh,is)}, {InnerRadSoft(lsh,is)}
           read: rcls(izeta,lsh,is), from izeta = 1 to nzls(l,is)
           read: { contrf(izeta,il,is) }, from izeta = 1 to nzls(l,is)
\end{verbatim}

\noindent
And here is the variable description:
\begin{itemize}
\item[-] {\tt Label}: Species label, this label determines 
the species index {\tt is} according to the block {\bf ChemicalSpecieslabel}
\item[-] {\tt l\_shells(is)}: Number of shells of orbitals 
with different angular momentum for species {\tt is}  
\item[-] {\tt type(is)}: {\it Optional input}.
Kind of basis set generation procedure for species {\tt is}. 
Same options as {\bf PAO.BasisType}
\item[-] {\tt ionic\_charge(is)}: {\it Optional input}. 
Net charge of species {\tt is}. This is  only used for
basis set generation purposes. {\it Default value}: {\tt 0.0} (neutral
atom). Note that if the pseudopotential was generated in an ionic
configuration, and no charge is specified in PAO.Basis, the ionic
charge setting will be that of pseudopotential generation.
\item[-] {\tt n}: Principal quantum number of the shell. This is an optional 
input for normal atoms, however it must be specified when there are
{\it semicore} states (i.e. when states that usually are not 
considered to belong to the 
valence shell have been included in the calculation)
\item[-] {\tt l}: Angular momentum of 
basis orbitals of this shell
\item[-] {\tt nzls(lsh,is)}: Number of 'zetas' for this shell.
\item[-] {\tt PolOrb(l+1)}: {\it Optional input}. If set equal to {\tt P}, a 
shell of  
polarization functions (with angular momentum $l+1$)  will be constructed 
from the first-zeta orbital of angular momentum $l$. {\it Default value}: ' ' 
(blank = No polarization orbitals). 
\item[-] {\tt NzetaPol(l+1)}: {\it Optional input}. Number of
'zetas' for the 
polarization shell (generated automatically in a split-valence fashion). 
Only active if {\tt PolOrb = P}. {\it Default value}: {\tt 1} 
\item[-] {\tt SplitNormFlag(lsh,is)}:\index{basis!soft confinement potential}
{\it Optional input}. If set equal to 
{\tt S}, the following number sets the split-norm parameter for that shell.
\item[-] {\tt SoftConfFlag(l,is)}:\index{basis!soft confinement potential}
{\it Optional input}. If set equal to 
{\tt E}, the new soft confinement potential proposed in formula (1) of
the paper by J. Junquera {\it et al.}, Phys. Rev. B {\bf 64}, 235111 (2001),
is used instead of the Sankey hard-well potential.
\item[-] {\tt PrefactorSoft(l,is)}: {\it Optional input}. Prefactor
of the soft confinement potential ($V_{0}$ in the formula). Units in Ry. 
{\it Default value}: 0 Ry.
\item[-] {\tt InnerRadSoft(l,is)}: {\it Optional input}. Inner radius where
the soft confinemet potential starts off ($r_{i}$ in the foormula). 
If negative, the inner radius will be computed as the given fraction
of the PAO cutoff radius.
Units in bohrs. {\it Default value}: 0 bohrs. 
\item[-] {\tt rcls(izeta,l,is)}: Cutoff radius (Bohr) of 
each 'zeta' for this shell.
\item[-] {\tt contrf(izeta,l,is)}: {\it Optional input}. 
Contraction factor\index{scale factor} of 
each 'zeta' for this shell.
{\it Default value}: {\tt 1.0}
\end{itemize} 

Polarization orbitals\index{perturbative
polarization}\index{basis!polarization} are generated by solving the
atomic problem in the presence of a polarizing electric field. The
orbitals are generated applying perturbation theory to the first-zeta
orbital of lower angular momentum.  They have the same cutoff radius
than the orbitals from which they are constructed.

There is a different possibility of generating polarization orbitals:
by introducing them explicitly in the {\bf PAO.Basis} block.
It has to be remembered, however, that they sometimes correspond to 
unbound states of the atom, their shape depending very much on the
cutoff radius, not converging by increasing it, similarly to the
multiple-zeta orbitals generated with the {\tt nodes} option. 
Using {\bf PAO.EnergyShift} makes no sense, and a cut off 
radius different from zero must be explicitly given (the same cutoff radius
as the orbitals they polarize is usually a sensible choice).

A species with atomic number = -100 will be considered by {\sc Siesta} as
a constant-pseudopotential atom, {\it i.e.}, the basis functions
generated will be spherical Bessel functions\index{Bessel functions}
with the specified $r_c$. In this case, $r_c$ has to be given, as
{\bf EnergyShift} will not calculate it.\index{basis!Bessel functions}

Other negative atomic numbers will be interpreted by {\sc Siesta} as 
{\it ghosts}\index{ghost atoms}\index{basis!ghost atoms} 
of the corresponding positive value: the orbitals
are generated and put in position as determined by the coordinates,
but neither pseudopotential nor electrons are considered for that
ghost atom. Useful for BSSE\index{basis!basis set superposition 
error (BSSE)} correction.

{\it Use:} This block is optional, except when Bessel functions or
semicore states are present. 

{\it Default:} Basis characteristics defined by global definitions given
above.

\end{description}


\vspace{5pt}
\subsection{Lattice, coordinates, $k$-sampling}

The position types, and behaviours of the atoms in the simulation
must be specified using the flags described in this section.

Firstly, the size of the cell itself should be specified, using
some combination of the flags
\textbf{LatticeConstant}, \textbf{LatticeParameters},
and \textbf{LatticeVectors}, and \textbf{SuperCell}.
If nothing is specified, {\sc Siesta} will construct a cubic
cell in which the atoms will reside as a cluster.

Secondly, the positions of the atoms within the cells
must be specified, using either the traditional {\sc Siesta}
input format (a modified xyz format) which must be described 
within
a \textbf{AtomicCoordinatesAndAtomicSpecies} block, or 
(from version 1.4) the newer Z-Matrix format, using a
\textbf{ZMatrix} block. The two formats cannot and
should not be mixed, but one of them must be used: 
this is one of the few pieces of information {\sc Siesta}
needs to find in the input (for which there is no default).

The advantage of the traditional format is that it is
much easier to set up a system. However, when working
on systems with constraints, there are only a limited
number of (very simple) constraints that may be expressed
within this format, and recompilation is needed for each
new constraint. 

For any more involved set of constraints, a 
full \textbf{ZMatrix} formulation should be used - this
offers much more control, and may be specified fully at
run time (thus not requiring recompilation) - but
it is more work to generate the input files for this form.

When using the traditional format, the following additional
flags may be used: \textbf{AtomicCoordinatesFormat} and
\textbf{AtomicCoordinatesOrigin}, and constraints should
be specified with a \textbf{GeometryConstraints} block.

When using the ZMatrix format, the following additional flags
may be used: \textbf{ZM.UnitsLength}, \textbf{ZM.UnitsAngle},
\textbf{ZM.ForceTolLength}, \textbf{ZM.ForceTolAngle}, 
\textbf{ZM.MaxDisplLength}, \textbf{ZM.MaxDisplAngle}

In addition, this section also shows the flags used to control
the calculation of $k$-points; \textbf{kgrid\_cutoff},
\textbf{kgrid\_Monkhorst\_Pack}, \textbf{BandLinesScale},
\textbf{BandLines}, \textbf{BandPoints},
\textbf{WaveFuncKPointsScale}, \textbf{WaveFuncKPoints}.


\begin{description}
\itemsep 10pt
\parsep 0pt


\item[{\bf LatticeConstant}] ({\it real length}): 
\index{LatticeConstant@{\bf LatticeConstant}} 
Lattice constant. This is just to define the scale of the lattice vectors.

{\it Default value:} Minimum size to include the system (assumed to be a 
molecule) without intercell interactions, plus 10\%. 

\item[{\bf LatticeParameters}] ({\it data block}): 
\index{LatticeParameters@{\bf LatticeParameters}}
Crystallographic way of specifying the lattice vectors, by giving
six real numbers: the three vector modules, $a$, $b$, and $c$, and
the three angles $\alpha$ (angle between $\vec b$ and $\vec c$),
$\beta$, and $\gamma$. The three modules are in units of 
{\bf LatticeConstant}, the three angles are in degrees.

{\it Default value:}
{\tt
\begin{verbatim}
           1.0   1.0   1.0    90.   90.  90.
\end{verbatim}
}
\noindent
(see the following)

\item[{\bf LatticeVectors}] ({\it data block}): 
\index{LatticeVectors@{\bf LatticeVectors}} 
The cell vectors are read in units of the lattice constant defined above. 
They are read as a matrix {\tt CELL(ixyz,ivector)}, each vector being
one line.

{\it Default value:} 
{\tt 
\begin{verbatim}
            1.0    0.0    0.0 
            0.0    1.0    0.0 
            0.0    0.0    1.0 
\end{verbatim}
}
\noindent
If the {\bf LatticeConstant} default is used, the default of 
{\bf LatticeVectors} is still diagonal but not necessarily cubic.

\item[{\bf SuperCell}] ({\it data block}): 
\index{SuperCell@{\bf SuperCell}} 
Integer 3x3 matrix defining a supercell in terms of the unit cell: 

\begin{verbatim}
     %block SuperCell
        M(1,1)  M(2,1)  M(3,1) 
        M(1,2)  M(2,2)  M(3,2) 
        M(1,3)  M(2,3)  M(3,3) 
     %endblock SuperCell
\end{verbatim}

and the supercell is defined as
$SuperCell(ix,i) = \sum_j CELL(ix,j)*M(j,i)$.
Notice that the matrix indexes are inverted: each input line 
specifies one supercell vector.

{\it Warning:} {\bf SuperCell} is disregarded if the geometry is read from
the XV file, which can happen unadvertedly.

{\it Use:} The atomic positions must be given only for the unit cell,
and they are 'cloned' automatically in the rest of the supercell.
The {\bf NumberOfAtoms} given must also be that in a single unit cell.
However, all values in the output are given for the entire supercell. 
In fact, CELL is inmediately redefined as the whole supercell and the 
program no longer knows the existence of an underlying unit cell.
All other input (apart from NumberOfAtoms and atomic positions), 
including {\bf kgridMonkhorstPack} must refer to the supercell 
(this is a change over the previous version). Therefore, to avoid
confusions, we recommend to use {\bf SuperCell} only to
generate atomic positions, and then to copy them from the output
to a new input file with all the atoms specified explicitly and
with the supercell given as a normal unit cell.

{\it Default value:} No supercell (supercell equal to unit cell).


\item[{\bf SuperCell}] ({\it data block}): 
\index{SuperCell@{\bf SuperCell}} 
Integer 3x3 matrix defining a supercell in terms of the unit cell: 

\begin{verbatim}
     %block SuperCell
        M(1,1)  M(2,1)  M(3,1) 
        M(1,2)  M(2,2)  M(3,2) 
        M(1,3)  M(2,3)  M(3,3) 
     %endblock SuperCell
\end{verbatim}

and the supercell is defined as
$SuperCell(ix,i) = \sum_j CELL(ix,j)*M(j,i)$.
Notice that the matrix indexes are inverted: each input line 
specifies one supercell vector.

{\it Warning:} {\bf SuperCell} is disregarded if the geometry is read from
the XV file, which can happen unadvertedly.

{\it Use:} The atomic positions must be given only for the unit cell,
and they are 'cloned' automatically in the rest of the supercell.
The {\bf NumberOfAtoms} given must also be that in a single unit cell.
However, all values in the output are given for the entire supercell. 
In fact, CELL is inmediately redefined as the whole supercell and the 
program no longer knows the existence of an underlying unit cell.
All other input (apart from NumberOfAtoms and atomic positions), 
including {\bf kgridMonkhorstPack} must refer to the supercell 
(this is a change over the previous version). Therefore, to avoid
confusions, we recommend to use {\bf SuperCell} only to
generate atomic positions, and then to copy them from the output
to a new input file with all the atoms specified explicitly and
with the supercell given as a normal unit cell.

{\it Default value:} No supercell (supercell equal to unit cell).


\item[{\bf AtomicCoordinatesFormat}] ({\it string}): 
\index{AtomicCoordinatesFormat@{\bf AtomicCoordinatesFormat}} 
Character string to specify the format of the atomic positions in
input. These can be expressed in four forms:
\begin{itemize}
\item {\tt Bohr} or {\tt NotScaledCartesianBohr} (atomic positions 
are given directly in Bohr, in cartesian coordinates)
\item {\tt Ang} or {\tt NotScaledCartesianAng} (atomic positions 
are given directly in {\AA}ngstr\"om, in cartesian coordinates)
\item {\tt ScaledCartesian} (atomic positions are given 
in cartesian coordinates, in units of the lattice constant)
\item {\tt Fractional} or {\tt ScaledByLatticeVectors} (atomic positions 
are given referred to the lattice vectors)
\end{itemize}

{\it Default value:} {\tt NotScaledCartesianBohr}


\item[{\bf AtomCoorFormatOut}] ({\it string}): 
\index{AtomCoorFormatOut@{\bf AtomCoorFormatOut}}
Character string to specify the format of the atomic positions in output.
Same possibilities as for input ({\bf AtomicCoordinatesFormat}).

{\it Default value:} value of {\bf AtomicCoordinatesFormat}


\item[{\bf AtomicCoordinatesOrigin}] ({\it data block}): 
\index{AtomicCoordinatesOrigin@{\bf AtomicCoordinatesOrigin}} 
Vector specifying a rigid shift to apply to the atomic coordinates,
given in the same format and units as these. Notice that the atomic
positions (shifted or not) need not be within the cell formed by
{\bf LatticeVectors}, since periodic boundary conditions are always
assumed.

{\it Default value:} 
{\tt
\begin{verbatim}
  0.000   0.000   0.000
\end{verbatim}
}

\item[{\bf AtomicCoordinatesAndAtomicSpecies}] ({\it data block}): 
\index{AtomicCoordinatesAndAtomicSpecies@{\bf 
AtomicCoordinatesAndAtomicSpecies}} 
Block specifying the position and species of each atom.
One line per atom, the reading is done this way:
\begin{verbatim}
       From ia = 1 to natoms
            read: xa(ix,ia), isa(ia)
\end{verbatim}
where {\tt xa(ix,ia)} is the {\tt ix} coordinate of atom 
{\tt iai} in the format (units) specified by
{\bf AtomCoordinatesFormat}, and {\tt isa(ia)} is the species 
index of atom {\tt ia}.

{\it Default:} There is no default. The positions must be introduced
either using this block or the $Z$ matrix (see {\bf Zmatrix}).

\item[{\bf UseStructFile}] ({\it logical}): 
\index{UseStructFile@{\bf UseStructFile}}
\index{Systemlabel.STRUCT\_IN@{{\it Systemlabel}.STRUCT\_IN}}
Logical variable to control whether the structural information
is read from an external file of name {\it SystemLabel}.STRUCT\_IN.
If \texttt{.true.}, all other structural information in the fdf file
will be ignored. 

The format of the file is implied by the following code:

\begin{verbatim}
read(*,*) ((cell(ixyz,ivec),ixyz=1,3),ivec=1,3)  ! Cell vectors, in Angstroms
read(*,*) na
do ia = 1,na
   read(iu,*) isa(ia), dummy, xfrac(1:3,ia)  ! Species number
                                             ! Dummy numerical column
                                             ! Fractional coordinates
enddo
\end{verbatim}

{\it Warning:} Note that the resulting geometry could be clobbered if
an XV file is read after this file. It is up to the user to remove
any XV files.\index{MD.UseSaveXV@{\bf MD.UseSaveXV}}.

{\it Default value:} {\tt .false.}


\item[{\bf SuperCell}] ({\it data block}): 
\index{SuperCell@{\bf SuperCell}} 
Integer 3x3 matrix defining a supercell in terms of the unit cell: 

\begin{verbatim}
     %block SuperCell
        M(1,1)  M(2,1)  M(3,1) 
        M(1,2)  M(2,2)  M(3,2) 
        M(1,3)  M(2,3)  M(3,3) 
     %endblock SuperCell
\end{verbatim}

and the supercell is defined as
$SuperCell(ix,i) = \sum_j CELL(ix,j)*M(j,i)$.
Notice that the matrix indexes are inverted: each input line 
specifies one supercell vector.

{\it Warning:} {\bf SuperCell} is disregarded if the geometry is read from
the XV file, which can happen unadvertedly.

{\it Use:} The atomic positions must be given only for the unit cell,
and they are 'cloned' automatically in the rest of the supercell.
The {\bf NumberOfAtoms} given must also be that in a single unit cell.
However, all values in the output are given for the entire supercell. 
In fact, CELL is inmediately redefined as the whole supercell and the 
program no longer knows the existence of an underlying unit cell.
All other input (apart from NumberOfAtoms and atomic positions), 
including {\bf kgridMonkhorstPack} must refer to the supercell 
(this is a change over the previous version). Therefore, to avoid
confusions, we recommend to use {\bf SuperCell} only to
generate atomic positions, and then to copy them from the output
to a new input file with all the atoms specified explicitly and
with the supercell given as a normal unit cell.

{\it Default value:} No supercell (supercell equal to unit cell).

\item[{\bf FixAuxiliaryCell}] ({\it logical}): 
\index{FixAuxiliaryCell@{\bf FixAuxiliaryCell}} 

Logical variable to control whether the auxiliary cell is changed
during a variable cell optimisation.

\item[{\bf NaiveAuxiliaryCell}] ({\it logical}): 
\index{NaiveAuxiliaryCell@{\bf NaiveAuxiliaryCell}} 

If true, the program does not check whether the auxiliary cell
constructed with a naive algorithm is appropriate. This variable
is only useful if one wishes to reproduce calculations done with
previous versions in which the auxiliary cell was not large enough, 
as indicated by warnings such as:\\
\texttt{xijorb: WARNING: orbital pair 1  341 is multiply connected}

Only small numerical differences in the results are to be expected.

{\it Default value:} {\tt .false.}


\item[{\bf GeometryConstraints}] ({\it data block}) 
\index{GeometryConstraints@{\bf GeometryConstraints}}
\index{constraints in relaxations}
Fixes constraints to the change of atomic coordinates during
geometry relaxation or molecular dynamics. Allowed constraints are:
\begin{itemize}
\item {\tt cellside}: fixes the unit-cell side lengths to
their initial values (not implemented yet).
\item {\tt cellangle}: fixes the unit-cell angles to
their initial values (not implemented yet).
\item {\tt stress}: fixes the specified stresses to
their initial values.
\item {\tt position}: fixes the positions of the specified atoms to
their initial values.
\item {\tt center}: fixes the center (mean position, not center of
mass) of a group of atoms to its initial value (not implemented yet).
\item {\tt rigid}: fixes the relative positions of a group of atoms,
without restricting their displacement or rotation as a rigid unit
(not implemented yet).
\item {\tt routine}: Additionally, the user may write a 
problem-specific routine called {\bf constr} (with the same 
interface as in the example below), which inputs the atomic
forces and stress tensor and outputs them orthogonalized to the
constraints. For example, to maintain the relative height of 
atoms 1 and 2:

\begin{verbatim}
      subroutine constr( cell, na, isa, amass, xa, stress, fa )
c real*8  cell(3,3)    : input lattice vectors (Bohr)
c integer na           : input number of atoms
c integer isa(na)      : input species indexes
c real*8  amass(na)    : input atomic masses
c real*8  xa(3,na)     : input atomic cartesian coordinates (Bohr)
c real*8  stress( 3,3) : input/output stress tensor (Ry/Bohr**3)
c real*8  fa(3,na)     : input/output atomic forces (Ry/Bohr)
c integer ntcon        : output total number of position constraints
c                        imposed in this routine
      integer na, isa(na), ntcon
      double precision amass(na), cell(3,3), fa(3,na),
     .                 stress(3,3), xa(3,na), fz
      fz = fa(3,1) + fa(3,2) 
      fa(3,1) = fz * amass(1)/(amass(1)+amass(2))
      fa(3,2) = fz * amass(2)/(amass(1)+amass(2))
      ntcon=1
      end
\end{verbatim}

NOTE that the input of the routine {\bf constr} has changed
with respect to {\sc Siesta} versions prior to 1.3. Now, it includes the
argument {\it ntcon}, where the routine should
store the number of position constraints imposed in it,
as an output.  The user should update older {\bf constr} routines
accordingly. In the example above, the number of constraints is one,
since only the relative z position of two atoms is constrained
to be constant.

\end{itemize}

Example: consider a diatomic molecule (atoms 1 and 2) above a surface, 
represented by a slab of 5 atomic layers, with 10 atoms per layer.
To fix the cell height, the slab's bottom layer (last 10 atoms),
the molecule's interatomic distance, its height above the surface and
the relative height of the two atoms
(but not its azimuthal orientation and lateral position):

\begin{verbatim}
     %block GeometryConstraints
        cellside   c 
        cellangle  alpha  beta  gamma
        position  from -1 to -10
        rigid  1  2
        center 1  2   0.0  0.0  1.0
        stress 4 5 6
        routine constr
     %endblock GeometryConstraints
\end{verbatim}

The first line fixes the height of the unit cell, leaving the width
and depth free to change (with the appropriate type of dynamics).  The
second line fixes all three unit-cell angles.  The third line fixes
all three coordinates of atoms 1 to 10, counted backwards from the
last one (you may also specify a given direction, like in center).
The fourth line specifies that atoms 1 and 2 form a rigid unit.  The
fifth line fixes the center of the molecule (atoms 1 and 2), in the z
direction (0.,0.,1.). This vector is given in cartesian coordinates
and, without it, all three coordinates will be fixed (to fix a center,
or a position, in the $x$ and $y$ directions, but not in the $z$
direction, two lines are required, one for each direction).  The sixth
line specifies that the stresses 4, 5 and 6 should be fixed.  The
convention used for numbering stresses is that 1=xx,2=yy,3=zz,
4=yz,5=xz,6=xy.  The list of atoms for a given constraint may contain
several atoms (as in lines 4 and 5) {\it or} a range (as in the third
line), but not both. But you may specify many constraints of the same
type, and a total of up to 10000 lines in the block.  Lines may be up
to 130 characters long. Ranges of atoms in a line may contain up to
1000 atoms. All names must be in lower case.

Notice that, if you only fix the position of one atom, the rest of the
system will move to reach the same relative position. In order to
fix the {\it relative} atomic position, you may fix the center of
the whole system by including a line specifying 'center'
without any list or range of atoms (though possibly with a direction).

Constraints are imposed by suppressing the forces in those directions,
before applying them to move the atoms. For nonlinear constraints
(like 'rigid'), this does not impose the exact conservation of the 
constrained magnitude, unless the displacement steps are very small.

{\it Default value:} No constraints

\item[{\bf Zmatrix}] ({\it data block}): 
\index{Zmatrix@{\bf Zmatrix}} 
This block provides a means for inputting the system geometry using a Z-matrix
format, as well as controlling the optimization variables. This is particularly
useful when working with molecular systems or restricted optimizations (such
as locating transition states or rigid unit movements). The format also
allows for hybrid use of Z-matrices and Cartesian or fractional blocks, as
is convenient for the study of a molecule on a surface.
As is always the case for a Z-matrix, the responsibility falls to the user to
chose a sensible relationship between the variables to avoid triads of atoms
that become linear. 

Below is an example of a Z-matrix input for a water molecule:
\begin{verbatim}
    %block Zmatrix
    molecule fractional
      1 0 0 0   0.0 0.0 0.0 0 0 0
      2 1 0 0   HO1 90.0 37.743919 1 0 0
      2 1 2 0   HO2 HOH 90.0 1 1 0
    variables
        HO1 0.956997
        HO2 0.956997
        HOH 104.4
    %endblock Zmatrix
\end{verbatim}

The sections that can be used within the Zmatrix block are as follows:

Firstly, all atomic positions must be specified within either a 
``\texttt{molecule}'' block or a ``\texttt{cartesian}'' block. 
Any atoms subject to
constraints more complicated than ``do not change this coordinate
of this atom'' must be specified within a ``\texttt{molecule}'' block.

\item \texttt{molecule}: 

There must be one of these blocks for each independent set of
constrained atoms within the simulation.

This specifies the atoms that make up each molecule and their
geometry. In addition, an option of ``\texttt{fractional}'' or ``\texttt{scaled}''
may be passed, which indicates that distances are specified
in scaled or fractional units. In the absence of such an option, the 
distance units are taken to be the value of ``\texttt{ZM.UnitsLength}''.
 
A line is needed for each atom in the molecule; the format of each line should be:

\noindent\texttt{    Nspecies i j k r a t ifr ifa ift}

Here the values \texttt{Nspecies}, \texttt{i}, \texttt{j}, \texttt{k}, \texttt{ifr}, \texttt{ifa}, and \texttt{ift} are integers and \texttt{r}, \texttt{a}, and \texttt{t}
are double precision reals. 

For most atoms, \texttt{Nspecies} is the species number of the atom, \texttt{r} is distance to atom number
\texttt{i}, \texttt{a} is the angle made by the present atom with atoms \texttt{j} and \texttt{i}, while \texttt{t} is the torsional
angle made by the present atom with atoms \texttt{k}, \texttt{j}, and \texttt{i}. The values \texttt{ifr}, \texttt{ifa} and \texttt{ift} are
integer flags that indicate whether \texttt{r}, \texttt{a}, and \texttt{t}, respectively, should be varied; 0 for fixed, 1 for varying.


The first three atoms in a molecule are a special case. Because there are insufficient
atoms defined to specify a distance/angle/torsion, the values are set differently. For
atom 1, \texttt{r}, \texttt{a}, and \texttt{t}, are the Cartesian coordinates of the atom.  For the
second atom, \texttt{r}, \texttt{a}, and \texttt{t} are the coordinates in spherical form of the second atom
relative to the first. Finally, for the third atom, the numbers take their normal form,
but the torsional angle is defined relative to a notional atom 1 unit in the z-direction
above the atom \texttt{j}.

Secondly. blocks of atoms all of which are subject to the simplest of constraints
may be specified in one of the following three ways, according to the units used
to specify their coordinates:


\item \texttt{cartesian}: This section specifies a block of atoms whose coordinates are to be
specified in Cartesian coordinates. Again, an option of ``\texttt{fractional}'' or ``\texttt{scaled}''
may be added, to specify the units used; and again, in their absence, the value of 
``\texttt{ZM.UnitsLength}'' is taken.

The format of each atom in the block will look like:

\noindent\texttt{      Nspecies x y z ix iy iz}

Here \texttt{Nspecies}, \texttt{ix}, \texttt{iy}, and \texttt{iz} are integers and \texttt{x}, \texttt{y}, \texttt{z} are reals. \texttt{Nspecies} is the
species number of the atom being specified, while \texttt{x}, \texttt{y}, and \texttt{z} are the Cartesian]
coordinates of the atom in whichever units are being used. The values \texttt{ix}, \texttt{iy} and \texttt{iz} 
are integer flags that indicate whether the \texttt{x}, \texttt{y}, and \texttt{z} coordinates, respectively,
should be varied or not. A value of 0 implies that the coordinate is fixed, while 1 implies that it should be varied.

A zmatrix block may also contain the following, additional, sections, which
are designed to make it easier to read.

\item \texttt{constants}: Instead of specifying a numerical value, it is possible to specify a symbol
within the above geometry definitions. This section allows the user to define the value
of the symbol as a constant. The format is just a symbol followed by the value:

\noindent\texttt{      HOH 104.4}

\item \texttt{variables}: Instead of specifying a numerical value, it is possible to specify a symbol
within the above geometry definitions. This section allows the user to define the value
of the symbol as a variable. The format is just a symbol followed by the value:

\noindent\texttt{      HO1 0.956997}

Finally, constraints must be specified in a \texttt{constraints} block.

\item \texttt{constraint} This sub-section allows the user to create constraints between symbols used
in a Z-matrix:
\begin{verbatim}
    constraint
      var1 var2 A B
\end{verbatim}
Here var1 and var2 are text symbols for two quantities in the Z-matrix definition, and A
and B are real numbers. The variables are related by $var1 = A*var2 + B$.

An example of a Z-matrix input for a benzene molecule over a metal surface is:
\begin{verbatim}
    %block Zmatrix
    molecule
      2 0 0 0 xm1 ym1 zm1 0 0 0
      2 1 0 0 CC 90.0 60.0 0 0 0
      2 2 1 0 CC CCC 90.0 0 0 0
      2 3 2 1 CC CCC 0.0 0 0 0
      2 4 3 2 CC CCC 0.0 0 0 0
      2 5 4 3 CC CCC 0.0 0 0 0
      1 1 2 3 CH CCH 180.0 0 0 0
      1 2 1 7 CH CCH 0.0 0 0 0
      1 3 2 8 CH CCH 0.0 0 0 0
      1 4 3 9 CH CCH 0.0 0 0 0
      1 5 4 10 CH CCH 0.0 0 0 0
      1 6 5 11 CH CCH 0.0 0 0 0
    fractional
      3 0.000000 0.000000 0.000000 0 0 0
      3 0.333333 0.000000 0.000000 0 0 0
      3 0.666666 0.000000 0.000000 0 0 0
      3 0.000000 0.500000 0.000000 0 0 0
      3 0.333333 0.500000 0.000000 0 0 0
      3 0.666666 0.500000 0.000000 0 0 0
      3 0.166667 0.250000 0.050000 0 0 0
      3 0.500000 0.250000 0.050000 0 0 0
      3 0.833333 0.250000 0.050000 0 0 0
      3 0.166667 0.750000 0.050000 0 0 0
      3 0.500000 0.750000 0.050000 0 0 0
      3 0.833333 0.750000 0.050000 0 0 0
      3 0.000000 0.000000 0.100000 0 0 0
      3 0.333333 0.000000 0.100000 0 0 0
      3 0.666666 0.000000 0.100000 0 0 0
      3 0.000000 0.500000 0.100000 0 0 0
      3 0.333333 0.500000 0.100000 0 0 0
      3 0.666666 0.500000 0.100000 0 0 0
      3 0.166667 0.250000 0.150000 0 0 0
      3 0.500000 0.250000 0.150000 0 0 0
      3 0.833333 0.250000 0.150000 0 0 0
      3 0.166667 0.750000 0.150000 0 0 0
      3 0.500000 0.750000 0.150000 0 0 0
      3 0.833333 0.750000 0.150000 0 0 0
    constants
        ym1 3.68
    variables
        zm1 6.9032294
        CC 1.417
        CH 1.112
        CCH 120.0
        CCC 120.0
    constraints
        xm1 CC -1.0 3.903229
    %endblock Zmatrix
\end{verbatim}

Here the species 1, 2 and 3 represent H, C, and the metal of the
surface, respectively.

(Note: the above example shows the usefulness of symbolic names 
for the relevant coordinates, in particular for those which are
allowed to vary. The current output options for Zmatrix information
work best when this approach is taken. By using a ``fixed'' symbolic
zmatrix block and specifying the actual coordinates in a ``variables''
section, one can monitor the progress of the optimization and
easily reconstruct the coordinates of intermediate steps in the
original format.)

{\it Use:} Specifies the geometry of the system according to a Z-matrix
{\it Default value:} Geometry is not specified using a Z-matrix

\item[{\bf ZM.UnitsLength}] ({\it length}):
\index{ZM.UnitsLength@{\bf ZM.UnitsLength}}
Parameter that specifies the units of length used during Z-matrix input.

{\it Use:} This option allows the user to chose between inputing distances
in Bohr or Angstroms within the Z-matrix data block.

{\it Default value:} {\tt Bohr}

\item[{\bf ZM.UnitsAngle}] ({\it angle}):
\index{ZM.UnitsAngle@{\bf ZM.UnitsAngle}}
Parameter that specifies the units of angles used during Z-matrix input.

{\it Use:} This option allows the user to chose between inputing angles
in radians or degrees within the Z-matrix data block.

{\it Default value:} {\tt rad}

\item[{\bf ZM.ForceTolLength}] ({\it real force}):
\index{ZM.ForceTolLength@{\bf ZM.ForceTolLength}}
Parameter that controls the convergence with respect to forces on Z-matrix lengths

{\it Use:} This option sets the convergence criteria for the forces that
act on Z-matrix components with units of length.

{\it Default value:} {\tt $0.00155574$ Ry/Bohr}

\item[{\bf ZM.ForceTolAngle}] ({\it torque}):
\index{ZM.ForceTolAngle@{\bf ZM.ForceTolAngle}}
Parameter that controls the convergence with respect to forces on Z-matrix angles

{\it Use:} This option sets the convergence criteria for the forces that
act on Z-matrix components with units of angle.

{\it Default value:} {\tt $0.00356549$ Ry/rad}

\item[{\bf ZM.MaxDisplLength}] ({\it real length}):
\index{ZM.MaxDisplLength@{\bf ZM.MaxDisplLength}}
Parameter that controls the maximum change in a Z-matrix length during an 
optimisation step.

{\it Use:} This option sets the maximum displacement for a Z-matrix length

{\it Default value:} {\tt 0.2 Bohr}

\item[{\bf ZM.MaxDisplAngle}] ({\it real angle}):
\index{ZM.MaxDisplAngle@{\bf ZM.MaxDisplAngle}}
Parameter that controls the maximum change in a Z-matrix angle during an 
optimisation step.

{\it Use:} This option sets the maximum displacement for a Z-matrix angle

{\it Default value:} {\tt 0.003 rad }



\item[{\bf kgrid\_cutoff}] ({\it real length}): 
\index{kgrid\_cutoff@{\bf kgrid\_cutoff}}
Parameter which determines
the fineness of the k-grid used for Brillouin zone sampling.
It is half the length of the smallest lattice vector of the supercell 
required to obtain the same sampling precision with a single k point.
Ref: Moreno and Soler, PRB 45, 13891 (1992).

{\it Use:} If it is zero, only the gamma point is used.
The resulting k-grid is chosen in an optimal way, according to the
method of Moreno and Soler (using an effective supercell which is
as spherical as possible, thus minimizing the number of k-points for
a given precision). The grid is displaced for even numbers of
effective mesh divisions.
This parameter is not used if {\bf kgrid\_Monkhorst\_Pack} is specified.

{\it Default value:} {\tt 0.0 Bohr}
        

\item[{\bf kgrid\_Monkhorst\_Pack}] ({\it data block}): 
\index{kgrid\_Monkhorst\_Pack@{\bf kgrid\_Monkhorst\_Pack}} 
Real-space supercell, whose reciprocal unit cell is that of the
k-sampling grid, and grid displacement for each grid coordinate.
Specified as an integer matrix an a real vector:

\begin{verbatim}
     %block kgrid_Monkhorst_Pack
        Mk(1,1)  Mk(2,1)  Mk(3,1)   dk(1) 
        Mk(1,2)  Mk(2,2)  Mk(3,2)   dk(2) 
        Mk(1,3)  Mk(2,3)  Mk(3,3)   dk(3) 
     %endblock kgrid_Monkhorst_Pack 
\end{verbatim}

where {\tt Mk(j,i)} are integers and {\tt dk(i)} are usually
either 0.0 or 0.5 (the program will warn the user if the displacements
chosen are not optimal).
The k-grid supercell is defined from {\tt Mk}
as in block {\bf SuperCell} above, i.e.:
$KgridSuperCell(ix,i) = \sum_j CELL(ix,j)*Mk(j,i)$.
Note again that the matrix indexes are inverted: each input line 
gives the decomposition of a supercell vector in terms of the unit
cell vectors.


{\it Use:} Used only if {\bf SolutionMethod} = {\tt diagon}.
The k-grid supercell is compatible and unrelated 
(except for the default value, see below)
with the {\bf SuperCell} specifier. Both supercells are given in 
terms of the CELL specified by the {\bf LatticeVectors} block.
If {\tt Mk} is the identity matrix and {\tt dk} 
is zero, only the $\Gamma$ point of the {\bf unit} cell is used. 
Overrides {\bf kgrid\_cutoff}

{\it Default value:} $\Gamma$ point of the (super)cell.
(Default used only when {\bf kgrid\_cutoff} is not defined).
        

\item[{\bf BandLinesScale}] ({\it string}): 
\index{BandLinesScale@{\bf BandLinesScale}} 
Specifies the scale of the k vectors given in {\bf BandLines} 
and {\bf BandPoints} below.
The options are:
\begin{itemize}
\item {\tt pi/a} (k-vector coordinates are given in cartesian 
coordinates, in units of $\pi/a$, where $a$ is the lattice constant)
\item {\tt ReciprocalLatticeVectors} (k vectors are given in
reciprocal-lattice-vector coordinates)
\end{itemize}

{\it Default value:} {\tt pi/a}


\item[{\bf BandLines}] ({\it data block}): 
\index{BandLines@{\bf BandLines}} 
Specifies the lines along which band energies are calculated
(usually along high-symmetry directions).
An example for an FCC lattice is:

\begin{verbatim}
     %block BandLines
       1  1.000  1.000  1.000  L        # Begin at L
      20  0.000  0.000  0.000  \Gamma   # 20 points from L to gamma
      25  2.000  0.000  0.000  X        # 25 points from gamma to X
      30  2.000  2.000  2.000  \Gamma   # 30 points from X to gamma
     %endblock BandLines
\end{verbatim}

where the last column is an optional LaTex label for use in the band plot.
If only given points (not lines) are required, simply specify 1 in the
first column of each line. The first column of the first line must be 
always 1.

{\it Use:} Used only if {\bf SolutionMethod} = {\tt diagon}.
The band k points are unrelated and compatible with any k-grid used
to calculate the total energy and charge density.
This block is superseded by {\bf BandPoints} if both are present.

{\it Default value:} No band energies calculated.

\item[{\bf BandPoints}] ({\it data block}): 
\index{BandPoints@{\bf BandPoints}} 
Band energies are calculated for the list of arbitrary $k$ points 
given in the block. Units defined by {\bf BandLinesScale} as 
for {\bf BandLines}. The generated {\it Systemlabel}.{\tt bands} file
will contain the $k$ point coordinates (in a.u.) and the corresponding
band energies (in eV). Example:

\begin{verbatim}
     %block BandPoints
        0.000  0.000  0.000   # This is a comment. eg this is gamma
        1.000  0.000  0.000 
        0.500  0.500  0.500   
     %endblock BandPoints
\end{verbatim}

{\it Use:} Used only if {\bf SolutionMethod} = {\tt diagon}.
The band $k$ points are unrelated and compatible with any k-grid used
to calculate the total energy and charge density. If both present, this 
block supersedes {\bf BandLines}.

{\it Default value:} No band energies calculated.

\item[{\bf WaveFuncKPointsScale}] ({\it string}): 
\index{WaveFuncKPointsScale@{\bf WaveFuncKPointsScale}} 
Specifies the scale of the k vectors given in 
{\bf WaveFuncKPoints} below.
The options are:
\begin{itemize}
\item {\tt pi/a} (k-vector coordinates are given in cartesian 
coordinates, in units of $\pi/a$, where $a$ is the lattice constant)
\item {\tt ReciprocalLatticeVectors} (k vectors are given in
reciprocal-lattice-vector coordinates)
\end{itemize}

{\it Default value:} {\tt pi/a}


\item[{\bf WaveFuncKPoints}] ({\it data block}): 
\index{WaveFuncKPoints@{\bf WaveFuncKPoints}} 
Specifies the k-points at which the electronic wavefunction
coefficients are written. 
An example for an FCC lattice is:

\begin{verbatim}
     %block WaveFuncKPoints              
     0.000  0.000  0.000  from 1 to 10   # Gamma wavefuncs 1 to 10
     2.000  0.000  0.000  1 3 5          # X wavefuncs 1,3 and 5
     1.500  1.500  1.500                 # K wavefuncs, all
     %endblock WaveFuncKPoints
\end{verbatim}

The number of wavefunction is defined by its energy, so that the
first one has lowest energy.
The output of the wavefunctions in described in Section \ref{subsec:wf}

{\it Use:} Used only if {\bf SolutionMethod} = {\tt diagon}.
These k points are unrelated and compatible with any k-grid used
to calculate the total energy,  charge density and band structure.

{\it Default value:} No wavefunctions are written.

\end{description}



\vspace{5pt}
\subsection{DFT, Grid, SCF}

\begin{description}
\itemsep 10pt
\parsep 0pt

\item[{\bf Harris\_functional}] ({\it logical}): 
\index{Harris\_functional@{\bf Harris\_functional}} 
Logical variable to choose between self-consistent Kohn-Sham functional or
 non self-consistent Harris functional to calculate energies and forces. 
\begin{itemize}
\item {\tt .false.} : Fully self-consistent Kohn-Sham functional. 
\item {\tt .true.} : Non self consistent Harris functional. Cheap but
pretty crude for some systems. The forces are computed within the
Harris functional in the first SCF step. Only implemented for LDA in
the Perdew-Zunger parametrization.

When this option is choosen, the values of DM.UseSaveDM,
MaxSCFIterations, and DM.MixSCF1 are automatically set up to False, 1
and False respectively, no matter whatever other specification are in
the INPUT file.
\end{itemize}
    
{\it Default value:} {\tt .false.} 

\item[{\bf XC.functional}] ({\it string}): 
\index{XC.functional@{\bf XC.functional}} 
Exchange-correlation functional type. May be {\tt LDA} 
(local density approximation, equivalent to {\tt LSD}) or 
{\tt GGA} (Generalized Gradient Approximation). 

{\it Use:} Spin polarization is defined by SpinPolarized label for
both {\tt LDA} and {\tt GGA}. There is no difference between {\tt LDA}
and {\tt LSD}.

{\it Default value:} {\tt LDA}
        

\item[{\bf XC.authors}] ({\it string}): 
\index{XC.authors@{\bf XC.authors}} 
Particular parametrization of the
exchange-correlation functional. Options are:
\begin{itemize}
\item {\tt CA} (Ceperley-Alder) \index{CA}
equivalent to {\tt PZ} (Perdew-Zunger). \index{PZ}
Local density approximation.
Ref: Perdew and Zunger, PRB 23, 5075 (1981)
\item {\tt PW92} (Perdew-Wang-92). \index{PW92}
Local density approximation.\index{LDA}\index{LSD}
Ref: Perdew and Wang, PRB, 45, 13244 (1992)
\item {\tt PBE} (Perdew-Burke-Ernzerhof). Generalized gradients
approximation.  Ref: Perdew, Burke and Ernzerhof, PRL 77, 3865
(1996)\index{GGA} \index{PBE}
\item {\tt revPBE} (Revised Perdew-Burke-Ernzerhof). Generalized gradients
approximation.  Ref: Y. Zhang and W. Yang, PRL 80, 890 
(1998)\index{GGA} \index{revPBE}
\item {\tt RPBE} (Revised Perdew-Burke-Ernzerhof). Generalized gradients
approximation.  Ref: Hammer, Hansen and Norskov PRB 59, 7413 
(1999)\index{GGA} \index{RPBE}
\item {\tt WC} (Wu-Cohen modification of PBE functional). Generalized gradients
approximation.  Ref: Z. Wu and R.E. Cohen, PRB 73, 235116 (2006)
\index{GGA} \index{WC}
\item {\tt LYP} Generalized gradients approximation \index{BLYP}
that implements Becke gradient exchange functional (A. D.
Becke, Phys. Rev. A {\bf 38}, 3098 (1988)) and Lee, Yang, Parr
correlation functional (C. Lee, W. Yang, R. G. Parr, Phys. Rev. B
{\bf 37}, 785 (1988)), as modified by Miehlich, Savin, Stoll and Preuss,
Chem. Phys. Lett. {\bf 157}, 200 (1989). See also Johnson, Gill and Pople,
J. Chem. Phys. {\bf 98}, 5612 (1993). (Some errors were detected in this
last paper, so not all of their expressions correspond exactly to those
implemented in {\sc Siesta}).

\end{itemize}

{\it Use:} {\bf XC.functional} and {\bf XC.authors} must be compatible.

{\it Default value:} {\tt PZ}

\item[{\bf XC.hybrid}] ({\it data block}): 
\index{XC.hybrid@{\bf XC.hybrid}} 
This data block allows the user to create a hybrid functional by
mixing the desire amounts of exchange and correlation from each of
the functionals described under XC.authors.
The format of the block is that the first line must contain the
number of functionals to be mixed. On the subsequent lines the
values of XC.functl and XC.authors must be given and then the
weights for the exchange and correlation, in that order. If only
one number is given then the same weight is applied to both
exchange and correlation.

The following is an example in which a 75:25 mixture of Ceperley-Alder
and PBE correlation is made, with an equal split of the exchange
energy:

\begin{verbatim}
     %block XC.hybrid
        2
        LDA CA  0.5 0.75
        GGA PBE 0.5 0.25
     %endblock XC.hybrid
\end{verbatim}

{\it Default value:} {\tt not hybrid}


\item[{\bf SpinPolarized}] ({\it logical}): 
\index{SpinPolarized@{\bf SpinPolarized}} 
Logical variable to choose between spin unpolarized ({\tt .false.}) 
or spin polarized ({\tt .true.}) calculation.

{\it Default value:} {\tt .false.}


\item[{\bf NonCollinearSpin}] ({\it logical}): 
\index{NonCollinearSpin@{\bf NonCollinearSpin}}\index{spin}
\index{spin!non-collinear}\index{LSD}
If {\tt .true.}, non-collinear spin is described using spinor wavefunctions
and $(2 \times 2)$ spin density matrices at every grid point.
Ref: T. Oda et al, PRL, {\bf 80}, 3622 (1998).
Not compatible with GGA because non-collinear density functional
theory has been developped only for a local functional.

{\it Default value:} {\tt .false.}


\item[{\bf FixSpin}] ({\it logical}): 
\index{FixSpin@{\bf FixSpin}}\index{spin}
\index{fixed spin state}\index{LSD}
If {\tt .true.}, the calculation is done with a fixed value of the
spin of the system, defined by variable  {\bf TotalSpin}.
This option can only be used for collinear spin polarized
calculations.

{\it Default value:} {\tt .false.}

\item[{\bf TotalSpin}] ({\it real}):
\index{TotalSpin@{\bf TotalSpin}}
\index{spin} 
\index{fixed spin state}\index{LSD}
Value of the imposed total spin polarization of the system (in units of the
electron spin, 1/2). It is only used
if {\bf FixSpin} = {\tt .true.}

{\it Default value:} 0.0

\item[{\bf SingleExcitation}] ({\it logical}):
\index{SingleExcitation@{\bf SingleExcitation}}
If true, {\sc Siesta} calculates a very rough approximation to 
the lowest excited state by swapping the populations of the HOMO 
and the LUMO. If there is no spin polarisation, it is half swap only. 
It is done for the first spin component (up) and first k vector.

{\it Default value:} {\tt .false.}

\item[{\bf MeshCutoff}] ({\it real energy}): 
\index{MeshCutoff@{\bf MeshCutoff}}\index{grid}\index{mesh}  
Defines the equivalent plane wave cutoff for the grid.

{\it Default value:} {\tt 100 Ry}

\item[{\bf MeshSubDivisions}] ({\it integer}): 
\index{MeshSubDivisions@{\bf MeshSubDivisions}}\index{grid}\index{mesh}
Defines the number of sub-mesh points in each direction used
to save index storage on the mesh.

{\it Default value:} {\tt 2}

\item[{\bf MaxSCFIterations}] ({\it integer}): 
\index{MaxSCFIterations@{\bf MaxSCFIterations}}
Maximum number of SCF\index{SCF} iterations per time step.

{\it Default value:} {\tt 50}

\item[{\bf DM.MixingWeight}] ({\it real}):\index{SCF!mixing}
\index{DM.MixingWeight@{\bf DM.MixingWeight}}\index{SCF!mixing!linear} 
Proportion $\alpha$ of 
output Density Matrix to be used for the input Density Matrix of 
next SCF cycle (linear mixing):
$\rho^{n+1}_{in} = \alpha \rho^{n}_{out} 
+(1 - \alpha) \rho^{n}_{in}$.

{\it Default value:} {\tt 0.25}

\item[{\bf DM.NumberPulay}] ({\it integer}):\index{Pulay mixing} 
\index{DM.NumberPulay@{\bf DM.NumberPulay}}\index{SCF!mixing!Pulay} 
It controls the Pulay convergence accelerator. Pulay mixing generally
accelerates convergence quite significantly, and can
reach convergence in cases where linear mixing cannot.
%One Pulay mixing will be performed every {\bf DM.NumberPulay} SCF
%iterations, the other iterations using linear mixing. If 
%it is less than 2, only linear mixing is used.
The guess for the $n+1$ iteration is constructed using the
input and output matrices of the {\bf DM.NumberPulay} former
SCF cycles, in the following way:
$\rho^{n+1}_{in} = \alpha \bar{\rho}^{n}_{out} 
+(1 - \alpha) \bar{\rho}^{n}_{in}$, where $\bar{\rho}^{n}_{out}$
and $\bar{\rho}^{n}_{in}$ are constructed from the previous
$N=${\bf DM.NumberPulay} cycles:
%
\begin{equation}
\bar{\rho}^{n}_{out} = \sum_{i=1}^N
\beta_i \rho_{out}^{(n-N+i)} \hspace{0.5truecm}; \hspace{0.5truecm}
\bar{\rho}^{n}_{in} = \sum_{i=1}^N
\beta_i \rho_{in}^{(n-N+i)}.
\nonumber
\end{equation}
%
The values of $\beta_i$ are obtained by minimizing the distance
between $\bar{\rho}^{n}_{out}$ and $\bar{\rho}^{n}_{in}$.
The value of $\alpha$ is given by variable {\bf DM.MixingWeight}.

If {\bf DM.NumberPulay} is 0 or 1, simple linear mixing is
performed.

{\it Default value:} {\tt 0}

\item[{\bf DM.PulayOnFile}] ({\it logical}): 
\index{DM.PulayOnFile@{\bf DM.PulayOnFile}} 
Store intermediate information of Pulay mixing in files
({\tt .true.}) or in memory ({\tt .false.}).
Memory storage can increase considerably the
memory requirements for large systems.
If files are used, the filenames will be 
{\tt SystemLabel}.P1 and 
{\tt SystemLabel}.P2,
where SystemLabel is the name associated
to parameter {\tt SystemLabel}.

{\it Default value:} {\tt .false.}

\item[{\bf DM.NumberBroyden}] ({\it integer}):\index{Broyden mixing} 
\index{DM.NumberBroyden@{\bf DM.NumberBroyden}}\index{SCF!mixing!Broyden} 
It controls the Broyden-Vanderbilt-Louie-Johnson
convergence accelerator, which is based on the use of past information
(up to {\bf DM.NumberBroyden} steps) to construct the input density
matrix for the next iteration.

See D.D. Johnson, Phys. Rev. B{\bf 38}, 12807 (1988), and references therein;
Kresse and Furthmuller, Comp. Mat. Sci {\bf 6}, 15 (1996).

If {\bf DM.NumberBroyden} is 0, the program performs linear mixings,
or, if requested, Pulay mixings.

Broyden mixing takes precedence over Pulay mixing if both are
specified in the input file.

{\bf Note:} The Broyden mixing algorithm is still in development,
notably with regard to the effect of its various modes of operation, and
the assigment of weights. In its default mode, its effectiveness is
very similar to Pulay mixing. As memory usage is not yet optimized,
casual users might want to stick with Pulay mixing for now.

{\it Default value:} {\tt 0}

\item[{\bf DM.Broyden.Cycle.On.Maxit}] ({\it logical}): 
\index{DM.Broyden.Cycle.On.Maxit@{\bf DM.Broyden.Cycle.On.Maxit}} 
\index{SCF!mixing!Broyden} 
Upon reaching the maximum number of historical data sets which are
kept for Broyden mixing (see description of variable {\bf
  DM.NumberBroyden}), throw away the oldest and shift the rest to make
room for a new data set. This procedure tends, heuristically, to
perform better than the alternative, which is to re-start the Broyden
mixing algorithm from a first step of linear mixing.

{\it Default value:} {\tt .true.}

\item[{\bf DM.Broyden.Variable.Weight}] ({\it logical}): 
\index{DM.Broyden.Variable.Weight@{\bf DM.Broyden.Variable.Weight}} 
\index{SCF!mixing!Broyden} 
If {\tt .true.}, the different historical data sets used in
the Broyden mixing (see description of variable {\bf
  DM.NumberBroyden}) are assigned a weight depending on the
norm of their residual ${\rho}^{n}_{out}-{\rho}^{n}_{in}$.

{\it Default value:} {\tt .true.}

\item[{\bf DM.NumberKick}] ({\it integer}):\index{Linear mixing kick} 
\index{DM.NumberKick@{\bf DM.NumberKick}}
%\index{SCF!mixing!linear!Pulay!Broyden} 
\index{SCF!mixing!linear} 
Option to skip the Pulay (or Broyden) mixing earch certain number of iterations,
and use a linear mixing instead. Linear mixing is done
every {\bf DM.NumberKick} iterations, using a mixing coefficient
$\alpha$ given by variable {\bf DM.KickMixingWeight} 
(instead of the usual mixing {\bf DM.MixingWeight}).
This allows in some difficult cases to bring the SCF out of
loops in which the selfconsistency is stuck.
If {\bf DM.MixingWeight}=0, no linear mix is used.

{\it Default value:} {\tt 0}

\item[{\bf DM.KickMixingWeight}] ({\it real}):\index{SCF!mixing!Pulay!Broyden}
\index{DM.KickMixingWeight@{\bf DM.KickMixingWeight}}
%\index{SCF!mixing!linear!Pulay!Broyden} 
\index{SCF!mixing!linear} 
Proportion $\alpha$ of 
output Density Matrix to be used for the input Density Matrix of 
next SCF cycle (linear mixing):
$\rho^{n+1}_{in} = \alpha \rho^{n}_{out} 
+(1 - \alpha) \rho^{n}_{in}$, for linear mixing kicks within the
Pulay or Broyden mixing schemes. 
This mixing is done every {\bf DM.NumberKick} cycles.

{\it Default value:} {\tt 0.50}


\item[{\bf DM.MixSCF1}] ({\it logical}):\index{SCF!mixing}
\index{DM.MixSCF1@{\bf DM.MixSCF1}}\index{SCF!mixing}
Logical variable to indicate whether mixing is done in the
first SCF cycle or not. Usually, mixing should not be done in
the first cycle, to avoid non-idempotency in density matrix
from Harris or previous steps. It can be useful, though,
for restarts of selfconsistency runs.

{\it Default value:} {\tt .false.}


\item[{\bf DM.Tolerance}] ({\it real}): 
\index{DM.Tolerance@{\bf DM.Tolerance}} 
Tolerance of Density Matrix.
When the maximum difference between the output and the
input on each element of the DM 
in a SCF cycle is smaller than DM.Tolerance,
the selfconsistency has been achieved.

{\it Default value:} {${\tt 10^{-4}}$}


\item[{\bf DM.EnergyTolerance}] ({\it real}): 
\index{DM.EnergyTolerance@{\bf DM.EnergyTolerance}} 
When the change in the total energy between cycles
of the SCF procedure is below this value and the
density matrix change criterion is also satisfied
then self-consistency has been achieved.

{\it Default value:} {${\tt 10^{-4}}$}

\item[{\bf DM.InitSpinAF}] ({\it logical}):\index{spin}
\index{spin!initialization}\index{ferromagnetic initial DM}
\index{antiferromagnetic initial DM} 
\index{DM.InitSpinAF@{\bf DM.InitSpinAF}} 
It defines the initial spin density for a spin polarized calculation. 
The spin density is initially constructed with the maximum possible
spin polarization for each atom in its atomic configuration.
This variable defines the relative orientation of the atomic
spins: 

\begin{itemize}
\item {\tt .false.} gives ferromagnetic order (all spins up).
\item {\tt .true.} gives antiferromagnetic order. Up and down are
assigned according to order in the block 
{\bf AtomicCoordinatesAndAtomicSpecies}: up for the odd atoms, down for even.
\end{itemize}

{\it Default value:} {\tt .false.}


\item[{\bf DM.InitSpin}] ({\it data block}): 
\index{DM.InitSpin@{\bf DM.InitSpin}} It defines the
initial spin density for a spin polarized calculation atom by atom.
In the block there is one line per atom to be spin-polarized, 
containing the atom index (integer, ordinal in the block
{\bf AtomicCoordinatesAndAtomicSpecies}) and the desired
initial spin-polarization (real, positive for spin up, negative for
spin down). A value larger than possible will be reduced
to the maximum possible polarization, keeping its sign. 
Maximum polarization can also be given by introducing the
symbol {\tt +} or {\tt -} instead of the polarization value.
There is no need to include a line for every atom, only for
those to be polarized. The atoms not contemplated in the block will
be given non-polarized initialization.
For non-collinear spin, the spin direction may be specified for
each atom by the polar angles theta and phi, given as the last
two arguments in degrees. If not specified, theta=0 is assumed.
{\bf NonCollinearSpin} must be {\tt .true.} to use the spin direction.

Example:

\begin{verbatim}
     %block DM.InitSpin
        5  -1.   90.   0.   # Atom index, spin, theta, phi (deg)
        3   +    45. -90.
        7   -
     %endblock DM.InitSpin
\end{verbatim}

{\it Default value:} If present but empty, all atoms are not polarized. 
If absent, {\bf DM.InitSpinAF} defines the polarization.

\item[{\bf MullikenInSCF}] ({\it logical}):
\index{MullikenInSCF@{\bf MullikenInSCF}}
If true, the Mulliken populations will be written for every SCF step
at the level of detail specified in {\bf WriteMullikenPop}. Useful
when dealing with SCF problems, otherwise too verbose.

{\it Default value:} {\tt .false.}


\item[{\bf NeglNonOverlapInt}] ({\it logical}): 
\index{NeglNonOverlapInt@{\bf NeglNonOverlapInt}} 
Logical variable to neglect or compute interactions between orbitals
which do not overlap. These come from the KB projectors.
Neglecting them makes the Hamiltonian more sparse, and
the Order-N calculation faster.  USE WITH CARE!!!

{\it Default value:} {\tt .false.}
        
\item[{\bf ExternalElectricField}] ({\it data block}): 
\index{ExternalElectricField@{\bf ExternalElectricField}}
It specifies an external electric field for molecules, chains and slabs.
The electric field should be orthogonal to `bulk directions', like
those parallel to a slab (bulk electric fields, like in
dielectrics or ferroelectrics, are not allowed). If it is not, an
error message is issued and the components of the field in bulk
directions are suppressed automatically. The input is a
vector in cartesian coordinates, in the specified units. Example:

\begin{verbatim}
     %block ExternalElectricField
        0.000  0.000  0.500  V/Ang
     %endblock ExternalElectricField
\end{verbatim}

{\it Default value:} zero field

\item[{\bf PolarizationGrids}] ({\it data block}):
\index{PolarizationGrids@{\bf PolarizationGrids}}
\index{bulk polarization}\index{Berry phase}
If specified, the macroscopic polarization will be calculated using the
geometric Berry phase approach (R.D. King-Smith, and D. Vanderbilt,
PRB {\bf 47}, 1651 (1993)). In this method the electronic 
contribution to the macroscopic polarization, along a given direction, 
is calculated using
a discretized version of the formula
\begin{equation}
\label{pol_formula}
    P_{e,\parallel}={ifq_e \over 8\pi^3} \int_A d{\bf k}_\perp
    \sum_{n=1}^M \int_0^{|G_\parallel|} dk_{\parallel}
     \langle u_{{\bf k} n} |{\delta \over \delta k_{\parallel}} |
      u_{{\bf k} n} \rangle
\end{equation}
where $f$ is the occupation (2 for a non-magnetic system), 
$q_e$ the electron charge, $M$ is the number of occupied bands (the
system {\bf must} be an insulator), and $u_{{\bf k} n}$ are
the periodic Bloch functions. ${\bf G}_\parallel$ is the shortest
reciprocal vector along the chosen direction. 

As it can be seen in formula (\ref{pol_formula}), to compute each 
component of the polarization we must perform a surface integration
of the result of a 1-D integral in the selected direction. 
The grids for the calculation along the direction of each of the
three lattice vectors are specified in the block
{\bf PolarizationGrids}. 
\begin{verbatim}
     %block PolarizationGrids
        10   3  4      yes
         2  20  2       no
         4   4 15
     %endblock PolarizationGrids
\end{verbatim}      

All three grids must be specified, therefore a 3$\times$3 matrix of
integer numbers must be given: the first row specifies the grid that
will be used to calculate the polarization along the direction of the
first lattice vector, the second row will be used for the calculation
along the the direction of the second lattice vector, and the third
row for the third lattice vector.  The numbers in the diagonal of the
matrix specifie the number of points to be used in the one dimensional
line integrals along the different directions. The other numbers
specifie the mesh used in the surface integrals.  The last column
specifies if the bidimensional grids are going to be diplaced from the
origin or not, as in the Monkhorst-Pack algorithm (PRB {\bf 13}, 5188
(1976)).  This last column is optional.  If the number of point in one
of the grids is zero, the calculation will not be performed for this
particular direction.

For example, in the given example, for the computation in the
direction of the first lattice vector, 15 points will be used
for the line integrals, while a 3$\times$4 mesh will be used
for the surface integration. This last grid will be displaced 
from the origin, so $\Gamma$ will not be included in the 
bidimensional integral. For the directions of the second
and third lattice vectors, the number of points will be
20 and 2$\times$2, and 15 and 4$\times$4, respectively. 

It has to be stressed that the macroscopic polarization can only be
meaningfully calculated using this approach for insulators.
Therefore, the presence of an energy gap is necessary, and no band can
cross the Fermi level. The program performs a simple check of this
condition, just by counting the electrons in the unit cell ( the
number must be even for a non-magnetic system, and the total spin
polarization must have an integer value for spin polarized systems),
however is the responsability of the user to check that the system
under study is actually an insulator (for both spin components if spin
polarized).

The total macroscopic polarization, given in the output of the
program, is the sum of the electronic contribution (calculated as the
Berry phase of the valence bands), and the ionic contribution, which
is simply defined as the sum of the atomic positions within the unit
cell multiply by the ionic charges ($\sum_i^{N_a} Z_i {\bf r}_i$).  In
the case of the magnetic systems, the bulk polarization for each spin
component has been defined as
\begin{equation}
       {\bf P}^\sigma = {\bf P}_e^\sigma + 
   {1 \over 2} \sum_i^{N_a}  Z_i {\bf r}_i
\end{equation}
$N_a$ is the number of atoms in the unit cell, and ${\bf r}_i$ and
$Z_i$ 
are the positions and charges of the ions.

It is also worth noting, that the macroscopic polarization given by
formula (\ref{pol_formula}) is only defined modulo a ``quantum" of
polarization (the bulk polarization per unit cell is only well defined
modulo $fq_e${\bf R}, being {\bf R} an arbitrary lattice
vector). However, the experimentally observable quantities are
associated to changes in the polarization induced by changes on the
atomic positions (dynamical charges), strains (piezoelectric tensor),
etc... The calculation of those changes, between different
configurations of the solid, will be well defined as long as they are
smaller than the ``quantum", i.e. the perturbations are small enough
to create small changes in the polarization.

{\it Use:} Only compatible with {\bf SolutionMethod} = diagon.\\
{\it Default value:} Empty. No calculation performed.

\item[{\bf BornCharge}] ({\it logical}):\index{Born effective charges}
\index{BornCharge@{\bf BornCharge}}
If true, the Born effective charge tensor is calculated for each atom
by finite differences, by calculating the change in electric polarization 
(see {\bf PolarizationGrids}) induced by the small displacements generated 
for the force constants calculation (see {\bf MD.TypeOfRun} = {\tt FC}):
\begin{equation}\label {eq:effective_charge}
Z^*_{i,\alpha,\beta}=\frac{\Omega_0}{e} \left. {\frac{\partial{P_\alpha}}
{\partial{u_{i,\beta}}}}\right|_{q=0}
\end{equation}
where e is the charge of an electron and $\Omega_0$ is the unit cell volume.

To calculate the Born charges it is necessary to specify both the Born 
charge flag and the mesh used to calculate the polarization, for example:
\begin{verbatim}
%block PolarizationGrids
7  3  3
3  7  3
3  3  7
%endblock PolarizationGrids
BornCharge True
\end{verbatim}

The Born effective charge matrix is then written to the file 
{\it SystemLabel}.{\tt BC}.

The method by which the polarization is calculated may introduce an arbitrary 
phase (poalrization quantum), which in general is far larger than the change 
in polarization which results from the atomic displacement. It is removed 
during the calculation of the Born effective charge tensor. 

The Born effective charges allow the calculation of LO-TO splittings and 
infrared activities. The version of the Vibra utility code in which these
magnitudes are calculated is not yet distributed with {\sc Siesta}, but can be 
obtained form Tom Archer (archert@tcd.ie).

{\it Use:} Only used if {\bf MD.TypeOfRun} is {\tt FC}.

{\it Default value:} {\tt false}

\item[{\bf OpticalCalculation}] ({\it logical}):\index{Dielectric function,
optical absorption} 
If specified, the imaginary part of the dielectric function
will be calculated and stored in a file called {\it Systemlabel}.{\bf EPSIMG}.
The calculation is performed using the simplest approach based on the
dipolar transition matrix elements between different eigenfunctions
of the self-consistent Hamiltonian. For molecules the calculation 
is performed using the position operator matrix elements, while
for solids the calculation is carried out in the momentum space
formulation.
Corrections due to the non-locality of the pseudopotentials
are introduced in the usual way. 

{\it Default value:} {\tt false}

\item[{\bf Optical.EnergyMinimum}] ({\it real energy}):
\index{Optical.EnergyMinimum@{\bf Optical.EnergyMinimum}}
This specifies the minimum of the energy range in which
the frequency spectrum will be calculated.

{\it Default value:} 0 Ry.

\item[{\bf Optical.EnergyMaximum}] ({\it real energy}):
\index{Optical.EnergyMaximum@{\bf Optical.EnergyMaximum}}
This specifies the maximum of the energy range in which
the frequency spectrum will be calculated.

{\it Default value:} 10 Ry.

\item[{\bf Optical.Broaden}] ({\it real energy}):
\index{Optical.Broaden@{\bf Optical.Broaden}}
If this is value is set then a Gaussian broadening will be
applied to the frequency values.

{\it Default value:} 0 Ry.

\item[{\bf Optical.Scissor}] ({\it real energy}):
\index{Optical.Scissor@{\bf Optical.Scissor}}
Because of the tendency of DFT calculations to under estimate
the band gap, a rigid shift of the unoccupied states, known as 
the scissor operator, can be added to correct the gap and
thereby improve the calculated results. This shift is only
applied to the optical calculation and no where else within
the calculation.

{\it Default value:} 0 Ry.

\item[{\bf Optical.NumberOfBands}] ({\it integer}):
\index{Optical.NumberOfBands@{\bf Optical.NumberOfBands}}
This option controls the number of bands that are included in
the optical property calculation. Clearly this number must be
larger than the number of occupied bands and less than or
equal to the number of basis functions (which determines the
number of unoccupied bands available). Note, while including
all the bands may be the most accurate choice this will also
be the most expensive!

{\it Default value:} All bands.

\item[{\bf Optical.Mesh}] ({\it data block}):
\index{Optical.Mesh@{\bf Optical.Mesh}}
This block contains 3 numbers that determine the mesh size
used for the integration across the Brillouin zone. For
example:

\begin{verbatim}
        %block  Optical.Mesh
          5 5 5
        %endblock  Optical.Mesh
\end{verbatim}

The three values represent the number of mesh points in
the direction of each reciprocal lattice vector.

{\it Default value:} Empty in general. For atoms 
or molecules a k-sampling of only one point is assumed.

\item[{\bf Optical.OffsetMesh}] ({\it logical}):
\index{Optical.OffsetMesh@{\bf Optical.OffsetMesh}}
If set to true, then the mesh is offset away from the
gamma point for odd numbers of points.

{\it Default value:} {\tt false}

\item[{\bf Optical.PolarizationType}] ({\it string}):
\index{Optical.PolarizationType@{\bf Optical.PolarizationType}}
This option has three possible values that represent the
type of polarization to be used in the calculation. The options
are {\bf polarized}, which implies the application of an electric
field in a given direction, {\bf unpolarized}, which implies the
propagation of light in a given direction, and {\bf polycrystal}.
In the case of the first two options a direction in space must
be specified for the electric field or propagation using the
{\it Optical.Vector} data block.

{\it Default value:} {\tt polycrystal}

\item[{\bf Optical.Vector}] ({\it data block}):
\index{Optical.Vector@{\bf Optical.Vector}}
This block contains 3 numbers that specify the vector direction
for either the electric field or light propagation, for a polarized
or unpolarized calculation, respectively. A typical block might look
like:

\begin{verbatim}
        %block  Optical.Vector
          1.0 0.0 0.5
        %endblock  Optical.Vector
\end{verbatim}

{\it Default value:} Empty.


\item[{\bf GridCellSampling}] ({\it data block}):\index{egg-box effect} 
\index{GridCellSampling@{\bf GridCellSampling}}\index{rippling}
For improving grid-cutoff convergence. It specifies points within
the grid cell for a symmetrization sampling: the space 
homogeneity (traslational invariance)
is broken by the grid. This symmetry breaking is clear
when moving one single atom in an otherwise empty simulation cell.
The total energy oscillates with the grid periodicity when moving
it around, like on an egg-box. This effect tends to disappear with 
finer grids. For a given grid it can be eliminated
by recovering the lost symmetry: by symmetrizing the sensitive
quantities. The full symmetrization implies an integration (averaging)
over the grid cell. Instead, a finite sampling can be performed.

It is a sampling of rigid displacements of the system with respect
to the grid. The original grid-system setup (one point of the grid
at the origin) is always calculated. It is the (0,0,0) displacement. 
The block {\bf GridCellSampling} gives the additional displacements
wanted for the sampling. They are given relative to the grid-cell 
vectors, i.e., (1,1,1) would displace to the next grid point across
the body diagonal, giving an equivalent grid-system situation
(a useless displacement for a sampling).

Examples: Assume a cubic cell, and therefore a (smaller) cubic grid cell.
If there is no block or the block is empty, then the original (0,0,0)
will be used only. The block: 

\begin{verbatim}
     %block GridCellSampling
        0.5    0.5    0.5
     %endblock GridCellSampling
\end{verbatim}

would use the body center as a second point in the sampling. Or:


\begin{verbatim}
     %block GridCellSampling
        0.5    0.5    0.0
        0.5    0.0    0.5
        0.0    0.5    0.5
     %endblock GridCellSampling
\end{verbatim}

gives an fcc kind of sampling, and

\begin{verbatim}
     %block GridCellSampling
        0.5    0.0    0.0
        0.0    0.5    0.0
        0.0    0.0    0.5
        0.0    0.5    0.5
        0.5    0.0    0.5
        0.5    0.5    0.0
        0.5    0.5    0.5
     %endblock GridCellSampling
\end{verbatim}

gives again a cubic sampling with half the original side length.
It is not trivial to choose a right set of displacements so as
to maximize the new 'effective' cutoff. It depends on the 
kind of cell. It may be automatized in the future, but it 
is now left to the user, who introduces the displacements
manually through this block.

The quantities which are symmetrized are: ($i$) energy terms
that depend on the grid, ($ii$) forces, ($iii$) stress
tensor, and ($iv$) electric dipole.

The symmetrization is performed at the end of every SCF cycle. The 
whole cycle is done for the (0,0,0) displacement, and, when the
density matrix is converged, the same (now fixed)
density matrix is used to obtain the desired quantities at the 
other displacements (the density matrix itself is {\it not}
symmetrized as it gives a much smaller egg-box effect).
The CPU time needed for each displacement
in the {\bf GridCellSampling} block
is of the order of one extra SCF iteration.

{\it Default value:} Empty.


\item[{\bf EggboxRemove}] ({\it data block}):\index{egg-box effect}
\index{EggboxRemove@{\bf EggboxRemove}}\index{rippling}

For recovering translational invariance in an approximate way.
The introduction of a finite 3D grid for the calculation of integrals
causes the breaking of translational symmetry (the egg-box effect). 
This symmetry breaking is clear when
moving one single atom in an otherwise empty simulation cell. The total
energy and the forces oscillate with the grid periodicity when the
atom is moved, as if the atom were moving on an eggbox. In the limit of
infinitely fine grid (large mesh cutoff) this effect disappears.

For reasonable values of the mesh cutoff, the effect of the eggbox 
on the total energy or on the relaxed structure is normally unimportant.
However, it can affect substantially the process of relaxation, by 
increasing the number of steps considerably, and can also spoil the
calculation of vibrations, usually much more demanding than relaxations.

It works by substracting from Kohn-Sham's total energy (and forces) an
approximation to the eggbox energy, sum of atomic contributions. 
Each atom has a predefined eggbox energy depending on where it sits on
the cell. This atomic contribution is species dependent and is 
obviously invariant under grid-cell translations. Each species
contribution is thus expanded in the appropriate Fourier series.
It is important to have a smooth eggbox, for it to  
be represented by a few Fourier components. A jagged egg-box is often an
indication of a problem with the pseudo.

In the block there is one line per Fourier component. The first integer
is for the atomic species it is associated with. The other three
represent the reciprocal lattice vector of the grid cell (in units
of the basis vectors of the reciprocal cell). The real number is
the Fourier coefficient in units of the energy scale given in 
{\bf EggboxScale} (see below), normally 1 eV.

The number and choice of Fourier components is free, as well as their
order in the block. One can choose to correct only some species and not
others if, for instance, there is a substantial difference in hardness
of the cores. The 0 0 0 components will add a species-dependent
constant energy per atom. It is thus irrelevant except if comparing
total energies of different calculations, in which case they 
have to be considered with care (for instance by putting them all to zero,
i.e. by not introducing them in the list). 
The other components average to zero representing no bias in the
total energy comparisons. 

If the total
energies of the free atoms are put as 0 0 0 coefficients (with
spin polarisation if adequate etc.) the corrected total energy
will be the cohesive energy of the system (per unit cell).

{\it Example:} For a two species system, this example would give a quite
sufficent set in many instances (the actual values of the Fourier 
coefficients are not realistic).

\begin{verbatim}
     %block EggBoxRemove
       1   0   0   0 -143.86904
       1   0   0   1    0.00031
       1   0   1   0    0.00016
       1   0   1   1   -0.00015
       1   1   0   0    0.00035
       1   1   0   1   -0.00017
       2   0   0   0 -270.81903
       2   0   0   1    0.00015
       2   0   1   0    0.00024
       2   1   0   0    0.00035
       2   1   0   1   -0.00077
       2   1   1   0   -0.00075
       2   1   1   1   -0.00002
     %endblock EggBoxRemove
\end{verbatim}

It represents an alternative to grid-cell sampling (above).
It is only approximate, but once the Fourier components for each 
species are given, it does not represent any computational 
effort (neither memory nor time), while the grid-cell sampling
requires CPU time (roughly one extra SCF step per point every
MD step).

It will be particularly helpful in atoms with substantial partial
core or semicore electrons.

{\it Use:} This technique as it stands should only be used for fixed cell 
calculations. 

For the time being, it is up to the user to obtain the Fourier
components to be introduced. They can be obtrained by moving one
isolated atom through the cell to be used in the calculation
(for a give cell size, shape and mesh), once for each species. 
There is a utility program that does it, calling siesta (Tom Archer).

{\it Default value:} Empty.


\item[{\bf EggboxScale}] ({\it real energy}):\index{egg-box effect}
\index{EggboxScale@{\bf EggboxScale}}\index{rippling}

Defines the scale in which the Fourier components of the 
egg-box energy are given in the {\bf EggboxRemove} block.

{\it Default value:} 1 eV.

\end{description}



\vspace{5pt}
\subsection{Eigenvalue problem: order-$N$ or diagonalization}

\begin{description}
\itemsep 10pt
\parsep 0pt

\item[{\bf SolutionMethod}] ({\it string}): 
\index{SolutionMethod@{\bf SolutionMethod}} 
Character string to chose between
diagonalization ({\tt diagon}) or Order-N ({\tt OrderN}) solution
of the LDA Hamiltonian.

{\it Default value:} {\tt diagon} for 100 atoms or less in the 
{\bf unit} cell, {\tt OrderN} for more than 100 atoms.

%Diagonalisation options

\item[{\bf NumberOfEigenStates}] ({\it integer}): 
\index{NumberOfEigenStates@{\bf NumberOfEigenStates}} 
This parameter allows the user to reduce the number of eigenstates
that are calculated from the maximum possible. The benefit is that,
for a gamma point calculation, the cost of the diagonalisation is
reduced by finding fewer eigenvectors. For example, during a geometry
optimisation, only the occupied states are required rather than the
full set of virtual orbitals. Note, that if the electronic temperature
is greater than zero then the number of partially occupied states
increases, depending on the band gap.
The value specified must greater than the number of occupied states
and less than the number of basis functions.

{\it Default value:} {\tt all orbitals}

\item[{\bf Diag.DivideAndConquer}] ({\it logical}): 
\index{Diag.DivideAndConquer@{\bf Diag.DivideAndConquer}} 
Logical to select whether the normal or Divide and Conquer algorithms are
used within the Lapack diagonalisation routines. 

(Note: Some system library implementations of the D\&C algorithm are
buggy. It is advisable to use Siesta's own (fixed) version -- configure will
try to do that.)

{\it Default value:} {\tt true}

\item[{\bf Diag.AllInOne}] ({\it logical}): 
\index{Diag.AllInOne@{\bf Diag.AllInOne}} 
Logical to select whether a single call to lapack/scalapack is made to 
perform the diagonalisation or whether the individual steps are controlled
by {\sc Siesta}. Normally this option should not need to be used.

{\it Default value:} {\tt false}

\item[{\bf Diag.NoExpert}] ({\it logical}): 
\index{Diag.NoExpert@{\bf Diag.NoExpert}} 
Logical to select whether the simple or expert versions of the lapack/
scalapack routines are used. Usually the expert routines are faster, but
may require slightly more memory.

{\it Default value:} {\tt false}

\item[{\bf Diag.PreRotate}] ({\it logical}): 
\index{Diag.PreRotate@{\bf Diag.PreRotate}} 
Logical to select whether the eigensystem is transformed according to 
previously saved eigenvectors to create a near diagonal matrix and then 
back transformed afterwards. This is included for future options, but
currently should not make any difference except to increase the 
computational work!

{\it Default value:} {\tt false}

\item[{\bf Diag.Use2D}] ({\it logical}): 
\index{Diag.Use2D@{\bf Diag.Use2D}} 
Logical to select whether a 1-D or 2-D data decomposition should be used
when calling scalapack. The use of 2-D leads to superior scaling to 
large numbers of processors and is therefore the default. This option 
only influences the parallel performance.

{\it Default value:} {\tt true}

%End diagonalisation options

\item[{\bf OccupationFunction}]({\it string}):
\index{OccupationFunction@{\bf OccupationFunction}}
\index{ElectronicTemperature@{\bf ElectronicTemperature}}
String variable to select the function that determines the occupation
of the electronic states. Two options are available:
\begin{itemize}
\item {\tt FD}: The usual Fermi-Dirac occupation function is used.
\item {\tt MP}: The occupation function proposed by Methfessel and 
Paxton (Phys. Rev. B, {\bf 40}, 3616 (1989)), is used.
\end{itemize}
The smearing of the electronic occupations is done, in both cases,
using an energy width defined by the {\bf ElectronicTemperature}
variable. Note that, while in the case of Fermi-Dirac, the occupations
correspond to the physical ones if the electronic temperature is
set to the physical temperature of the system, this is not the case
in the Methfessel-Paxton function. In this case, the tempeature
is just a mathematical artifact to obtain a more accurate
integration of the physical quantities at a lower cost. In
particular, the Methfessel-Paxton scheme has the advantage
that, even for quite large smearing temperatures, the 
obtained energy is very close to the physical energy at T=0.
Also, it allows a much faster convergence with respect to
k-points, specially for metals. Finally, the convergence to
selfconsistency is very much improved (allowing the use
of larger mixing coefficients).

For the Methfessel-Paxton case, one can use relatively large
values for the {\bf ElectronicTemperature} parameter. How large
depends on the specific system. A guide can be found in the
article by J. Kresse and J. Furthm\"uller, Comp. Mat. Sci.
{\bf 6}, 15 (1996).

If Methfessel-Paxton smearing is used, the order of
the corresponding Hermite polynomial expansion must also be chosen
(see description of variable {\bf OccupationMPOrder}).

We finally note that, in both cases (FD and MP), once a finite
temperature has been chosen, the relevant energy is not the Kohn-Sham
energy, but the Free energy. In particular, the
atomic forces are derivatives of the Free energy, not the KS 
energy. See R. Wentzcovitch {\it et al.}, Phys. Rev. B {\bf 45},
11372 (1992); S. de Gironcoli, Phys. Rev. B {\bf 51}, 6773 (1995);
J. Kresse and J. Furthm\"uller, Comp. Mat. Sci.
{\bf 6}, 15 (1996),
for details.


{\it Use:} Used only if {\bf SolutionMethod} = {\tt diagon}

{\it Default value:} {\tt FD} 

\item[{\bf OccupationMPOrder}]({\it integer}):
\index{OccupationMPOrder@{\bf OccupationMPOrder}}
\index{OccupationFunction@{\bf OccupationFunction}}
Order of the Hermite-Gauss polynomial expansion for the
electronic occupation functions in the Methfessel-Paxton
scheme (see Phys. Rev. B  {\bf 40}, 3616 (1989)).
Specially for metals, higher order expansions provide better convergence
to the ground state result, even with larger smearing
temperatures, and provide also better convergence with k-points.


{\it Use:} Used only if {\bf SolutionMethod} = {\tt diagon}
and {\bf OccupationFunction} =  {\tt MP}

{\it Default value:} {\tt 1} 



\item[{\bf ElectronicTemperature}] ({\it real temperature or energy}): 
\index{ElectronicTemperature@{\bf ElectronicTemperature}}
\index{OccupationFunction@{\bf OccupationFunction}}
Temperature for Fermi-Dirac or Methfessel-Paxton
distribution. Useful specially for
metals, and to accelerate selfconsistency in some cases.

{\it Use:} Used only if {\bf SolutionMethod} = {\tt diagon}

{\it Default value:} {\tt 300.0 K}


\item[{\bf ON.functional}] ({\it string}): 
\index{ON.functional@{\bf ON.functional}} 
Choice of order-N minimization functionals: 
\begin{itemize}
\item {\tt Kim}:\index{Kim@{\tt Kim}} 
Functional of Kim, Mauri and Galli, PRB 52, 1640 (1995).
\item {\tt Ordejon-Mauri}:\index{Ordejon-Mauri@{\tt Ordejon-Mauri}} 
Functional of Ordej\'on et al, or Mauri et al, see PRB 51, 1456 (1995).
\item {\tt files}:\index{files (ON.functional)@{\tt files} (ON.functional)} 
Reads localized-function information from a file and 
chooses automatically the functional to be used. 
\end{itemize}

{\it Use:} Used only if {\bf SolutionMethod} = {\tt ordern}

{\it Default value:} {\tt Kim}

\item[{\bf ON.MaxNumIter}] ({\it integer}): 
\index{ON.MaxNumIter@{\bf ON.MaxNumIter}} 
Maximum number of iterations
in the conjugate minimization of the electronic
energy, in each SCF cycle.

{\it Use:} Used only if {\bf SolutionMethod} = {\tt OrderN}

{\it Default value:} {\tt 1000}

\item[{\bf ON.etol}] ({\it real}): 
\index{ON.etol@{\bf ON.etol}} 
Relative-energy tolerance in the conjugate minimization of the electronic
energy. The minimization finishes if 
\hspace{0.2truecm} $2 (E_n - E_{n-1}) / (E_n + E_{n-1}) \leq $ ON.etol.

{\it Use:} Used only if {\bf SolutionMethod} = {\tt OrderN}

{\it Default value:} ${\tt 10^{-8}}$

\item[{\bf ON.eta}] ({\it real energy}): 
\index{ON.eta@{\bf ON.eta}} 
Fermi level parameter of Kim 
{\it et al.}. This should be in the energy gap, and tuned to obtain
the correct number of electrons. If the calculation is spin polarised,
then separate Fermi levels for each spin can be specified.

{\it Use:} Used only if {\bf SolutionMethod} = {\tt OrderN}

{\it Default value:} {\tt 0.0 eV}

\item[{\bf ON.eta\_alpha}] ({\it real energy}): 
\index{ON.eta\_alpha@{\bf ON.eta\_alpha}} 
Fermi level parameter of Kim {\it et al.} for alpha spin electrons. 
This should be in the energy gap, and tuned to obtain
the correct number of electrons. Note that if the Fermi
level is not specified individually for each spin then the
same global eta will be used.

{\it Use:} Used only if {\bf SolutionMethod} = {\tt OrderN}

{\it Default value:} {\tt 0.0 eV}

\item[{\bf ON.eta\_beta}] ({\it real energy}): 
\index{ON.eta\_beta@{\bf ON.eta\_beta}} 
Fermi level parameter of Kim {\it et al.} for beta spin electrons. 
This should be in the energy gap, and tuned to obtain
the correct number of electrons. Note that if the Fermi
level is not specified individually for each spin then the
same global eta will be used.

{\it Use:} Used only if {\bf SolutionMethod} = {\tt OrderN}

{\it Default value:} {\tt 0.0 eV}

\item[{\bf ON.RcLWF}] ({\it real legth}): 
\index{ON.RcLWF@{\bf ON.RcLWF}}\index{Localized Wave Functions} 
Localization redius for the Localized Wave Functions (LWF's).

{\it Use:} Used only if  {\bf SolutionMethod} = {\tt OrderN}

{\it Default value:} {\tt 9.5 Bohr}
        
\item[{\bf ON.ChemicalPotential}] ({\it logical}): 
\index{ON.ChemicalPotential@{\bf ON.ChemicalPotential}}
\index{Chemical Potential} 
Specifies whether to calculate an order-{\it N} estimate of the
Chemical Potential, by the projetion method 
(Goedecker and Teter, PRB {\bf 51}, 9455 (1995); 
Stephan, Drabold and Martin, PRB {\bf 58}, 13472 
(1998)). This is
done by expanding the Fermi function (or density matrix)
at a given temperature, by means of Chebishev
polynomials\index{Chebishev Polynomials}, and imposing a
real space truncation on the density matrix.
To obtain a realistic estimate, the temperature 
should be small enough (typically, smaller than
the energy gap), the localization range large enough
(of the order of the one you would use for the Localized Wannier
Functions), and the order of the polynomial expansion
sufficiently large (how large depends on the temperature; 
typically, 50-100).

{\it Use:} Used only if {\bf SolutionMethod} = {\tt OrderN}

{\it Default value:} {\tt .false.}

\item[{\bf ON.ChemicalPotentialUse}] ({\it logical}): 
\index{ON.ChemicalPotentialUse@{\bf ON.ChemicalPotentialUse}}
\index{Chemical Potential} 
Specifies whether to use the calculated estimate of the
Chemical Potential, instead of the parameter 
{\bf ON.eta}\index{ON.eta@{\bf ON.eta}} 
for the order-{\it N} energy functional minimization.
This is useful if you do not know the position
of the Fermi level, typically in the beginning
of an order-{\em N} run.

{\it Use:} Used only if {\bf SolutionMethod} = {\tt OrderN}.
Overrides the value of {\bf ON.eta}.
Overrides the value of {\bf ON.ChemicalPotential}, setting
it to {\tt .true.}.

{\it Default value:} {\tt .false.}

\item[{\bf ON.ChemicalPotentialRc}]  ({\it real length}):
\index{ON.ChemicalPotentialRc@{\bf ON.ChemicalPotentialRc}}
\index{Chemical Potential} 
Defines the cutoff radius for the density matrix or Fermi
operator in the calculation of the estimate of the
Chemical Potential.

{\it Use:} Used only if {\bf SolutionMethod} = {\tt OrderN}
and {\bf ON.ChemicalPotential} or  {\bf ON.ChemicalPotentialUse} 
= {\tt .true.}

{\it Default value:} {\tt 9.5 Bohr}.

\item[{\bf ON.ChemicalPotentialTemperature}]  ({\it real temperature 
or energy}):
\index{ON.ChemicalPotentialTemperature@{\bf ON.ChemicalPotentialTemperature}}
\index{Chemical Potential} 
Defines the temperature to be used in the Fermi function expansion
in the calculation of the estimate of the Chemical Potential.
To have an accurate results, this temperature should be smaller 
than the gap of the system.

{\it Use:} Used only if {\bf SolutionMethod} = {\tt OrderN},
and {\bf ON.ChemicalPotential} or  {\bf ON.ChemicalPotentialUse} = 
{\tt .true.}

{\it Default value:} {\tt 0.05 Ry}.

\item[{\bf ON.ChemicalPotentialOrder}] ({\it integer}):
\index{ON.ChemicalPotentialOrder@{\bf ON.ChemicalPotentialOrder}}
\index{Chemical Potential}
Order of the Chebishev expansion to calculate the estimate
of the Chemical Potential. 

{\it Use:} Used only if {\bf SolutionMethod} = {\tt OrderN},
and {\bf ON.ChemicalPotential} or  {\bf ON.ChemicalPotentialUse} = 
{\tt .true.}

{\it Default value:} {\tt 100}

\item[{\bf ON.LowerMemory}] ({\it logical}): 
\index{ON.LowerMemory@{\bf ON.LowerMemory}}\index{Lower order N memory} 
If .true., then a slightly reduced memory algorithm is used in the 
3-point line search during the order N minimisation. Only affects
parallel runs.

{\it Use:} Used only if  {\bf SolutionMethod} = {\tt OrderN}

{\it Default value:} {\tt .false.}
        
\end{description}



\vspace{5pt}
\subsection{Molecular dynamics and relaxations}

\begin{description}
\itemsep 10pt
\parsep 0pt

\item[{\bf MD.TypeOfRun}] ({\it string}): 
\index{MD.TypeOfRun@{\bf MD.TypeOfRun}} 
Type of Molecular Dynamics (MD)  run. 
Several options for MD and structural optimizations are 
implemented. Note that some options specified in later variables
(like quenching) modify the behavior of these MD options.
If the system contains just one atom, {\tt CG} is the only 
available dynamics option.
\begin{itemize}

\item {\tt CG} (Coordinate optimization by conjugate
  gradients). Optionally (see variable MD.VariableCell below), the
  optimization can include the cell vectors.

\item {\tt Broyden} (Coordinate optimization by a modified Broyden
  scheme). Optionally, (see variable MD.VariableCell below), the
  optimization can include the cell vectors.

\item {\tt Verlet} (Standard Verlet algorithm MD)

\item {\tt Nose}  (MD with temperature controlled  by means of a Nos\'e 
thermostat)

\item {\tt ParrinelloRahman}  (MD with pressure controlled by 
the Parrinello-Rahman method)

\item {\tt NoseParrinelloRahman}  (MD with temperature controlled
by means of a Nos\'e thermostat and pressure controlled by 
the Parrinello-Rahman method)

\item {\tt Anneal}  (MD with annealing to a desired
temperature and/or pressure (see variable MD.AnnealOption below)

\item {\tt FC}  (Compute force constants matrix\index{Force Constants
Matrix} for phonon
calculations. The output can be analyzed to extract phonon
frequencies and vectors with the VIBRA\index{VIBRA} package in the Util/
directory. For computing the Born effective charges together with the
force constants, see {\bf BornCharge}
\index{BornCharge@{\bf BornCharge}})

\item {\tt Phonon} (Compute forces for a specified set of atomic
displacements. These are choosen with the help of the program {\sc
Phonon} \footnote{{\sc Phonon} is \copyright\ copyright by Krzysztof
Parlinski}  \index{Phonon program} \index{Force Constants
Matrix!using phonon@using {\sc Phonon}} for phonon calculations). See
also {\tt MD.ATforPhonon} block. (Deprecated feature which might be
removed in future versions.)

\item {\tt Forces} (Receive coordinates from, and return forces to,
an external driver program, using Unix pipes for communication.
The routines in module Util/SiestaSubroutine/fsiesta.f90 allow the
user's program to perform this communication transparently, as if
siesta were a conventional force-field subroutine. See file README
in that directory for details. WARNING: if this option is specified
without a driver program sending data, siesta may hang without
any notice).
\end{itemize}
    
{\it Default value:} {\tt Verlet} ({\tt CG} for one-atom systems)


\item[{\bf MD.VariableCell}] ({\it logical}):
\index{MD.VariableCell@{\bf MD.VariableCell}} \index{cell relaxation}
If true, the lattice is relaxed together with the atomic coordinates
in the conjugate gradient (or Broyden) minimization. It allows to target
hydrostatic pressures or arbitrary stress tensors.  See {\bf
MD.MaxStressTol}, {\bf MD.TargetPressure}, {\bf MD.TargetStress}, {\bf
  MD.ConstantVolume}, {\bf MD.PreconditionVariableCell}, and {\bf Optim.Broyden}.

{\it Use:} Used only if MD.TypeOfRun is {\tt CG} or {\tt Broyden}

{\it Default value:} {\tt .false.}

\item[{\bf MD.ConstantVolume}] ({\it logical}):
\index{MD.ConstantVolume@{\bf MD.ConstantVolume}}
\index{constant-volume cell relaxation}
If true, the cell volume is kept constant in a variable-cell
relaxation: only the cell shape and the atomic coordinates are allowed
to change in the conjugate gradient (or Broyden) minimization. 
Note that it does not make much sense to specify a target stress or
pressure in this case, except for anisotropic (traceless) stresses.
See {\bf MD.VariableCell}, {\bf MD.TargetStress}.

{\it Use:} Used only if MD.TypeOfRun is {\tt CG} or {\tt Broyden}  and
 MD.VariableCell is
 {\tt .true.}. 

{\it Default value:} {\tt .false.}


\item[{\bf MD.NumCGsteps}] ({\it integer}): 
\index{MD.NumCGsteps@{\bf MD.NumCGsteps}} 
Maximum number of conjugate gradient (or Broyden) minimization
moves (the minimization will stop
if tolerance is reached before; see MD.MaxForceTol below).

{\it Use:} Used only if MD.TypeOfRun is {\tt CG} or {\tt Broyden}
    
{\it Default value:} {\tt 0}

\item[{\bf MD.MaxCGDispl}] ({\it real length}): 
\index{MD.MaxCGDispl@{\bf MD.MaxCGDispl}} 
Maximum atomic displacements on a {\tt CG} optimization move.

{\it Use:} Used only if MD.TypeOfRun is {\tt CG}. (For the Broyden
optimization method, it is only possible to limit indirectly the {\it initial\/}
atomic displacements using {\bf MD.Broyden.Initial.Inverse.Jacobian}.)
    
{\it Default value:} {\tt 0.2 Bohr}


\item[{\bf MD.PreconditionVariableCell}] ({\it real length}):
\index{MD.PreconditionVariableCell@{\bf MD.PreconditionVariableCell}}
A length to multiply to the strain components in a variable-cell conjugate
gradient minimization. The strain components enter the minimization
on the same footing as the coordinates. For a good CG efficiency, 
this length should make
the scale of energy variation with strain similar to the one 
due to atomic displacements. It is also
used for the application of the {\bf MD.MaxCGDispl} value to
the strain components.
 
{\it Use:} Used only if MD.TypeOfRun is {\tt CG} or {\tt Broyden} and
Md.VariableCell is {\tt .true.}
   
{\it Default value:} {\tt 5.0 Ang}
 
 
\item[{\bf MD.MaxForceTol}] ({\it real force}): 
\index{MD.MaxForceTol@{\bf MD.MaxForceTol}} 
Force tolerance in CG coordinate optimization.  
Run stops if the maximum atomic force is
smaller than {\bf MD.MaxForceTol} (see {\bf MD.MaxStressTol}
for variable cell).

{\it Use:} Used only if MD.TypeOfRun is {\tt CG} or {\tt Broyden}
    
{\it Default value:} {\tt 0.04 eV/Ang}


\item[{\bf MD.MaxStressTol}] ({\it real pressure}):
\index{MD.MaxStressTol@{\bf MD.MaxStressTol}}
Stress tolerance in variable-cell CG optimization. Run stops
if the maximum atomic force is smaller than {\bf MD.MaxForceTol}
and the maximum stress component is smaller than {\bf MD.MaxStressTol}.

{\it Use:} Used only if MD.TypeOfRun is {\tt CG} or {\tt Broyden} and
Md.VariableCell is {\tt .true.}

Special consideration is needed if used with Sankey-type basis sets, since 
the combination of orbital kinks at the cutoff radii and the finite-grid
integration originate discontinuities in the
stress components, whose magnitude depends on the cutoff radii (or 
energy shift) and the mesh cutoff. The tolerance has to be larger
than the discontinuities to avoid endless optimizations if the target
stress happens to be in a discontinuity.
   
{\it Default value:} {\tt 1.0 GPa}
 

\item [{\bf MD.Broyden.History.Steps}] ({\it integer}):
\index{MD.Broyden.History.Steps@{\bf MD.Broyden.History.Steps}}
\index{Broyden optimization} 

Number of relaxation steps during which the modified Broyden algorithm builds up
the Jacobian matrix. (See D.D. Johnson, PRB 38, 12807
(1988)).

{\it Use:} Used only if MD.TypeOfRun is {\tt Broyden}. 

{\it Default value:} {\tt 5}

\item [{\bf MD.Broyden.Cycle.On.Maxit}] ({\it logical}):
\index{MD.Broyden.Cycle.On.Maxit@{\bf MD.Broyden.Cycle.On.Maxit}}

Upon reaching the maximum number of history data sets which are kept
for Jacobian estimation, throw away the oldest and shift the rest to
make room for a new data set. The alternative is to re-start the
Broyden minimization algorithm from a first step of a diagonal inverse
Jacobian (which might be useful when the minimization is
stuck). 

{\it Use:} Used only if MD.TypeOfRun is {\tt Broyden}. 

{\it Default value:} {\tt .true.}

\item[{\bf MD.Broyden.Initial.Inverse.Jacobian}] ({\it real}):
\index{MD.Broyden.Initial.Inverse.Jacobian@
{\bf MD.Broyden.Initial.Inverse.Jacobian}}

Initial inverse Jacobian for the optimization procedure. (The units 
are those implied by the internal Siesta usage (Bohr for lenghts and
Ry for energies). The default value seems to work well for most systems.

{\it Use:} Used only if MD.TypeOfRun is {\tt Broyden}. 

{\it Default value:} {\tt 1.0}

   
\item[{\bf MD.InitialTimeStep}] ({\it integer}): 
\index{MD.InitialTimeStep@{\bf MD.InitialTimeStep}} 
Initial time step of the MD simulation.
In the current version of {\sc Siesta} it must be 1.

{\it Use:} Used only if MD.TypeOfRun is not {\tt CG} or {\tt Broyden}
    
{\it Default value:} {\tt 1}

\item[{\bf MD.FinalTimeStep}] ({\it integer}): 
\index{MD.FinalTimeStep@{\bf MD.FinalTimeStep}} 
Final time step of the MD simulation.

{\it Use:} Used only if MD.TypeOfRun is not {\tt CG} or {\tt Broyden}
    
{\it Default value:} {\tt 1}

\item[{\bf MD.LengthTimeStep}] ({\it real time}): 
\index{MD.LengthTimeStep@{\bf MD.LengthTimeStep}} 
Length of the time step of the MD simulation.

{\it Use:} Used only if MD.TypeOfRun is not {\tt CG} or {\tt Broyden}
    
{\it Default value:} {\tt 1.0 fs}

\item[{\bf MD.InitialTemperature}] ({\it real temperature or energy}): 
\index{MD.InitialTemperature@{\bf MD.InitialTemperature}} 
Initial temperature for MD run. The atoms are assigned random 
velocities drawn from the Maxwell-Bolzmann distribution with the
corresponding temperature. The constraint of zero center of
mass velocity is imposed.

{\it Use:} Used only if {\bf MD.TypeOfRun} = {\tt Verlet, Nose, 
ParrinelloRahman, NoseParrinelloRahman}
or {\tt Anneal}.

{\it Default value:} {\tt 0.0 K}

\item[{\bf MD.Quench}] ({\it logical}): 
\index{MD.Quench@{\bf MD.Quench}} 
Logical option to perform a power quench during the molecular dynamics. 
In the power quench, each velocity component is set to
zero if it is opposite to the corresponding force
of that component. This affects atomic velocities,
or unit-cell velocities (for cell shape optimizations).

{\it Use:} Used only if {\bf MD.TypeOfRun} = {\tt Verlet} or 
{\tt ParrinelloRahman}.
It is incompatible with Nose thermostat options.
The quench option allows structural relaxations of
only atomic coordinates (with {\bf MD.TypeOfRun} = {\tt Verlet})
or atomic coordinates AND cell shape 
(with {\bf MD.TypeOfRun} = {\tt ParrinelloRahman}).
{\bf MD.Quench} is superseded by {\bf MD.FireQuench} (see below).

{\it Default value:} {\tt .false.}

\item[{\bf MD.FireQuench}] ({\it logical}): 
\index{MD.FireQuench@{\bf MD.FireQuench}} 
Logical option to perform a FIRE quench during a Verlet molecular dynamics
run, as described by Bitzek {\it et al.} in Phys. Rev. Lett. {\bf 97},
170201 (2006). It is a relaxation algorithm, and thus the dynamics
are of no interest per se: the initial time-step can be played with
(it uses {\bf MD.LengthTimeStep} as initial $\Delta t$),
as well as the initial temperature (recommended 0) and the atomic
masses (recommended equal). Preliminary tests seem to indicate that 
the combination of $\Delta t = 5$ fs and a value of 20 for the atomic 
masses works reasonably. The dynamics stops when the force
tolerance is reached ({\bf MD.MaxForceTol}). The other
parameters controlling the algorithm (initial damping, 
increase and decrease thereof etc.) are hardwired in the code,
at the recommended values in the cited paper,
including $\Delta t_{max} = 10$ fs.

{\it Use:} Used only if {\bf MD.TypeOfRun} = {\tt Verlet}. 
It is incompatible with Nose thermostat options. No variable
cell option implemented for this at this stage.
{\bf MD.FireQuench} supersedes {\bf MD.Quench}.

{\it Default value:} {\tt .false.}

\item[{\bf MD.TargetTemperature}] ({\it real temperature or energy}): 
\index{MD.TargetTemperature@{\bf MD.TargetTemperature}} 
Target temperature for Nose thermostat and annealing options.

{\it Use:} Used only if {\bf MD.TypeOfRun} = {\tt Nose, NoseParrinelloRahman}
or {\tt Anneal} (if {\bf MD.AnnealOption} = {\tt Temperature} or 
{\tt TemperatureandPressure})

{\it Default value:} {\tt 0.0 K}

\item[{\bf MD.TargetPressure}] ({\it real pressure}): 
\index{MD.TargetPressure@{\bf MD.TargetPressure}} 
Target pressure for Parrinello-Rahman method, variable cell CG optimizations,
and annealing options.

{\it Use:} Used only if MD.TypeOfRun = 
{\tt ParrinelloRahman}, {\tt NoseParrinelloRahman},
{\tt CG} (variable cell), or {\tt Anneal} 
(if MD.AnnealOption = {\tt Pressure} or {\tt TemperatureandPressure})

{\it Default value:} {\tt 0.0 GPa}


\item[{\bf MD.TargetStress}] ({\it data block}):
External or target stress tensor for variable cell optimizations.
Stress components are given in a line, in the order {\tt
xx, yy, zz, xy, xz, yz}. In units of {\bf MD.TargetPressure},
but with the opposite sign. For example, a uniaxial compressive stress 
of 2 GPa along the 100 direction would be given by
\begin{verbatim}
       MD.TargetPressure  2. GPa
       %block MD.TargetStress
           -1.0  0.0  0.0  0.0  0.0  0.0
       %endblock MD.TargetStress
\end{verbatim}

{\it Use:} Used only if MD.TypeOfRun is {\tt CG} and 
MD.VariableCell is {\tt .true.} 
 
{\it Default value:} Hydrostatic target pressure: 
{\tt -1., -1., -1., 0., 0., 0.}



\item[{\bf MD.NoseMass}] ({\it real moment of inertia}): 
\index{MD.NoseMass@{\bf MD.NoseMass}} 
Generalized mass of Nose variable.
This determines the time scale of the Nose variable
dynamics, and the coupling of the thermal bath to
the physical system.

{\it Use:} Used only if {\bf MD.TypeOfRun} = {\tt Nose} or 
{\tt NoseParrinelloRahman}

{\it Default value:} {\tt 100.0 Ry*fs**2}

\item[{\bf MD.ParrinelloRahmanMass}] ({\it real moment of inertia}): 
\index{MD.ParrinelloRahmanMass@{\bf MD.ParrinelloRahmanMass}} 
Generalized mass of Parrinello-Rahman variable.
This determines the time scale 
of the Parrinello-Rahman variable
dynamics, and its coupling to
the physical system.

{\it Use:} Used only if {\bf MD.TypeOfRun} = {\tt ParrinelloRahman} 
or {\tt NoseParrinelloRahman}

{\it Default value:} {\tt 100.0 Ry*fs**2}

\item[{\bf MD.AnnealOption}] ({\it string}): 
\index{MD.AnnealOption@{\bf MD.AnnealOption}} 
Type of annealing MD to perform. The target temperature or pressure are
achieved by velocity and unit cell rescaling, 
in a given time determined by the variable
{\bf MD.TauRelax} below.
\begin{itemize}
\item {\tt Temperature} (Reach a target temperature by velocity rescaling)
\item {\tt Pressure} (Reach a target pressure by scaling of the unit
cell size and shape)
\item {\tt TemperatureandPressure}  (Reach a target temperature 
and pressure by velocity rescaling and by scaling of the unit
cell size and shape)
\end{itemize}

{\it Use:} Used only if {\bf MD.TypeOfRun} = {\tt Anneal}

{\it Default value:} {\tt TemperatureAndPressure}

\item[{\bf MD.TauRelax}] ({\it real time}): 
\index{MD.TauRelax@{\bf MD.TauRelax}} 
Relaxation time to reach target temperature
and/or pressure in annealing MD. Note that this is a ``relaxation
time'', and as such it gives a rough estimate of the time needed to
achieve the given targets. As a normal simulation also exhibits
oscillations, the actual time needed to reach the {\it averaged}
targets will be significantly longer.

{\it Use:} Used only if {\bf MD.TypeOfRun} = {\tt Anneal}

{\it Default value:} {\tt 100.0 fs}

\item[{\bf MD.BulkModulus}] ({\it real pressure}): 
\index{MD.BulkModulus@{\bf MD.BulkModulus}} 
Estimate (may be rough) of the bulk modulus of the system.
This is needed to set the rate of change of cell shape
to reach target pressure in annealing MD.

{\it Use:} Used only if {\bf MD.TypeOfRun} = {\tt Anneal}, when
{\bf MD.AnnealOption} = {\tt Pressure} or {\tt TemperatureAndPressure}

{\it Default value:}  {\tt 100.0 Ry/Bohr**3}
        
\item[{\bf MD.FCDispl}] ({\it real length}): 
\index{MD.FCDispl@{\bf MD.FCDispl}} 
Displacement to use for the computation of the force constant
matrix\index{Force Constants Matrix} for phonon calculations.

{\it Use:} Used only if {\bf MD.TypeOfRun} = {\tt FC}.

{\it Default value:}  {\tt 0.04 Bohr}

\item[{\bf MD.FCfirst}] ({\it integer}): 
\index{MD.FCfirst@{\bf MD.FCfirst}} 
Index of first atom to displace for the computation of the force constant
matrix\index{Force Constants Matrix} for phonon calculations.

{\it Use:} Used only if {\bf MD.TypeOfRun} = {\tt FC}.

{\it Default value:}  {\tt 1}

\item[{\bf MD.FClast}] ({\it integer}): 
\index{MD.FClast@{\bf MD.FClast}} 
Index of last atom to displace for the computation of the force constant
matrix\index{Force Constants Matrix} for phonon calculations.

{\it Use:} Used only if {\bf MD.TypeOfRun} = {\tt FC}.

{\it Default value:}  Same as {\bf NumberOfAtoms}

\item[{\bf MD.ATforPhonon}] ({\it data block}): List of ``symmetry
irreducible'' atomic displacements for which to compute forces. Each
line gives the fractional displacement for an atom, identified by its
number in the atom list, and by a one-character code generated by the
{\sc Phonon} program. These codes are put in correspondence with the
species labels in block \hbox{\it PhononLabels}).\index{Force
Constants Matrix!using phonon@using {\sc Phonon}}

\begin{verbatim}
    %block MD.ATforPhonon
      0.002358   0.000000   0.000000  L     1
      0.000000   0.000000   0.003488  L     1
      0.002358   0.000000   0.000000  A    33
      0.000000   0.000000   0.003488  A    33
     -0.002358   0.000000   0.000000  L     1
      0.000000   0.000000  -0.003488  L     1
     -0.002358   0.000000   0.000000  A    33
      0.000000   0.000000  -0.003488  A    33
    %endblock MD.ATforPhonon
  
\end{verbatim}

{\it Note:} The presence of this block 
atomatically sets MD.TypeOfRun to {\tt Phonon}.
 
{\it Default value:} None.

\end{description}


\vspace{5pt}
\subsection{Parallel options}

(Note: These features are not available in all distributions.)

\begin{description}
\itemsep 10pt
\parsep 0pt

\item[{\bf BlockSize}] ({\it integer}): \index{BlockSize@{\bf
BlockSize}} The orbitals are distributed over the processors when
running in parallel using a 1-D block-cyclic algorithm. {\bf
BlockSize} is the number of consecutive orbitals which are located on
a given processor before moving to the next one. Large values of this
parameter lead to poor load balancing, while small values can lead to
inefficient execution.  The performance of the parallel code can be
optimised by varying this parameter until a suitable value is found.

{\it Use:} Controls the blocksize used for distributing orbitals over
processors

{\it Default value:}  8

\item[{\bf ProcessorY}] ({\it integer}): \index{ProcessorY@{\bf
ProcessorY}} The mesh points are divided in the Y and Z directions
over the processors in a 2-D grid. {\bf ProcessorY} specifies the
dimension of the processor grid in the Y-direction and must be a
factor of the total number of processors. Ideally the processors
should be divided so that the number of mesh points per processor
along each axis is as similar as possible.

{\it Use:} Controls the dimensions of the 2-D processor grid for mesh
distribution

{\it Default value:} Variable - chosen using multiples of factors of
  the total number of processors

\item[{\bf Diag.Memory}] ({\it real no units}):
\index{Diag.Memory@{\bf Diag.Memory}}
Whether the parallel diagonalisation of a matrix is successful or not can 
depend on how much workspace is available to the routine when there are
clusters of eigenvalues. {\bf Diag.Memory} allows the user to increase
the memory available, when necessary, to achieve successful diagonalisation
and is a scale factor relative to the minimum amount of memory that
SCALAPACK might need. 

{\it Use:} Controls the amount of workspace available to parallel
matrix diagonalisation

{\it Default value:}  1.0

\item[{\bf Diag.ParallelOverK}] ({\it logical}): \index{Diag.ParallelOverK@{\bf
Diag.ParallelOverK}} For the diagonalisation there is a choice in strategy
about whether to parallelise over the K points or over the orbitals. K
point diagonalisation is close to perfectly parallel but is only
useful where the number of K points is much larger than the number of
processors and therefore orbital parallelisation is generally
preferred. The exception is for metals where the unit cell is small,
but the number of K points to be sampled is very large. In this last
case it is recommend that this option be used.

{\it Use:} Controls whether the diagonalisation is parallelised with
respect to orbitals or K points - not allowed for non-co-linear spin
case.

{\it Default value:}  false

\item[{\bf RcSpatial}] ({\it real distance}):
\index{RcSpatial@{\bf RcSpatial}}
When performing a parallel order N calculation, a domain/spatial
decomposition algorithm is used in which the system is divided into
cells, which are then assigned to the nodes. The size of the cells
is, by default, equal to the maximum distance at which there is a non-zero 
matrix element in the Hamiltonian between two orbitals, or the
radius of the Wannier function - which ever is the larger. If this is 
the case, then an orbital will only interact with other orbitals in the same
or neighbouring cells. However, by decreasing the cell size and searching
over more cells it is possible to achieve better load balance in some
cases.

{\it Use:} Controls the domain size during the spatial decomposition

{\it Default value:}  maximum of the matrix element range or the Wannier radius

\end{description}


\vspace{5pt}
\subsection{Efficiency options}

\begin{description}
\itemsep 10pt
\parsep 0pt

\item[{\bf DirectPhi}] ({\it logical}):
\index{DirectPhi@{\bf DirectPhi}}

In the calculation of the matrix elements on the mesh this requires the
value of the orbitals on the mesh points. This array represents one of
the largest uses of memory within the code. If set to true this option
allows the code to generate the orbital values when needed rather than
storing the values. This obviously costs more computer time but will
make it possible to run larger jobs where memory is the limiting factor.,

{\it Use:} Controls whether the values of the orbitals at the mesh points
  are stored or calculated on the fly.

{\it Default value:}  false

\item[{\bf SaveMemory}] ({\it logical}):
\index{SaveMemory@{\bf SaveMemory}}

When calculating values that are stored in arrays whose dimensions
cannot be accurately predicted ahead of time, there are two choices
as to what to do when an array bound is exceeded. Firstly, the
contents can be copied into buffer arrays while the dimensions
are increased and then copied back (which is the default for
expensive operations) or secondly, the arrays can be re-initialised
and filled from scratch after re-dimensioning. The first approach
is the fastest but requires larger amounts of memory, particular
in {\bf dhscf}, whereas the second uses the minimum memory at the
expense of re-calculating a number of quantities.

{\it Use:} Controls whether the program uses algorithms which save memory
  at the expense of CPU time by not preserving the contents of arrays
  when re-initialising the dimensions due to bounds being exceeded. 

{\it Default value:}  false

\end{description}


\vspace{5pt}
\subsection{Output options}

\begin{description}
\itemsep 10pt
\parsep 0pt

\item[{\bf LongOutput}] ({\it logical}):
\index{LongOutput@{\bf LongOutput}}\index{output!long}
{\sc Siesta} can write to standard output different data sets
depending on the values for output options described below.
By default {\sc Siesta} will not write most of them. They can be
large for large systems (coordinates, eigenvalues, forces, etc.)
and, if written to standard output, they accumulate for all the steps of 
the dynamics. {\sc Siesta} writes the information in other files
(see Output Files) in addition to the standard output, and these
can be accumulative or not.

Setting {\bf LongOutput} to {\tt .true.} changes the default of
some options, obtaining more information in the output (verbose).
In particular, it redefines the defaults for the following:

\begin{itemize}

\item
{\bf WriteCoorStep}\index{WriteCoorStep@{\bf WriteCoorStep}}:
\index{output!atomic coordinates!in a dynamics step} {\tt .true.}
\item
{\bf WriteForces}\index{WriteForces@{\bf WriteForces}}:
\index{output!forces} {\tt .true.}
\item
{\bf WriteKpoints}\index{WriteKpoints@{\bf WriteKpoints}}:
\index{output!grid $\vec k$ points} {\tt .true.}
\item
{\bf WriteEigenvalues}\index{WriteEigenvalues@{\bf WriteEigenvalues}}:
\index{output!eigenvalues} {\tt .true.}
\item
{\bf WriteKbands}\index{WriteKbands@{\bf WriteKbands}}:
\index{output!band $\vec k$ points} {\tt .true.}
\item
{\bf WriteBands}\index{WriteBands@{\bf WriteBands}}:
\index{output!band structure} {\tt .true.}
\item
{\bf WriteWaveFunctions}
\index{WriteWaveFunctions@{\bf WriteWaveFunctions}}:
\index{output!wave functions} {\tt .true.}
\item
{\bf WriteMullikenPop}\index{WriteMullikenPop@{\bf WriteMullikenPop}} 
\index{output!Mulliken analysis}\index{Mulliken population analysis} 1

\end{itemize}

The specific changing of any of these options overrides the
{\bf LongOutput} setting for it.

{\it Default value:} {\tt .false.}


\item[{\bf WriteCoorInitial}] ({\it logical}):
\index{WriteCoorInitial@{\bf WriteCoorInitial}}
\index{output!atomic coordinates!initial}
It determines whether the initial atomic coordinates of the simulation are
dumped into the main output file. These coordinates correspond to the
ones actually used in the first step, i.e., after reading (if pertinent)
the {\it Systemlabel}.XV file. It is not affected by the {\bf LongOutput}
flag.

{\it Default value:} {\tt .true.}

 
\item[{\bf WriteCoorStep}] ({\it logical}):
\index{WriteCoorStep@{\bf WriteCoorStep}}
\index{output!atomic coordinates!in a dynamics step}
If {\tt .true.} it writes the atomic coordinates at every 
time or relaxation step. Otherwise it does not. They are
always written in the {\it Systemlabel}.XV file, but
overriden at every step. They can be also accumulated
in the {\it Systemlabel}.MD or {\it Systemlabel}.MDX files
depending on {\bf WriteMDhistory}. Unless the contrary is specified
(see {\bf WriteMDXmol}), if {\bf WriteCoorStep} is {\tt .false.}, 
the coordinates are accumulated in {\sc XMol} {\tt xyz} format in the
{\it Systemlabel}.ANI file.
\index{WriteMDhistory@{\bf WriteMDhistory}}
\index{output!atomic coordinates!history}
For using the {\sc Sies2arc} utility\index{Sies2arc@{\sc Sies2arc}}
for generating a CERIUS .arc animation file,
WriteCoorStep should be {\tt .true.}

{\it Default value:} {\tt .false.} (see {\bf LongOutput})
 
 
\item[{\bf WriteForces}] ({\it logical}):
\index{WriteForces@{\bf WriteForces}}\index{output!forces}
If {\tt .true.} it writes the atomic forces at every
time or relaxation step. Otherwise it does not. In this case,
the forces of the last step can be found in the file {\it Systemlabel}.FA .
 
{\it Default value:} {\tt .false.} (see {\bf LongOutput})

\item[{\bf WriteKpoints}] ({\it logical}):
\index{WriteKpoints@{\bf WriteKpoints}}\index{output!grid $\vec k$ points}
If {\tt .true.} it writes the coordinates of the $\vec k$ vectors
used in the grid for $k$-sampling, into the main output file.
Otherwise, it does not.

{\it Default value:} {\tt .false.} (see {\bf LongOutput})

 
\item[{\bf WriteEigenvalues}] ({\it logical}):
\index{WriteEigenvalues@{\bf WriteEigenvalues}}\index{output!eigenvalues}
If {\tt .true.} it writes the Hamiltonian eigenvalues for the sampling
$\vec k$ points, in the main output file.
Otherwise it does not, but writes them in the file {\it Systemlabel}.EIG
to be used by the {\sc Eig2dos}\index{Eig2dos@{\sc Eig2dos}} postprocessing 
utility (in the Util/ directory) for obtaining the density of
states.\index{density of states}

{\it Use:} Only if {\bf SolutionMethod} is {\tt diagon}.
 
{\it Default value:} {\tt .false.} (see {\bf LongOutput})

\item[{\bf WriteDM}] ({\it logical}): \index{WriteDM@{\bf WriteDM}}
\index{output!density matrix} It determines whether the density matrix
is output as a {\it Systemlabel}.DM file or not. For large systems
this file can be quite big and therefore it may be necessary to turn
this option off to conserve disk space.

{\it Default value:} {\tt .true.}

\item[{\bf WriteKbands}] ({\it logical}):
\index{WriteKbands@{\bf WriteKbands}}\index{output!band $\vec k$ points} 
If {\tt .true.} it writes the coordinates of the $\vec k$ vectors
defined for band plotting, into the main output file.
Otherwise, it does not.

{\it Use:} Only if {\bf SolutionMethod} is {\tt diagon}.

{\it Default value:} {\tt .false.} (see {\bf LongOutput})


\item[{\bf WriteBands}] ({\it logical}): \index{WriteBands@{\bf
WriteBands}}\index{output!band structure} If {\tt .true.} it writes
the Hamiltonian eigenvalues corresponding to the $\vec k$ vectors
defined for band plotting, in the main output file.  Otherwise it does
not. They are, however, dumped into the file {\it Systemlabel}.bands
to be used by postprocessing utilities for plotting the band
structure.\index{band structure} The {\sc Gnubands}\index{Gnubands@{\sc 
Gnubands}} program
(found in the Util/ directory) reads the {\it Systemlabel}.bands from
standard input and dumps to standard output a file directly plotable
by {\sc Gnuplot}.\index{Gnuplot@{\sc Gnuplot}}
\footnote{{\sc Gnuplot} is under \copyright\ copyright of GNU software}

{\it Use:} Only if {\bf SolutionMethod} is {\tt diagon}.
 
{\it Default value:} {\tt .false.} (see {\bf LongOutput})

\item[{\bf WriteWaveFunctions}] ({\it logical}): 
\index{WriteWaveFunctions@{\bf WriteWaveFunctions}}
\index{output!wave functions} If {\tt .true.} it writes
the Hamiltonian eigenvectors (coefficients of the wavefunctions
in the basis set orbitals expansion) corresponding to the $\vec k$ vectors
defined by the {\bf WaveFuncKPoints}
\index{WaveFuncKPoints@{\bf WaveFuncKPoints}}
descriptor to the main output file.  Otherwise it does
not, but they are dumped into the file {\it Systemlabel}.WFS
to be used by postprocessing utilities for handling the 
wave functions
\index{wave functions}. The READWF\index{READWF} program
(found in the Util/ directory) reads the {\it Systemlabel}.WFS from
standard input and dumps to standard output a file readable
by the user.

{\it Use:} Only if {\bf SolutionMethod} is {\tt diagon}.
 
{\it Default value:} {\tt .false.} (see {\bf LongOutput})


\item[{\bf WriteMullikenPop}] ({\it integer}): 
\index{WriteMullikenPop@{\bf WriteMullikenPop}}\index{Mulliken population
analysis}\index{output!Mulliken analysis} 
It determines the level of Mulliken population analysis printed:
\begin{itemize}
\item 0 = None
\item 1 = atomic and orbital charges
\item 2 = 1 + atomic overlap pop.
\item 3 = 2 + orbital overlap pop.
\end{itemize}
The order of the orbitals in the population lists is defined
by the order of atoms. For each atom, populations for PAO orbitals and
double-$z$, triple-$z$, etc... derived from them are displayed first for 
all the angular momenta. Then, populations for perturbative polarization
orbitals are written.
Within a $l$-shell be aware that the order is not
conventional, being $y$, $z$, $x$ for $p$ orbitals, and
$xy$, $yz$, $z^2$, $xz$, and $x^2-y^2$ for $d$ orbitals. 

{\it Default value:} {\tt 0} (see {\bf LongOutput})


\item[{\bf WriteCoorXmol}] ({\it logical}): 
\index{WriteCoorXmol@{\bf WriteCoorXmol}}\index{XMol@{\sc XMol}}
\index{JMol@{\sc JMol}}
If {\tt .true.} it originates the writing of an extra file
named {\it SystemLabel}{\tt .xyz} containing the final atomic
coordinates in a format directly readable by {\sc XMol}.\footnote{XMol
is under \copyright\ copyright of Research Equipment Inc., dba Minnesota
Supercomputer Center Inc.} Coordinates come out in {\AA}ngstr\"om
independently of what specified in {\bf AtomicCoordinatesFormat} and
in {\bf AtomCoorFormatOut}. There is a present {\sc Java} implementation 
of {\sc XMol} called {\sc JMol}.

{\it Default value:} {\tt .false.}
        

\item[{\bf WriteCoorCerius}] ({\it logical}): 
\index{WriteCoorCerius@{\bf WriteCoorCerius}}\index{Cerius2@{\sc Cerius2}}
If {\tt .true.} it originates the writing of an extra file
named {\it SystemLabel}{\tt .xtl} containing the final atomic
coordinates in a format directly readable by {\sc Cerius}.\footnote{{\sc 
Cerius} is under \copyright\ copyright of Molecular Simulations Inc.} 
Coordinates come out in 
{\tt Fractional} format (the same as {\tt ScaledByLatticeVectors})
independently of what specified in {\bf AtomicCoordinatesFormat} and
in {\bf AtomCoorFormatOut}.
If negative coordinates are to be avoided, it has to be 
done from the start by shifting all the coordinates rigidly
to have them positive, by using {\bf AtomicCoordinatesOrigin}.

{\it Default value:} {\tt .false.}

 
\item[{\bf WriteMDXmol}] ({\it logical}):
\index{WriteMDXmol@{\bf WriteMDXmol}}\index{XMol@{\sc XMol}}
If {\tt .true.} it originates the writing of an extra file
named {\it SystemLabel}{\tt .ANI} containing all the atomic
coordinates of the simulation in a format directly readable by 
{\sc XMol} for animation.\index{animation} Coordinates come out in 
{\AA}ngstr\"om independently of what specified in 
{\bf AtomicCoordinatesFormat} and in {\bf AtomCoorFormatOut}.
This file is accumulative even for different runs.
There is the alternative for animation by generating a .arc file for 
CERIUS. It is through the {\sc Sies2arc}\index{Sies2arc@{\sc Sies2arc}} 
postprocessing utility 
in the Util/ directory, and it requires the coordinates to be
accumulated in the output file, i.e., WriteCoorStep = {\tt .true.}
 
{\it Default value:} {\tt .false.} if WriteCoorStep is {\tt .true.}
and vice-versa.

 
\item[{\bf WriteMDhistory}] ({\it logical}):
\index{WriteMDhistory@{\bf WriteMDhistory}}
\index{output!molecular dynamics!history}
If {\tt .true.} {\sc Siesta} accumulates the molecular dynamics
trajectory in the following files:
\begin{itemize}
\item
{\it Systemlabel}.MD : atomic coordinates and velocities (and
lattice vectors and their time derivatives, if the dynamics implies
variable cell). The information is stored unformatted for postprocessing
with utility programs to analyze the MD trajectory. 
\item
{\it Systemlabel}.MDE : shorter description of the run, with energy,
temperature, etc., per time step.
\end{itemize}
These files are accumulative even for different runs.
 
{\it Default value:} {\tt .false.}

\item[{\bf WarningMinimumAtomicDistance}] ({\it physical}): 
\index{WarningMinimumAtomicDistance@{\bf WarningMinimumAtomicDistance}}
Fixes a threshold interatomic distance below which a warning
message is printed.

{\it Default value:} {\tt 1.0 Bohr}

        
\item[{\bf AllocReportLevel}] ({\it integer}): 
\index{AllocReportLevel@{\bf AllocReportLevel}}
Sets the level of the allocation report, printed in file 
{\tt SystemLabel}.alloc:
\begin{itemize}
\item
  level 0 : no report at all (the default)
\item
  level 1 : only total memory peak and where it occurred
\item
  level 2 : detailed report printed only at 
            normal program termination
\item
  level 3 : detailed report printed at every new memory peak
\item
  level 4 : print every individual (re)allocation or deallocation
\end{itemize}

{\it Default value:} {\tt 0}

\end{description}



\vspace{5pt}
\subsection{Options for saving/reading information}
\index{reading saved data}

\begin{description}
\itemsep 10pt
\parsep 0pt


\item[{\bf UseSaveData}] ({\it logical}): 
\index{UseSaveData@{\bf UseSaveData}} 
\index{reading saved data!all}
Instructs to use as much information as possible stored from
previous runs in files {\tt SystemLabel}.XV, {\tt SystemLabel}.DM and
{\tt SystemLabel}.LWF, where SystemLabel is the name associated
to parameter {\tt SystemLabel}.

{\it Use:} If the required files do not exist, warnings are
printed but the program does not stop.

{\it Default value:} {\tt .false.}


\item[{\bf DM.UseSaveDM}] ({\it logical}): 
\index{DM.UseSaveDM@{\bf DM.UseSaveDM}} 
\index{reading saved data!density matrix}
Instructs to read the density matrix stored in file
{\tt SystemLabel}.DM by a previous run.

{\it Use:} If the required file does not exist, a warning is
printed but the program does not stop. Overrides {\bf UseSaveData}.

{\it Default value:} {\tt .false.}


\item[{\bf DM.FormattedFiles}] ({\it logical}): 
\index{DM.FormattedFiles@{\bf DM.FormattedFiles}} 
\index{reading saved data!density matrix}
Instructs to use formatted files for reading and writing
the density matrix. In this case, the files are labelled
{\tt SystemLabel}.DMF.

{\it Use:} This makes for much larger files, and slower i/o. However, the
files are transferable between different computers, which is
not the case normally.

{\it Default value:} {\tt .false.}


\item[{\bf DM.FormattedInput}] ({\it logical}): 
\index{DM.FormattedInput@{\bf DM.FormattedInput}} 
\index{reading saved data!density matrix}
Instructs to use formatted files for reading the density
matrix.

{\it Use:} Overrides the value of {\bf DM.FormattedFiles}.

{\it Default value:} {\tt .false.}


\item[{\bf DM.FormattedOutput}] ({\it logical}): 
\index{DM.FormattedInput@{\bf DM.FormattedOutput}} 
\index{reading saved data!density matrix}
Instructs to use formatted files for writing the density
matrix.

{\it Use:} Overrides the value of {\bf DM.FormattedFiles}.

{\it Default value:} {\tt .false.}


\item[{\bf ON.UseSaveLWF}] ({\it logical}): 
\index{ON.UseSaveLWF@{\bf ON.UseSaveLWF}} 
\index{reading saved data!localized wave functions (order-$N$)}
Instructs to read the localized wave functions stored in file
{\tt SystemLabel}.LWF by a previous run.

{\it Use:} Used only if {\bf SolutionMethod} is {\tt OrderN}.
If the required file does not exist, a warning is
printed but the program does not stop. Overrides {\bf UseSaveData}.

{\it Default value:} {\tt .false.}

 
\item[{\bf MD.UseSaveXV}] ({\it logical}):
\index{MD.UseSaveXV@{\bf MD.UseSaveXV}}
\index{reading saved data!XV}
Instructs to read the atomic positions and velocities stored
in file {\tt SystemLabel}.XV by a previous run.
 
{\it Use:} If the required file does not exist, a warning is
printed but the program does not stop. Overrides {\bf UseSaveData}.
 
{\it Default value:} {\tt .false.}


\item[{\bf MD.UseSaveCG}] ({\it logical}): 
\index{MD.UseSaveCG@{\bf MD.UseSaveCG}} 
\index{reading saved data!CG}
Instructs to read the conjugate-gradient hystory information stored
in file {\tt SystemLabel}.CG by a previous run.

{\it Use:} To get actual continuation of iterrupted CG runs, use
together with {\bf MD.UseSaveXV} = {\tt .true.} with the XV
file generated in the same run as the CG file.
If the required file does not exist, a warning is
printed but the program does not stop. Overrides {\bf UseSaveData}.

{\it Default value:} {\tt .false.}
        

\item[{\bf MD.UseSaveZM}] ({\it logical}): 
\index{MD.UseSaveZM@{\bf MD.UseSaveZM}} 
\index{reading saved data!ZM}
Instructs the program to read the zmatrix information stored
in file {\tt SystemLabel}.ZM by a previous run.

{\it Use:} If the required file does not exist, a warning is
printed but the program does not stop. Overrides {\bf UseSaveData}.

{\it Warning:} Note that the resulting geometry could be clobbered if
an XV file is read after this file. It is up to the user to remove
any XV files.\index{MD.UseSaveXV@{\bf MD.UseSaveXV}}.

{\it Default value:} {\tt .false.}
        

\item[{\bf SaveHS}] ({\it logical}): 
\index{SaveHS@{\bf SaveHS}}\index{output!Hamiltonian \& overlap} 
Instructs to write the hamiltonian and overlap matrices, as well
as other data required to generate bands and density of states,
in file {\tt SystemLabel}.HS. This file can be read by routine IOHS,
which may be used by an application program in later versions.

{\it Use:} File {\tt SystemLabel}.HS is only written, not read, by siesta.

{\it Default value:} {\tt .false.}
        

\item[{\bf SaveRho}] ({\it logical}): 
\index{SaveRho@{\bf SaveRho}}\index{output!charge density} 
Instructs to write the valence pseudocharge density at the
mesh used by DHSCF,
in file {\tt SystemLabel}.RHO. This file can be read by routine IORHO,
which may be used by an application program in later versions.

{\it Use:} File {\tt SystemLabel}.RHO is only written, not read, by siesta.

{\it Default value:} {\tt .false.}
        

\item[{\bf SaveDeltaRho}] ({\it logical}): 
\index{SaveDeltaRho@{\bf SaveDeltaRho}}\index{output!$\delta \rho(\vec r)$} 
Instructs to write $\delta \rho(\vec r) = \rho(\vec r) - \rho_{atm}(\vec r)$,
i.e., the valence pseudocharge density minus the sum of atomic valence
pseudocharge densities. It is done for the mesh points used by DHSCF and it
comes in file {\tt SystemLabel}.DRHO. This file can be read by routine IORHO,
which may be used by an application program in later versions.

{\it Use:} File {\tt SystemLabel}.DRHO is only written, not read, by siesta.

{\it Default value:} {\tt .false.}
        

\item[{\bf SaveElectrostaticPotential}] ({\it logical}): 
\index{SaveElectrostaticPotential@{\bf SaveElectrostaticPotential}}
\index{output!electrostatic potential} 
Instructs to write the total electrostatic potential, defined as the
sum of the hartree potential plus the local pseudopotential, at the
mesh used by DHSCF,
in file {\tt SystemLabel}.VH. This file can be read by routine IORHO,
which may be used by an application program in later versions.

{\it Use:} File {\tt SystemLabel}.VH is only written, not read, by siesta.

{\it Default value:} {\tt .false.}
        

\item[{\bf SaveTotalPotential}] ({\it logical}): 
\index{SaveTotalPotential@{\bf SaveTotalPotential}} 
\index{output!total potential}
Instructs to write the valence total effective local potential
(local pseudopotential + Hartree + Vxc), at the
mesh used by DHSCF,
in file {\tt SystemLabel}.VT. This file can be read by routine IORHO,
which may be used by an application program in later versions.

{\it Use:} File {\tt SystemLabel}.VT is only written, not read, by siesta.

{\it Default value:} {\tt .false.}
        

\item[{\bf SaveIonicCharge}] ({\it logical}): 
\index{SaveIonicCharge@{\bf SaveIonicCharge}} 
\index{output!ionic charge}
Instructs to write the soft diffuse ionic charge at the
mesh used by DHSCF,
in file {\tt SystemLabel}.IOCH. This file can be read by routine IORHO,
which may be used by an application program in later versions.
Remember that, within the {\sc Siesta} sign convention, the electron charge
density is positive and the ionic charge density is negative. 


{\it Use:} File {\tt SystemLabel}.IOCH is only written, not read, by siesta.

{\it Default value:} {\tt .false.}
        
\item[{\bf SaveTotalCharge}] ({\it logical}): 
\index{SaveTotalCharge@{\bf SaveTotalCharge}} 
\index{output!total charge}
Instructs to write the total charge density (ionic+electronic) at the
mesh used by DHSCF,
in file {\tt SystemLabel}.TOCH. This file can be read by routine IORHO,
which may be used by an application program in later versions.
Remember that, within the {\sc Siesta} sign convention, the electron charge
density is positive and the ionic charge density is negative. 

{\it Use:} File {\tt SystemLabel}.TOCH is only written, not read, by siesta.

{\it Default value:} {\tt .false.}
        

\item[{\bf LocalDensityOfStates}] ({\it block}): 
\index{LocalDensityOfStates@{\bf LocalDensityOfStates}} 
\index{output!local density of states}
Instructs to write the LDOS, integrated between two given energies,
at the mesh used by DHSCF,
in file {\tt SystemLabel}.LDOS. This file can be read by routine IORHO,
which may be used by an application program in later versions.
The block must be a single line with the energies of the range for 
LDOS integration
(relative to the program's zero, i.e. the same as the eigenvalues
printed by the program) and their units.
An example is:

\begin{verbatim}
     %block LocalDensityOfStates
        -3.50  0.00   eV
     %endblock LocalDensityOfStates
\end{verbatim}

{\it Use:} The two energies of the range must be ordered,
with lowest first.
File {\tt SystemLabel}.LDOS is only written, not read, by siesta.

{\it Default value:} LDOS not calculated nor written.
        

\item[{\bf ProjectedDensityOfStates}] ({\it block}):
\index{ProjectedDensityOfStates@{\bf ProjectedDensityOfStates}}
\index{output!projected density of states}

Instructs to write the Total Density Of States (Total DOS) and the 
Projected Density Of States (PDOS) on the basis orbitals,
between two given energies,
in files {\tt SystemLabel}.DOS and 
{\tt SystemLabel}.PDOS, respectively. 
The block must be a single line with the energies of the range for 
PDOS projection,
(relative to the program's zero, i.e. the same as the eigenvalues
printed by the program), the peak width (an energy) for broadening
the eigenvalues, the number of points in the energy window, 
and the energy units.
An example is:

\begin{verbatim}
     %block ProjectedDensityOfStates
        -20.00  10.00  0.200  500  eV
     %endblock ProjectedDensityOfStates
\end{verbatim}

{\it Use:} The two energies of the range must be ordered, with lowest
first. 

{\it Output:} The Total DOS is dumped into a file
called {\tt SystemLabel}.DOS. The format of this file is:

Energy value, Total DOS (spin up), Total DOS (spin down)

The Projected Density Of States for all the orbitals in the unit cell
is dumped sequentially into a file called {\tt SystemLabel}.PDOS. This
file is more structured, so auxiliary tools can process it easily.

In all cases, the units for the DOS are (number of states/eV), and the
Total DOS, $g \left(\epsilon\right)$, is normalized as follows:

\begin{eqnarray}
   \int_{-\infty}^{+\infty} g \left(\epsilon\right) d\epsilon = 
   number \;\; of \;\; basis \;\; orbitals \;\; in \;\;  unit \;\; cell \;\;
   \nonumber \\
\end{eqnarray}


{\it Default value:} PDOS not calculated nor written.

\item[{\bf WriteDenchar}] ({\it logical}):
\index{WriteDenchar@{\bf WriteDenchar}}
\index{output!charge density for DENCHAR code}
Instructs to write information needed by the utility program
DENCHAR (by J. Junquera) to plot the valence charge density
contours. The information is written in file {\tt SystemLabel}.PLD.
 
{\it Use:} File {\tt SystemLabel}.PLD is only written, not read, by siesta.
 
{\it Default value:} {\tt .false.}

\end{description}
        

\subsection{User-provided basis orbitals}

See {\it User.Basis} and {\it User.Basis.NetCDF} descriptors.

\subsection{Pseudopotentials}

The pseudopotentials will be read by {\sc Siesta} from different files, one
for each defined species (species defined either in block
{\bf ChemicalSpeciesLabel}).\index{pseudopotential!files}
The name of the files should be:

{\it Chemical\_label}{\tt .vps} (unformatted) or
{\it Chemical\_label}{\tt .psf} (ASCII)

\noindent 
where {\it Chemical\_label} corresponds to the label defined in the
{\bf ChemicalSpeciesLabel} block.
        
\section{OUTPUT FILES}

\subsection{Standard output} \index{output!main output file}

{\sc Siesta} writes its main output to standard output. 

A brief description follows. See the example cases in the 
siesta/Examples directory for illustration. 

The program starts writing the version of
the code which is used. Then, the input FDF file is dumped into
the output file as is (except for empty lines). The program does
part of the reading and digesting of the data at the beginning
within the {\tt redata} subroutine. It prints some of the information
it digests. It is important to note that it is only part of it,
some other information being accessed by the different subroutines
when they need it during the run (in the spirit of FDF input).
A complete list of the input used by the code can be found at the
end in the file {\tt out.fdf}, including defaults used by the code
in the run.

After that, the program reads the pseudopotentials, factorizes them
into Kleinman-Bylander form, and generates (or reads) the atomic basis
set to be used in the simulation. These stages are documented in the
output file.

The simulation begins after that, the output showing information of
the MD (or CG) steps and the SCF cycles within.  Basic descriptions of
the process and results are presented. The user has the option to
customize it, however,\index{output!customization} by defining
different options that control the printing of informations like
coordinates, forces, $\vec k$ points, etc.  Here is a list of useful
options:
 
\begin{itemize}
\item
{\bf WriteCoorInitial}\index{WriteCoorInitial@{\bf WriteCoorInitial}}
\index{output!atomic coordinates!initial}
for writing the initial atomic coordinates,
\item
{\bf WriteCoorStep}\index{WriteCoorStep@{\bf WriteCoorStep}}
\index{output!atomic coordinates!in a dynamics step}
for writing the atomic coordinates at every step,
\item
{\bf WriteForces}\index{WriteForces@{\bf WriteForces}}
\index{output!forces}
for writing the forces on the atoms at every step,
\item
{\bf WriteKpoints}\index{WriteKpoints@{\bf WriteKpoints}}
\index{output!grid $\vec k$ points}
for writing the coordinates of the $\vec k$ points used for the sampling,
\item
{\bf WriteEigenvalues}\index{WriteEigenvalues@{\bf WriteEigenvalues}}
\index{output!eigenvalues}
for writing the eigenvalues of the Hamiltonian at those $\vec k$ points,
\item
{\bf WriteKbands}\index{WriteKbands@{\bf WriteKbands}}
\index{output!band $\vec k$ points}
for writing the $\vec k$ points used to band-structure plots,
\item
{\bf WriteBands}\index{WriteBands@{\bf WriteBands}}
\index{output!band structure} 
for writing the band structure at those $\vec k$ points,
\item
{\bf WriteWaveFunctions}\index{WriteWaveFunctions@{\bf WriteWaveFunctions}}
\index{output!wave functions} 
for writing the coefficients of the wave functions at
certain $\vec k$ points,
\item
{\bf WriteMullikenPop}\index{WriteMullikenPop@{\bf WriteMullikenPop}}
\index{output!Mulliken analysis}\index{Mulliken population analysis} 
for writing the Mulliken population analysis at different levels of detail.
\end{itemize}

Except for the first one, which is {\tt .true.} by default, the
default of {\sc Siesta} for this options is {\tt .false.} (or 0 for the last)
which means no writing. That gives the short output format.

There is a long output possibility (verbose) defined in {\sc Siesta}, which is
obtained by setting {\bf LongOutput} to {\tt .true.} . It changes 
the default of the previous flags to {\tt .true.} (to 1 for 
Mulliken), with the consequent appearance of the 
corresponding information in the output file.
Of course, the explicit setting of any of these options overrides the
{\bf LongOutput} setting of it.



\subsection{Used parameters}\index{out.fdf@{\tt out.fdf}}
The file {\it out.fdf} contains all the parameters used by {\sc Siesta}
in a given run, both those specified in the input fdf file and
those taken by default. They are written in fdf format, so that
you may reuse them as input directly. Input data blocks are 
copied to the out.fdf file only if you specify the {\it dump} option
for them.


\subsection{Array sizes}
\index{siesta.size@{\tt siesta.size}} \index{memory required}
\index{array sizes}
The file {\it siesta.size} contains the memory required by the
large arrays of most subroutines. Generally, only problem-dependent
arrays are considered, since fixed-size arrays are generally much smaller.


\subsection{Basis}
\index{basis}\index{output!basis}
{\sc Siesta} (and the standalone program {\sc Gen-basis}) 
\index{basis!Gen-basis standalone program}
always generate the files
{\it Atomlabel}{\tt .ion}, where {\it Atomlabel} is the atomic label
specified in block {\it ChemicalSpeciesLabel}.  Optionally, if
the NetCDF support subsystem is compiled in, the programs generate
NetCDF files \index{NetCDF format}
{\it Atomlabel}{\tt .ion.nc}.
See an Appendix for information on the optional NetCDF package.

\subsection{Pseudopotentials}
\index{Pseudopotentials} {\sc Siesta} uses as local pseudopotential a smooth
function up to the core cutoff radius (normally the potential generated
by the core positive charge spread with a gaussian form). The
Kleinman-Bylander pseudopotentials are generated
accordingly. They appear in the {\tt .ion} files.
\index{pseudopotential!writing vlocal} 
\index{pseudopotential!writing KB projectors} 


\subsection{Hamiltonian and overlap matrices}
(file SystemLabel.HS) See the {\bf SaveHS} data descriptor above.


\subsection{Forces on the atoms}
\index{output!forces}
The atomic forces of the last step are stored in the file
SystemLabel.FA if they are not written to the main output.
See the {\bf WriteForces} data descriptor above.


\subsection{Sampling $\vec k$ points}
\index{output!grid $\vec k$ points}
The coordinates of the $\vec k$ points used in the sampling
are stored in the file SystemLabel.KP .
See the {\bf WriteKpoints} data descriptor above.


\subsection{Charge densities and potentials}
(files {\it SystemLabel}.{\tt RHO}, {\it SystemLabel}.{\tt DRHO}, 
{\it SystemLabel}.{\tt VH}, {\it SystemLabel}.{\tt VT}) 
See {\bf SaveRho}, {\bf SaveDeltaRho}, {\bf SaveElectrostaticPotential},
and {\bf SaveTotalPotential} data descriptors above.
3D-plots of these files can be done using the pacakges
{\sc Plrho}\index{Plrho@{\sc Plrho}} and {\sc Grid2cube}
\index{Grid2cube@{\sc Grid2cube}} in the Util/
\index{output!total charge}
directory.


\subsection{Energy bands}
(file SystemLabel.bands) The format of this file is:

\noindent
FermiEnergy (all energies in eV) \\
kmin, kmax (along the k-lines path, i.e. range of k in the band plot) \\
Emin, Emax (range of all eigenvalues) \\
NumberOfBands, NumberOfSpins (1 or 2), NumberOfkPoints \\
k1, ((ek(iband,ispin,1),iband=1,NumberOfBands),ispin=1,NumberOfSpins) \\
k2, ek \\
 . \\
 . \\
 . \\
klast, ek \\
NumberOfkLines \\
kAtBegOfLine1, kPointLabel \\
kAtEndOfLine1, kPointLabel \\
  . \\
  . \\
  . \\
kAtEndOfLastLine, kPointLabel \\

\noindent
The {\sc Gnubands}\index{Gnubands@{\sc Gnubands}} postprocessing utility
program (found in the Util/ directory) reads the {\it Systemlabel}.bands
for plotting.
See the {\bf BandLines} data descriptor above for more information.

\subsection{Wavefunction coefficients}\label{subsec:wf}
(file SystemLabel.WFS) This unformatted file has the
information of the k-points where wavefunctions coefficients
are written, and the energies and coefficients of each
wavefunction which was specified in the input file
(see {\bf WaveFuncKPoints}\index{WaveFuncKPoints@{\bf WaveFuncKPoints}}
descriptor above).
It also writes information on the atomic species and
the orbitals for postprocessing purposes.

\noindent
The READWF\index{READWF} postprocessing utility
program (found in the Util/ directory) reads the {\it Systemlabel}.WFS
file and generates a readable file.

\subsection{Eigenvalues}

\index{output!eigenvalues}
The Hamiltonian eigenvalues for the sampling $\vec k$ points are
dumped into SystemLabel.EIG in a format analogous to SystemLabel.bands,
but without the kmin, kmax, emin, emax information, and without
the abscissa. The {\sc Eig2dos}\index{Eig2dos@{\sc Eig2dos}}
postprocessing utility can be then used to obtain the density of
states.\index{density of states} 
See the {\bf WriteEigenvalues} descriptor above.


\subsection{Coordinates in specific formats}

\begin{itemize}
\item{\bf XMol:}\footnote[1]{XMol is under \copyright\ copyright of Research 
Equipment Inc., dba Minnesota Supercomputer Center Inc.}
\index{XMol@{\sc XMol}}
See {\bf WriteCoorXmol} data descriptor in subsection {\bf Output options}
above for obtaining a {\tt .xyz} file with coordinates in 
XMol-readable format.

\item{\bf CERIUS:}\footnote[2]{CERIUS is under \copyright\ copyright of 
Molecular Simulations Inc.}\index{Cerius2@{\sc Cerius2}}
See {\bf WriteCoorCerius} data descriptor in subsection {\bf Output options}
above for obtaining a {\tt .xtl} file with coordinates in 
CERIUS-readable format. See the {\sc Sies2arc}\index{Sies2arc@{\sc Sies2arc}}
utility in Util/ directory
for generating .arc files for CERIUS animation.

\item{\bf STRUCT\_OUT file:}
\index{Systemlabel.STRUCT\_OUT@{{\it Systemlabel}.STRUCT\_OUT}}
Siesta always produces a {\tt .STRUCT\_OUT} file with cell vectors in {\AA}
and atomic positions in fractional coordinates. This file, renamed to
{\it SystemLabel}.STRUCT\_IN can be used for crystal-structure input.
See {\bf MD.UseStructFile}\index{MD.UseStructFile@{\bf MD.UseStructFile}}.


\end{itemize}


\subsection{Dynamics history files}
\index{output!molecular dynamics!history}
The trajectory of a molecular dynamics run (or a conjugate gradient
minimization) can be accumulated in different files: SystemLabel.MD,
SystemLabel.MDE, and SystemLabel.ANI. The first keeps the whole trajectory
information, meaning updated positions and velocities at every time step, 
including lattice vectors if the cell varies.
The second gives global infomation (energy, temperature, etc),
and the third has the coordinates in a form suited for XMol animation.
See the {\bf WriteMDhistory} and {\bf WriteMDXmol} data descriptors 
above for information. {\sc Siesta} always append new information on these
files, making them accumulative even for different runs.

The {\tt iomd} subroutine can generate both
an unformatted file SystemLabel.MD (default) or  ASCII formatted files
SystemLabel.MDX and SystemLabel.MDC carrying the atomic and lattice
trajectories, respectively. Edit the file to change the settings if desired.

If {\sc Siesta} is compiled with netCDF support, an additional file
SystemLabel.MD.nc is generated. Its structure is currently
experimental and subject to change. Example Python commands for
processing and plotting can be found in {\tt
Util/MD/md.py}.\index{Python language}


\subsection{Force Constant Matrix file}
\index{output!molecular dynamics!Force Constants Matrix}

If the dynamics option is set to the calculation 
of the force constants ({\bf MD.TypeOfRun}={\tt FC}),
the force constants matrix is written in file {\it SystemLabel}.{\tt FC}.
The format is the following: for the displacement of
each atom in each direction, the forces on each of the other
atoms is writen (divided by the value of the displacement),
in units of eV/\AA$^2$. Each line has the forces in the $x$, $y$
and $z$ direction for one of the atoms.

\subsection{PHONON forces file}
\index{output!molecular dynamics!PHONON forces file}

If the dynamics option is set to the calculation of the forces for
selected displacements ({\bf MD.TypeOfRun}={\tt Phonon}, and/or the
block {\tt MD.ATforPhonon} exists), the forces
are written in file {\it SystemLabel}.{\tt PHONON}.  The format is the
following: Comment line, cell vectors in {\AA}, and for each
displacement: atom displaced and its coordinates plus fractional
displacement, cartesian components of forces on all the atoms in units
of eV/\AA. 


\subsection{Intermediate and restart files}

\begin{itemize}
\item {\bf Positions and velocities:}

Every time the atoms move, either during coordinate relaxation or
molecular dynamics, their updated positions and current velocities are
stored in file SystemLabel.XV, where SystemLabel is the value of that
FDF descriptor (or siesta by default).  The shape of the unit cell and
its associated 'velocity' (in Parrinello-Rahman dynamics) are also
stored in this file. For MD runs of type Verlet, Parrinello-Rahman,
Nose, or Nose-Parrinello-Rahman, a file named
SystemLabel.VERLET\_RESTART, SystemLabel.PR\_RESTART,
SystemLabel.NOSE\_RESTART, or SystemLabel.NPR\_RESTART, respectively, is
created to hold the values of auxiliary variables needed for a
completely seamless continuation. Due to the introduction of this
enhanced continuation feature in Siesta 2.0, an MD run made with
Siesta 1.3 cannot be directly restarted with Siesta 2.0: the user
would need to create the right kind of restart file in addition to
setting the MD.UseSaveXV flag in the FDF file. 


\item {\bf Conjugate-gradient history information:}
Together with the SystemLabel.XV file, the information
stored in the SystemLabel.CG file allows a smooth
continuation of an interrupted conjugate-gradient relaxation
process. (No such feature exists yet for a Broyden-based relaxation.)

\item {\bf Localized Wave Functions:} 
At the end of each conjugate gradient
minimization of the energy functional, the LWF's are
stored on disk. These can be used as an input for
the same system in a restart, or in case something goes
wrong.  The LWF's are stored in sparse form in
file SystemLabel.LWF

It is important to keep very good care of this file,
since the first minimizations can take MANY
steps. Loosing them will mean performing the 
whole minimization again. It is also a good practice 
to save it periodically during the
simulation, in case a mid-run restart is necessary.

\item {\bf Density Matrix:} At the end of each SCF cycle
the Density Matrix is stored
disk. These can be used as an input for
the same system in a restart, or in case something goes
wrong.  The DM is stored in sparse form in files
SystemLabel.DM
If the file does not exist, the initial density
matrix is build from the neutral atom charges.

It is important NOT to use a saved DM as an starting
point for a run if the conjugate gradients minimization
which produced the DM file was not highly converged.
Otherwise, the charge density represented by it
could be far from the actual charge density, and
the calculation would most probably not converge.
\end{itemize}


\section{SPECIALIZED OPTIONS}

Experimental or very low-level options.

\begin{description}
\itemsep 10pt
\parsep 0pt

\item[{\bf Reparametrize.Pseudos}] ({\it logical}): 
\index{Reparametrize.Pseudos@{\bf Reparametrize.Pseudos}}

By changing the $a$ and $b$ parameters of the logarithmic grid, a new one
with a more homogeneous overall grid-point separation can be used for
the generation of basis sets and projectors. For example, by using
$a=5x10^{-4}$ and $b=10$, the grid point separations at $r=0$ and 10 bohrs are
0.005 and 0.01 bohrs, respectively. More points are needed to reach r's of
the order of a hundred bohrs, but the extra computational effort is negligible.

\item[{\bf New.A.Parameter}] ({\it real}):
\index{New.A.Parameter@{\bf
    New.A.Parameter}}\index{basis!reparametrization of pseudopotential}

New setting for the pseudopotential grid's $a$ parameter

{\it Default value:} { $5.0x10^{-4}$}

\item[{\bf New.B.Parameter}] ({\it real}):
\index{New.B.Parameter@{\bf
    New.B.Parameter}}\index{basis!reparametrization of pseudopotential}

New setting for the pseudopotential grid's $b$ parameter

{\it Default value:} { $10.0$}

\end{description}

\section{PROBLEM HANDLING}

\subsection{Error and warning messages}

\begin{description}
\itemsep 10pt
\parsep 0pt

\item[{\tt chkdim: ERROR: In {\it routine} dimension {\it parameter} =
{\it value}. It must be  ...}]

And other similar messages.

{\it Description:}
Some array dimensions which change infrequently, and do not lead to
much memory use, are fixed to oversized values. This message means that
one of this parameters is too small and neads to be increased.
However, if this occurs and your system is not very large, or unusual in
some sense, you should suspect first of a mistake in the data file (incorrect
atomic positions or cell dimensions, too large cutoff radii, etc).

{\it Fix:}
Check again the data file. 
Look for previous warnings or suspicious values in the output.
If you find nothing unusual, edit the specified routine and change the 
corresponding parameter.
After running the program, look at the siesta.size file to check that
the memory use is still acceptable.

\end{description}



\subsection{Known but unsolved problems and bugs}

\begin{itemize}

\item
Input (fdf) files with CRLF line endings (the DOS standard) are not
correctly read by {\sc Siesta} on Unix machines. 

{\it Solution:} Please convert to the normal LF-terminated form. This
is easy, running for example: {\tt \$ dos2unix} {\it yourinput.}{\tt
fdf}

\item
$k$-points are not properly generated ({\tt kgrid}) if using a
{\bf SuperCell} block with a non-diagonal matrix. 

{\it Solution:} Make an empty run with {\bf SuperCell} first to generate 
the whole geometry, and then run for the large unit cell (without the 
{\bf SuperCell}) with $k$-points at will.

\item 
For some systems the program stops with the error message

{\tt "Failure to converge standard eigenproblem
Stopping Program from Node:    0}

It is related to the use of the Divide \& Conquer algorithm for
diagonalisation.

{\it Solution:} If it happens, disable Diag.DivideAndConquer and run again. 

\end{itemize}

The following are known problems of the order-{\it N} methods used:

\begin{itemize}

\item
The convergence of the conjugate-gradient minimization of the
electronic energy in the first selfconsistency step (with
{\bf SolutionMethod} = {\tt orderN}) may be extremely slow 
(up to 2000 CG iterations, compared to 20 in further 
selfconsistency steps).

\item
Adjusting the {\bf ON.eta} parameter, so that the total charge
is conserved, may be notably difficult for small-gap systems.

\end{itemize}


\section{PROJECTED CHANGES AND ADDITIONS}

The following are major projected changes and improvements.

\begin{itemize}

\item
Smoothing of the eggbox effect by filtering and alternatively
by moving to atomic grids.

\item
HF and hybrid functionals.

\item
Introduction of {\sc TranSiesta} in Util/ .

\item
QM/MM.

\item
Solution of the Poisson-Boltzman equation for molecules in solution,
using multigrid methods.

\item
Implementation of other linear-scaling solvers.

\item 
An enhanced MD history framework.

\end{itemize}


\section{REPORTING BUGS}
\index{bug reports}
Your assistance is essential to help improve the program. If you find
any problem, please report it back to us through the {\sc Siesta} mailing
list (details in {\tt www.uam.es/siesta}). Please keep in mind
the following guidelines:

\begin{itemize}

\item To be useful, bug reports should be as detailed as possible, yet
concise and to the point.

\item Describe the exact steps you followed to see the problem. You
might want to include a copy of the fdf file you used in the
calculation, details about the pseudopotentials, etc, or provide a
means for us to download the information. (It might be unwise to swamp
all the list users with huge files. State the problem in the most
concise form possible and we will request more info from you.)

\item Be specific. Describe what happened and how it differs from what
should have happened.

\item If you have any idea about how to fix the problem, by all means
tell us!

\item Please make sure that your bug report includes:

\begin{itemize}

\item Your name and email address.  This is essential for a proper
followup of the problem.

\item A brief one-line synopsis of the problem.

\item The {\sc Siesta} version in which the problem was found. We can't
assume that you have the very latest version, and a problem that exists in
one version may not exist in another. Use the version number printed
at the top of any output file (also found in file Src/version.F).

\item The platform on which the problem was found, and the operating
system and compiler version.

\item The {\sc Siesta} mailing list is an open forum.
If for some reason you do not want your report
to be seen by others, please arrange with a developer to look into the
matter directly.

\end{itemize}

\item Please limit your communication to one bug report per form or message.

\end{itemize}

\section{ACKNOWLEDGMENTS}

We want to acknowledge the use of a small number of routines,
written by other authors, in developing the siesta code.
In most cases, these routines were acquired from now-forgotten 
routes, and the reported authorships are based on their headings.
If you detect any incorrect or incomplete attribution, or suspect
that other routines may be due to different authors, please
let us know.

\begin{itemize}
\item
The main nonpublic contribution, that we thank thoroughly, are modified
versions of a number of routines, originally written by {\bf A. R.\ Williams} 
around 1985, for the solution of the radial Schr\"odinger and Poisson 
equations in the APW code of Soler and Williams (PRB {\bf 42}, 9728 (1990)).
Within {\sc Siesta}, they are kept in files arw.f and periodic\_table.f, 
and they are used for the generation of the basis orbitals and the screened
pseudopotentials.

\item
Routine pulayx, used for the SCF mixing, was originally written by
{\bf In-Ho Lee} in 1997.

\item
The exchange-correlation routines contained in file xc.f were written
by J.M.Soler in 1996 and 1997, in collaboration with {\bf C.\ Balb\'as} 
and {\bf J. L.\ Martins}.
Routine pzxc (in the same file), which implements the Perdew-Zunger
LDA parametrization of xc, is based on routine velect, written by
{\bf S.\ Froyen}.

\item
A small number of routines are modified versions of those from 
{\em Numerical Recipes. The Art of Scientific Computing}
by {\bf W. H.\ Press, S. A.\ Teukolsky, W. T.\ Veterling and B. P.\ Flannery}
(Cambridge U.P. 1987-1992), and are kept in file recipes.f

\item
Some standard diagonalization routines by {\bf B. S. Garbow} are kept
in files rdiag.f and cdiag.f. Other diagonalization routines from
the {\bf EISPACK} package are in file eispack.f

\item
The multivariate fast fourier transform in cft.f was written by
{\bf R. C. Singleton} in 1968. It is used to solve Poisson's equation.

\item
Subroutine iomd.f for writing MD history in files was originally written
by {\bf J. Kohanoff}.
\end{itemize}

We want to thank very specially {\bf O. F.\ Sankey, D. J.\ Niklewski} and 
{\bf D. A.\ Drabold} for making the FIREBALL code available to P.\ Ordej\'on. 
Although we no longer use the routines in that code, it 
was essential in the initial development of the {\sc Siesta} project,
which still uses many of the algorithms developed by them.

We thank {\bf V. Heine} for his supporting and encouraging us in this
project. 

The {\sc Siesta} project has been supported by Spanish DGES
through project PB95-0202 and by the Fundaci\'on Ram\'on Areces.


\section{APPENDIX: Physical unit names recognized by FDF}

\begin{center}
\begin{tabular}{llr}
Magnitude & Unit name & MKS value \\
\hline  
mass     & Kg         & 1.E0 \\
mass     & g          & 1.E-3 \\
mass     & amu        & 1.66054E-27 \\
length   & m          & 1.E0 \\
length   & cm         & 1.E-2 \\
length   & nm         & 1.E-9 \\
length   & Ang        & 1.E-10 \\
length   & Bohr       & 0.529177E-10 \\
time     & s          & 1.E0 \\
time     & fs         & 1.E-15 \\
time     & ps         & 1.E-12 \\
time     & ns         & 1.E-9 \\
energy   & J          & 1.E0 \\
energy   & erg        & 1.E-7 \\
energy   & eV         & 1.60219E-19 \\
energy   & meV        & 1.60219E-22 \\
energy   & Ry         & 2.17991E-18 \\
energy   & mRy        & 2.17991E-21 \\
energy   & Hartree    & 4.35982E-18 \\
energy   & K          & 1.38066E-23 \\
energy   & kcal/mol   & 6.94780E-21 \\
energy   & mHartree   & 4.35982E-21 \\
energy   & kJ/mol     & 1.6606E-21 \\
energy   & Hz         & 6.6262E-34 \\
energy   & THz        & 6.6262E-22 \\
energy   & cm-1       & 1.986E-23 \\
energy   & cm**-1     & 1.986E-23 \\
energy   & cm\^-1      & 1.986E-23 \\
force    & N          & 1.E0 \\
force    & eV/Ang     & 1.60219E-9 \\
force    & Ry/Bohr    & 4.11943E-8 \\
\hline
\end{tabular}

\begin{tabular}{llr}
Magnitude & Unit name & MKS value \\
\hline  
pressure & Pa         & 1.E0 \\
pressure & MPa        & 1.E6 \\
pressure & GPa        & 1.E9 \\
pressure & atm        & 1.01325E5 \\
pressure & bar        & 1.E5 \\
pressure & Kbar       & 1.E8 \\
pressure & Mbar       & 1.E11 \\
pressure & Ry/Bohr**3 & 1.47108E13 \\
pressure & eV/Ang**3  & 1.60219E11 \\
charge   & C          & 1.E0 \\
charge   & e          & 1.602177E-19 \\
dipole   & C*m        & 1.E0 \\
dipole   & D          & 3.33564E-30 \\
dipole   & debye      & 3.33564E-30 \\
dipole   & e*Bohr     & 8.47835E-30 \\
dipole   & e*Ang      & 1.602177E-29 \\
MomInert & Kg*m**2    & 1.E0 \\
MomInert & Ry*fs**2   & 2.17991E-48 \\
Efield   & V/m        & 1.E0 \\
Efield   & V/nm       & 1.E9  \\
Efield   & V/Ang      & 1.E10 \\
Efield   & V/Bohr     & 1.8897268E10 \\
Efield   & Ry/Bohr/e  & 2.5711273E11 \\
Efield   & Har/Bohr/e & 5.1422546E11 \\
angle    & deg        & 1.d0 \\
angle    & rad        & 5.72957795E1 \\
torque   & eV/deg     & 1.E0 \\
torque   & eV/rad     & 1.745533E-2 \\
torque   & Ry/deg     & 13.6058E0 \\
torque   & Ry/rad     & 0.237466E0 \\
torque   & meV/deg    & 1.E-3 \\
torque   & meV/rad    & 1.745533E-5 \\
torque   & mRy/deg    & 13.6058E-3 \\
torque   & mRy/rad    & 0.237466E-3 \\
\hline
\end{tabular}
\end{center}

\newpage
\section{APPENDIX: NetCDF}
\index{NetCDF format}
From the NetCDF User's Guide:

\begin{quotation}
   The purpose of the Network Common Data Form (netCDF) interface is to
   allow you to create, access, and share array-oriented data in a form 
   that is self-describing and portable. "Self-describing" means that a
   dataset includes information defining the data it contains. "Portable"
   means that the data in a dataset is represented in a form that can be
   accessed by computers with different ways of storing integers,     
   characters, and floating-point numbers. Using the netCDF interface for
   creating new datasets makes the data portable. Using the netCDF 
   interface in software for data access, management, analysis, and  
   display can make the software more generally useful.

   [...] 

   NetCDF is an abstraction that supports a view of data as a collection
   of self-describing, portable objects that can be accessed through a
   simple interface. Array values may be accessed directly, without
   knowing details of how the data are stored. Auxiliary information
   about the data, such as what units are used, may be stored with the
   data. Generic utilities and application programs can access netCDF
   datasets and transform, combine, analyze, or display specified fields
   of the data. The development of such applications may lead to improved
   accessibility of data and improved reusability of software for
   array-oriented data management, analysis, and display. 

\end{quotation}

In the context of electronic structure calculations, such an interface
is useful to share pseudopotential, wavefunction, and other
files among different computers, regardless of their native floating
point format or their endian-ness. At present, some degree of
transportability can be achieved by using ascii-binary converters.
However, the other major advantage of the
NetCDF format, the self-description of the data and the ease of
accessibility is of great interest also.

\begin{quotation}
   A netCDF dataset contains dimensions, variables, and attributes, which
   all have both a name and an ID number by which they are identified.   
   These components can be used together to capture the meaning of data 
   and relations among data fields in an array-oriented dataset. The
   netCDF library allows simultaneous access to multiple netCDF datasets
   which are identified by dataset ID numbers, in addition to ordinary
   file names.           
\end{quotation}

\index{NetCDF library}
To be able to generate NetCDF files in {\sc Siesta}, the public domain
NetCDF library (V. 3.5 or higher) must be instaled. It can be
downloaded from 

{\tt http://www.unidata.ucar.edu/packages/netcdf/index.html}. 

In the {\tt arch.make} file, the following lines must exist:
\begin{verbatim}
NETCDF_LIBS=-L/path/to/netcdf-3.5/library/directory -lnetcdf
NETCDF_INTERFACE=libnetcdf_f90.a
DEFS_CDF=-DCDF
\end{verbatim}

({\sc Siesta} currently includes an old f90 interface to NetCDF in the
Src/NetCDF directory. Current versions of NetCDF now come with their
own, so that directory will disappear in the near future.)

While it might seem a hassle to have to install the library, the added
functionality is quite large, and it is set to grow in future releases
of {\sc Siesta}. The area in which this is already beginning to be
felt is in the visualization of atomic and molecular dynamics
information. When a NetCDF file is accessed by means of a scripting
language with the proper interface (Python, {\tt
http://www.python.org} is highly recommended in this regard), one can
explore and plot its contents very easily.  See the Utils/PyAtom and Utils/MD
directories for some example scripts.  \index{Python language}


\newpage
\section{APPENDIX: Parallel {\sc Siesta}}
\index{Parallel {\sc Siesta}}

(Note: This feature might not be available in all distributions.)

At present, {\sc Siesta} has been parallelised with moderate system sizes
in mind and is suitable for comensurately moderate parallel computing
systems of the type most widely available. A version suitable for
massively parallel systems in order to tackle grand challenge problems
will hopefully be available in the future.

Apart from the possibility of faster real time performance, there is
another major driving force for the use of the parallel version. All
significant parts of the code have been written using a distributed
data strategy over the Nodes. This means that the use of a parallel
machine can allow access to a larger amount of physical memory.

Given the targets for the present version, the strategy for parallelism
does not employ spatial decomposition since this is only beneficial for
very large problem sizes. Hence the work is divided in 2 ways depending
on the section of the code :

\begin{itemize}
\item
For operations that are orbital based, a 1-D block cyclic distribution
has been used to divide the work over processors. This is controlled
by the parameter {\bf BlockSize}.\index{BlockSize@{\bf Blocksize}}
For optimal performance, this parameter
should be adjusted according to the size of problem and the machine
being used. Very small and very large values tend to be inefficient
and typically values in the range 8 - 32 tend to be optimal. Parts of
the code that parallelise in this way are, evaluation of the kinetic
energy, the non-local pseudopotential contribution, determination of
the overlap integrals and matrix diagonalisation/order N. Note that 
for matrix diagonalisation, the default option is now to transform the
Hamiltonian and Overlap matrices into a 2-D blocked distribution since
this gives better scaling within Scalapack. The 1-D block cyclic
data distribution can be maintained by setting the option {\bf Diag.Use2D}
to false.

\item
For operations that are grid based, a 2-D block cyclic distribution
over mesh points has been used to divide the work. The mesh is divided
in the Y and Z directions, but not currently in the X direction. How
the mesh points are divided is controlled by the {\bf ProcessorY}
\index{ProcessorY@{\bf ProcessorY}}
option which must be a factor of the total number of processors.
Performance will be optimal when the load is balanced evenly
over all processors. For dense bulk materials this is straightforward
to achieve. For surfaces, where there is a region of vacuum, it is
worth ensuring that the mesh is divided so as to ensure that some
processors do not have just vacuum regions. Parts of the code that
parallelise in this way are anything connected to the mesh (i.e. within
DHSCF), including the evaluation of the Hartree and exchange-correlation
energies.
\end{itemize}

There is also a second mode in which the parallel version can be used. For
systems where the number of K points is very large and the size of the
Hamiltonian/Overlap matrices is small, then the work can be parallelised
over K points. This is far more efficient in the diagonalisation step
since this phase becomes embarrassingly parallel once the matrices have
been distributed to each Node. This mode is selected using the 
{\bf ParallelOverK} \index{ParallelOverk@{\bf ParallelOverK}}
option.

The order-N facility of SIESTA has been rewritten for the present version 
to parallelise over spatial regions with a domain decomposition. In the 
ideal situation, each domain interacts with only the neighbouring domains
if the size of the domains is greater than the Wannier function radius and
the range of matrix elements in the Hamiltonian. In order to achieve load
balance, it may be advantageous to use smaller domain sizes. The domain 
size can be controlled through the {\bf RcSpatial} option. 

In the current implementation of the domain decomposition parallelisation,
both the local elements of the orbital coefficients in the Wannier functions
and those connected via the transpose are locally stored on each node in order
to minimise communication. However, this leads to greater demands on the memory
and works best when the system size to processor ratio is high. Work is in 
progress to offer a modified algorithm with higher communication, but lower
memory demands. 

In order to use the parallel version of the code you must have the following
libraries installed on your computer :

\begin{verbatim}

(a) MPI :       The Message Passing Interface library - this allows the 
                processors to communicate. Most machine vendors have their 
                own implementations available for their own platforms. 
                However, there are two freely available versions that can 
                be installed :

                MPICH :

                  http://www-unix.mcs.anl.gov/mpi/mpich/

                LAMMPI :

                  http://www.lam-mpi.org/

(b) Blacs :     This is a communications library that runs on top of MPI. Again
                if can be obtained for free from :

                  http://www.netlib.org/

                Both source code and pre-compiled binaries are available.

(c) Scalapack : This is a parallel library for dense linear algebra, equivalent
                to "lapack" but for parallel systems. Once again this is freely
                available as source code or in precompiled form from :

                  http://www.netlib.org/

\end{verbatim}

\noindent
Parallel versions of the files for {\tt arch.make} suitable for a
number of systems are provided in the {\tt Src/Sys} directory. Should
there be no suitable file there for your system, then the following
are the key variables to be set in the {\tt arch.make} file:

\begin{verbatim}
MPI_INTERFACE=libmpi_f90.a
MPI_INCLUDE=/usr/local/include
DEFS_MPI=-DMPI
#
LIBS= -lscalapack -lblacs -lmpi
\end{verbatim}

Here {\tt MPI\_INTERFACE} indicates that the interface to MPI provided
should be used which handles the issue of the variable type being
passed. This will be needed in nearly all cases. {\tt MPI\_INCLUDE}
indicates the directory where the header file "mpif.h" can be found on
the present machine. The environment variable {\tt DEFS\_MPI} should always
be set to {\tt "-DMPI"}, since this causes the preprocessor to include the
parallel code in the source. Finally {\tt LIBS} must now include all the
libraries required - namely Scalapack, Blacs and MPI, in addition to
any machine optimised Blas, etc.

To execute the parallel version, on most machine, the command will now
be of the form :

{\tt mpirun -np <nproc> siesta < input.fdf > output}

Where {\tt <nproc>} is the desired number of processors, {\tt
input.fdf} is the {\sc Siesta} input file and {\tt output} is the name
of the output file.

Finally, a word concerning performance of parallel execution. This is
a very variable quantity and depends on the exact system you are using
since it will vary according to the latency and bandwidth of the
communication mechanism.  This is a function of the means by which the
processors are physically connected and by software factors relating
to the implementation of MPI. The one almost universal truth is that
for significant system sizes is that parallel diagonalisation becomes
the bottleneck and the place where efficiency is most readily
lost. This is basically just the nature of diagonalisation, but it is
always worth tuning the BlockSize parameter.

\newpage
\section{APPENDIX: XML Output}
\index{XML}
\index{CML}

From version 2.0, {\sc Siesta} includes an option to write its output to an 
XML file. The XML it produces is in accordance with the CML schema version 
2.2 (see http://www.xml-cml.org).

The main motivation for standarised XML (CML) output is as a step
towards standarising formats for uses like the following.

\begin{itemize}

\item To have {\sc Siesta} communicating with other softwares, either
usual postprocessing or as part of a larger workflow scheme. In such a
scenario, the XML output of one {\sc Siesta} simulation may be easily parsed
in order to direct further simulations. Detailed discussion of this is
outwith the scope of this manual.

\item To generate webpages showing {\sc Siesta} output in a more accessible,
graphically rich, fashion. This section will explain how to do this.

\end{itemize}

Prerequisites:

The translation of the {\sc Siesta} XML output to a HTML-based webpage is
done using XSLT technology. The stylesheets conform to XSLT-1.0 plus
EXSLT extensions; an xslt processor capable of dealing with this is
necessary. The following processors have been tested and are known to
work.

\begin{itemize}
  \item \texttt{xsltproc} from libxml2 (\url{http://xmlsoft.org})
  \item \texttt{4xslt} from 4suite (\url{http://4suite.org})
  \item \texttt{saxon} (\url{http://saxon.sourceforge.net/})
\end{itemize}

The generated webpages include support for viewing three-dimensional
interactive images of the system. If you want to do this, you will
also need jMol (\url{http://jmol.sourceforge.net}) installed; as this
is a Java applet, you will also need a working Java Runtime
Environment and browser plugin - installation instructions for these
are outside the scope of this manual, though. However, the webpages
are still useful and may be viewed without this plugin.

XSLT stylesheets are being developed within a CML project, and they
are directly applicable to {\sc Siesta}. They can be found in

{\tt http://www.eminerals.org/siesta/XSLT}

To use these to produce the webpage, the
output of a {\sc Siesta} run, \texttt{SystemLabel.xml} should be taken and
placed in an empty directory. The XSLT processor should then be
run. The necessary invocation differs according to the processor - the
following are some examples; consult the documentation for your own
processor.

\texttt{xsltproc -o SystemLabel.xhtml \$SIESTA/XSLT/display.xsl
SystemLabel.xml}\\ \texttt{4xslt -o SystemLabel.xhtml SystemLabel.xml
\$SIESTA/XSLT/display.xsl}\\ \texttt{java /usr/share/java/saxon.jar -o
SystemLabel.xhtml SystemLabel.xml \$SIESTA/XSLT/display.xsl}

Note that when using saxon, the path to the saxon \.jar may be
different on your system.

After this is done, a set of html files will have been created. Point
your web-browser at \texttt{SystemLabel.xhtml} to view this output.

\newpage
\section{APPENDIX: Selection of precision for storage}
\index{Precision selection}

Some of the real arrays used in Siesta are by default
single-precision, to save memory. This applies to the grid-related
magnitudes, and to the historical data sets in Broyden mixing. The
defaults can be changed by using pre-processing symbols at compile
time:

\begin{itemize}
\item Add {\tt -DGRID\_DP} to the {\tt DEFS} variable in {\tt
  arch.make} to use double-precision arrays on the grid.
\item Add {\tt -DBROYDEN\_DP} to the {\tt DEFS} variable in {\tt
  arch.make} to use double-precision arrays for the Broyden historical
  data sets. (Remember that the Broyden mixing for SCF convergence
  acceleration is an experimental feature.)\index{Broyden mixing} 
\end{itemize}

\addcontentsline{toc}{section}{Index}
\printindex

\end{document}

